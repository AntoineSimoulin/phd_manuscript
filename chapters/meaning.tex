% \setchapterstyle{kao}
\setchapterpreamble[u]{\margintoc}
\chapter{Defining sentence meaning}
\labch{meaning}

\cleanchapterquote{\textup{[\,\dots]}
%But the problem, you see, when you ask why something happens, 
how does a person answer why something happens? For example, Aunt Minnie is in the hospital. Why? Because she went out, slipped on the ice, and broke her hip. That satisfies people. It satisfies, but it wouldn’t satisfy someone who came from another planet and knew nothing about why when you break your hip do you go to the hospital. \textup{[\,\dots]}
% How do you get to the hospital when the hip is broken? Well, because her husband, seeing that her hip was broken, called the hospital up and sent somebody to get her. All that is understood by people. And 
when you explain a why, you have to be in some framework that you allow something to be true. Otherwise, you’re perpetually asking why. 
\textup{[\,\dots]}
%Why did the husband call up the hospital? Because the husband is interested in his wife’s welfare. Not always, some husbands aren’t interested in their wives’ welfare when they’re drunk, and they’re angry.
}{Richard Feynman}{TV program \textit{Fun to Imagine}, 1983}
% https://www.sciencealert.com/watch-richard-feynman-on-why-he-can-t-tell-you-how-magnets-work

\section{The meaning of meaning}

According to \textcite{fromkin_2017}, we can define the meaning of a sentence given our ability to judge it as true or false. Evaluating this boolean value usually requires some context: given other sentences, we deduce whether the sentence is true or false. We may use make the deduction step by step using inductive relations. We define entailment as follows: if sentence A is true, then sentence B is also true. We define paraphrases as two sentences A and B that entail each other.

However, language is ambiguous. For example, given the sentence “The boy saw the man with a telescope”, we may evaluate the sentence are true or false given the same context. Indeed, we may decompose the sentence using distinct structures.

\section{Formal semantic representations}
\labsec{meaning:formal}

\section{Distributional semantic representations}
\labsec{meaning:distributional}

Distributional semantic, vector space