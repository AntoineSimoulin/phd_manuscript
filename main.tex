%%%%%%%%%%%%%%%%%%%%%%%%%%%%%%%%%%%%%%%%%
% kaobook
% LaTeX Template
% Version 1.3 (December 9, 2021)
%
% This template originates from:
% https://www.LaTeXTemplates.com
%
% For the latest template development version and to make contributions:
% https://github.com/fmarotta/kaobook
%
% Authors:
% Federico Marotta (federicomarotta@mail.com)
% Based on the doctoral thesis of Ken Arroyo Ohori (https://3d.bk.tudelft.nl/ken/en)
% and on the Tufte-LaTeX class.
% Modified for LaTeX Templates by Vel (vel@latextemplates.com)
%
% License:
% CC0 1.0 Universal (see included MANIFEST.md file)
%
%%%%%%%%%%%%%%%%%%%%%%%%%%%%%%%%%%%%%%%%%

%----------------------------------------------------------------------------------------
%	PACKAGES AND OTHER DOCUMENT CONFIGURATIONS
%----------------------------------------------------------------------------------------

\documentclass[
	a4paper, % Page size
	fontsize=12pt, % Base font size
	twoside=true, % Use different layouts for even and odd pages (in particular, if twoside=true, the margin column will be always on the outside)
	%open=any, % If twoside=true, uncomment this to force new chapters to start on any page, not only on right (odd) pages
	%chapterentrydots=true, % Uncomment to output dots from the chapter name to the page number in the table of contents
	numbers=noenddot, % Comment to output dots after chapter numbers; the most common values for this option are: enddot, noenddot and auto (see the KOMAScript documentation for an in-depth explanation)
]{kaobook}

% Compile quality PDF/A
% https://www.mathstat.dal.ca/~selinger/pdfa/
\usepackage[a-1b]{pdfx}   % for PDF/A-1b
% \usepackage[x-1a]{pdfx}  % for PDF/X-1a
\usepackage{hyperref}

% Choose the language
\ifxetexorluatex
	\usepackage{polyglossia}
	\setmainlanguage{english}
\else
	\usepackage[english]{babel} % Load characters and hyphenation
\fi
\usepackage[english=british]{csquotes}	% English quotes
% \usepackage{quotchap}

% Load packages for testing
\usepackage{blindtext}
%\usepackage{showframe} % Uncomment to show boxes around the text area, margin, header and footer
%\usepackage{showlabels} % Uncomment to output the content of \label commands to the document where they are used

% Load the bibliography package
\usepackage{kaobiblio} %[backend=bibtex, linkeverything]
\addbibresource{main.bib} % Bibliography file
% \usepackage{natbib} % [authoryear,round,longnamesfirst]

% Add link to title
% Inspired from:
% https://tex.stackexchange.com/questions/23832/biblatex-make-title-hyperlink-to-doi-url-if-available
\ExecuteBibliographyOptions{doi=false}
% \newbibmacro{string+doiurl}[1]{%
%   \iffieldundef{url}{%
%     \iffieldundef{doi}{%
%       #1%
%     }{\href{http://dx.doi.org/\thefield{doi}}{#1}}%
%   }{\href{\thefield{url}}{#1}}%
% }

\newbibmacro{string+doiurlisbn}[1]{%
  \iffieldundef{doi}{%
    \iffieldundef{url}{%
      \iffieldundef{isbn}{%
        \iffieldundef{issn}{%
          #1%
        }{%
          \href{http://books.google.com/books?vid=ISSN\thefield{issn}}{#1}%
        }%
      }{%
        \href{http://books.google.com/books?vid=ISBN\thefield{isbn}}{#1}%
      }%
    }{%
      \href{\thefield{url}}{#1}%
    }%
  }{%
    \href{http://dx.doi.org/\thefield{doi}}{#1}%
  }%
}

% \newbibmacro{string+doiurl}[1]{%
%   \iffieldundef{url}{%
%     \iffieldundef{doi}{%
%       #1%
%     }{%
%       \href{http://dx.doi.org/\thefield{doi}}{#1}}%
%     }%
%   {\href{\thefield{url}}{#1}}%
% }
\DeclareFieldFormat[article]{title}{\usebibmacro{string+doiurlisbn}{#1}}
\DeclareFieldFormat[inproceedings]{title}{\usebibmacro{string+doiurlisbn}{#1}}
\DeclareFieldFormat[incollection]{title}{\usebibmacro{string+doiurlisbn}{#1}}
\DeclareFieldFormat{title}{\usebibmacro{string+doiurlisbn}{#1}}

% Load mathematical packages for theorems and related environments
\usepackage[framed=true]{kaotheorems}

% Load the package for hyperreferences
\usepackage{kaorefs}

\graphicspath{{examples/documentation/images/}{images/}} % Paths in which to look for images

\makeindex[columns=3, title=Alphabetical Index, intoc] % Make LaTeX produce the files required to compile the index

\makeglossaries % Make LaTeX produce the files required to compile the glossary
\input{glossary.tex} % Include the glossary definitions

\makenomenclature % Make LaTeX produce the files required to compile the nomenclature

% Load the packages for tables
\usepackage{booktabs, tabularx}
\usepackage{subcaption}
\usepackage{numprint}
% French
% \npthousandsep{\,}
% English
\npthousandsep{,}
\npthousandthpartsep{}
\npdecimalsign{.}
\newcolumntype{Y}{>{\centering\arraybackslash}X}
\renewcommand\tabularxcolumn[1]{m{#1}}
% for vertical centering text in X column
\usepackage{sistyle}

\usepackage{xspace}
\newcommand{\const}{\textsc{Const}\xspace}
\newcommand{\dep}{\textsc{Dep}\xspace}
\newcommand{\seq}{\textsc{Seq}\xspace}
\newcommand{\bert}{\textsc{Bert}\xspace}
\newcommand{\albert}{\textsc{Albert}\xspace}
\newcommand{\gpt}{\textsc{GPT}\xspace}
\newcommand{\bow}{\textsc{Bow}\xspace}
\newcommand{\cls}{\textsc{Bert-cls}\xspace}
% \DeclarePairedDelimiter\norm{\lVert}{\rVert}
\newcommand{\norm}[1]{\left\lVert#1\right\rVert}

\DeclareMathOperator{\softmax}{softmax}
\DeclareMathOperator{\relu}{ReLU}
\DeclareMathOperator{\lrelu}{LReL}
\DeclareMathOperator{\biaff}{Biaffine}
\DeclareMathOperator{\mlp}{MLP}
\DeclareMathOperator{\mst}{mst}

\usepackage[dvipsnames]{xcolor}

\newcommand{\bcomment}[2]{{\bfseries\color{red} #1} {\bfseries\color{blue} #2}}
\newcommand{\acomment}[1]{{\color{Green} #1}}


\newenvironment{dedi}{\phantom{}\vfill\begin{flushright}\begin{minipage}{1.0\textwidth}\raggedleft}{\end{minipage}\end{flushright}\vfill}


\setcounter{secnumdepth}{3} % seting level of numbering (default for "report" is 3). With ''-1'' you have non number also for chapters
% \setcounter{tocdepth}{5}
% \setcounter{tocdepth}{-1} -- only parts
% \setcounter{tocdepth}{0} -- only parts and chapters
% \setcounter{tocdepth}{1} -- part,chapters,sections
% \setcounter{tocdepth}{2} -- part,chapters,sections, subsections
% \setcounter{tocdepth}{3} -- part,chapters,sections, subsections,subsubsections
% \setcounter{tocdepth}{4} -- part,chapters,sections, subsections,subsubsections and paragraphs
% \setcounter{tocdepth}{5} -- part,chapters,sections, subsections, subsubsections, paragraphs and subparagraphs.

% Reset sidenote counter at chapters
%\counterwithin*{sidenote}{chapter}

%----------------------------------------------------------------------------------------

\begin{document}

%----------------------------------------------------------------------------------------
%	BOOK INFORMATION
%----------------------------------------------------------------------------------------

% \titlehead{The \texttt{kaobook} class}
% \subject{Use this document as a template}

%\title[Example and documentation of the {\normalfont\texttt{kaobook}} class]{Example and documentation \\ of the {\normalfont\texttt{kaobook}} class}
% \title[A study on sentence embeddings and their relation with sentence structure]{A study on sentence embeddings and their relation with sentence structure}
% \title[Sentence embeddings and their relation with sentence structures]{Sentence embeddings and their relation with sentence structures}

% \title[Sentence embeddings and their relation with sentence structures]{
%     \includegraphics[width=6cm]{images/paris-cité.png}
    
%     \huge
%     Université Paris Cité\\
%     \large 
%     Sciences Mathématiques de Paris Centre (ED 386)\\
%     Laboratoire de Linguistique Formelle (LLF)\\[30pt]

% 	\Huge
%     Sentence embeddings and their relation with sentence structures\\[30pt]
    
%     \LARGE 
%     \textmd{Par Antoine Simoulin\\[1cm]
%     \Large
%     Thèse de doctorat d’informatique\\[1cm]
%     Dirigée par Benoit Crabbé \\[20pt]
%     \large 
%     Présentée et soutenue publiquement le 7 juillet 2022\\[20pt]}
    
%     \raggedright\large 
%     \textmd{Devant un jury composé de : \\[0.2cm]
%     \normalsize
%     Claire Gardent, directrice de recherche, CNRS et Université de Lorraine, rapporteuse\\
%     Éric Gaussier, professeur HDR, Université Grenoble Alpes, rapporteur\\
%     Rachel Bawden, chargée de recherche, INRIA, examinatrice\\
%     Loïc Barrault, maitre de conférence, Le Mans Université, examinateur\\
%     Benoit Crabbé, professeur HDR, Université Paris Cité, examinateur\\
%     Nicolas Brunel, maitre de conférence, ENSIIE et Laboratoire de Mathématiques et Modélisation d'Évry, membre invité du jury\\
%   }
% }
% \subtitle{Customise this page according to your needs}

% \author[Antoine Simoulin]{Federico Marotta\thanks{A \LaTeX\ lover}}
% \author[Antoine Simoulin]{Antoine Simoulin}

% \date{\today}
\date{}

% \publishers{An Awesome Publisher}

%----------------------------------------------------------------------------------------

\frontmatter % Denotes the start of the pre-document content, uses roman numerals

%----------------------------------------------------------------------------------------
%	OPENING PAGE
%----------------------------------------------------------------------------------------

% \makeatletter
% \extratitle{
% 	% In the title page, the title is vspaced by 9.5\baselineskip
% 	\vspace*{9\baselineskip}
% 	\vspace*{\parskip}
% 	\begin{center}
% 		% In the title page, \huge is set after the komafont for title
% 		\usekomafont{title}\huge\@title
% 	\end{center}
% }
% \makeatother

%----------------------------------------------------------------------------------------
%	COPYRIGHT PAGE
%----------------------------------------------------------------------------------------

% \makeatletter
% \uppertitleback{\@titlehead} % Header

% \lowertitleback{
% 	\textbf{Disclaimer}\\
% 	You can edit this page to suit your needs. For instance, here we have a no copyright statement, a colophon and some other information. This page is based on the corresponding page of Ken Arroyo Ohori's thesis, with minimal changes.
	
% 	\medskip
	
% 	\textbf{No copyright}\\
% 	\cczero\ This book is released into the public domain using the CC0 code. To the extent possible under law, I waive all copyright and related or neighbouring rights to this work.
	
% 	To view a copy of the CC0 code, visit: \\\url{http://creativecommons.org/publicdomain/zero/1.0/}
	
% 	\medskip
	
% 	\textbf{Colophon} \\
% 	This document was typeset with the help of \href{https://sourceforge.net/projects/koma-script/}{\KOMAScript} and \href{https://www.latex-project.org/}{\LaTeX} using the \href{https://github.com/fmarotta/kaobook/}{kaobook} class.
	
% 	The source code of this book is available at:\\\url{https://github.com/fmarotta/kaobook}
	
% 	(You are welcome to contribute!)
	
% 	\medskip
	
% 	\textbf{Publisher} \\
% 	First printed in May 2019 by \@publishers
% }
% \makeatother

%----------------------------------------------------------------------------------------
%	OUTPUT TITLE PAGE AND PREVIOUS
%----------------------------------------------------------------------------------------

% Note that \maketitle outputs the pages before here

% \maketitle
\vspace{0.1cm}
\includegraphics[width=6cm]{images/paris-cité.png}

\huge
Université Paris Cité\\[0.1cm]
\large 
Sciences Mathématiques de Paris Centre (ED 386)\\[0.1cm]
Laboratoire de Linguistique Formelle (LLF)\\[2cm]

\Huge
Sentence embeddings and their relation with sentence structures\\[2cm]

\LARGE 
\textmd{Par Antoine Simoulin\\[2cm]
\Large
Thèse de doctorat d’informatique\\[1cm]
Dirigée par Benoit Crabbé \\[1cm]
\large 
Présentée et soutenue publiquement le 7 juillet 2022\\[1cm]}

\raggedright\large 
\textmd{Devant un jury composé de : \\[0.2cm]
\normalsize
Claire Gardent, directrice de recherche, CNRS et Université de Lorraine, rapporteuse\\
Éric Gaussier, professeur HDR, Université Grenoble Alpes, rapporteur\\
Rachel Bawden, chargée de recherche, INRIA, examinatrice\\
Loïc Barrault, maitre de conférence, Le Mans Université, examinateur\\
Benoit Crabbé, professeur HDR, Université Paris Cité, examinateur\\
Nicolas Brunel, maitre de conférence, ENSIIE et Laboratoire de Mathématiques et Modélisation d'Évry, membre invité du jury\\}

%----------------------------------------------------------------------------------------
%	DEDICATION
%----------------------------------------------------------------------------------------

% \begin{dedi}
%     \`{A} ma famille,\\
%     \`{A} mes parents qui m'ont construit un monde de bonheur dans lequel j'ai pu grandir \\
%     \`{A} bien des égards, le monde extérieur m'a paru froid et tristre\\
%     \`{A} Eva qui m'a permis de construire mon propre monde de bonheur et de couleurs\\
% \end{dedi}

%----------------------------------------------------------------------------------------
%	PREFACE
%----------------------------------------------------------------------------------------

\chapter*{}
% \addcontentsline{toc}{chapter}{Preface} % Add the preface to the table of contents as a chapter

\paragraph{Titre :} Plongements de phrases et leurs relations avec les structures de phrases

\paragraph{Résumé (court) :} Historiquement, la modélisation du langage humain suppose que les phrases ont une structure symbolique et que cette structure permet d’en calculer le sens par composition. Ces dernières années, les modèles d’apprentissage profond sont parvenus à traiter automatiquement des tâches sans s’appuyer sur une structure explicite du langage, remettant ainsi en question cette hypothèse fondamentale. Cette thèse cherche ainsi à mieux identifier le rôle de la structure lors de la modélisation du langage par des modèles d’apprentissage profonds. Elle se place dans le cadre spécifique de la construction de plongements de phrases—des représentations sémantiques basées sur des vecteurs—par des réseaux de neurones profonds. Dans un premier temps, on étudie l’intégration de biais linguistiques dans les architectures de réseaux neuronaux, pour contraindre leur séquence de composition selon une structure traditionnelle, en arbres. Dans un second temps, on relâche ces contraintes pour analyser les structures latentes induites par ces réseaux neuronaux. Dans les deux cas, on analyse les propriétés de composition des modèles ainsi que les propriétés sémantiques des plongements. La thèse s’ouvre sur un état de l’art présentant les principales méthodes de représentation du sens des phrases, qu’elles soient symboliques, ou basées sur des méthodes d’apprentissage profond. La deuxième partie propose plusieurs expériences introduisant des biais linguistiques dans les architectures des réseaux de neurones pour construire des plongements de phrases. Le premier chapitre combine explicitement plusieurs structures de phrases pour construire des représentations sémantiques. Le deuxième chapitre apprend conjointement des structures symboliques et des représentations vectorielles. Le troisième chapitre introduit un cadre formel pour les transformers selon une structure de graphes. Finalement, le quatrième chapitre étudie l’impact de la structure vis-à-vis de la capacité de généralisation et de compositions des modèles. La thèse se termine par une mise en concurrence de ces approches avec des méthodes de passage à l’échelle. On cherche à y discuter les tendances actuelles qui privilégient des modèles plus gros, plus facilement parallélisables et entraînés sur plus de données, aux dépens de modélisations plus fines. Les deux chapitres de cette partie relatent l'entraînement de larges modèles de traitement automatique du langage et comparent ces approches avec celles développées dans la deuxième partie d’un point de vue qualitatif et quantitatif.

\paragraph{Résumé (long) :} Historiquement, la modélisation du langage humain suppose que les phrases ont une structure symbolique et que cette structure permet d’en calculer le sens par composition. Ces dernières années, les modèles d’apprentissage profond sont parvenus à traiter automatiquement des tâches sans s’appuyer sur une structure explicite du langage, remettant ainsi en question cette hypothèse fondamentale. Cette thèse cherche ainsi à mieux identifier le rôle de la structure lors de la modélisation du langage par des modèles d’apprentissage profonds. Elle se place dans le cadre spécifique de la construction de plongements de phrases—des représentations sémantiques basées sur des vecteurs—par des réseaux de neurones profonds. Dans un premier temps, nous étudions l’intégration de biais linguistiques dans les architectures de réseaux neuronaux, pour contraindre leur séquence de composition selon une structure traditionnelle, en arbres. Dans un second temps, nous relâchons ces contraintes pour analyser les structures latentes induites par ces réseaux neuronaux. Dans les deux cas, nous analysons les propriétés de composition des modèles ainsi que les propriétés sémantiques des plongements. 

La thèse s’ouvre sur un état de l’art présentant les principales méthodes de représentation du sens des phrases, qu’elles soient symboliques, ou basées sur des méthodes d’apprentissage profond. La deuxième partie propose plusieurs expériences introduisant des biais linguistiques dans les architectures des réseaux de neurones pour construire des plongements de phrases. 

Le premier chapitre combine explicitement plusieurs structures de phrases pour construire des représentations sémantiques. Nous supposons que la signification d'une phrase soit une fonction des aspects syntaxiques et sémantiques. À cet égard, nous proposons une méthode auto-supervisée qui construit des plongements de phrases à partir de la combinaison de diverses structures syntaxiques explicites. La nouveauté consiste à apprendre conjointement des modèles structurés dans un cadre multi-vues contrastif qui induit une interaction explicite entre les modèles pendant la phase d'apprentissage. Nous pré-entraînons plusieurs modèles en utilisant un objectif contrastif avec un corpus de 40 millions de phrases. Nous évaluons ensuite nos modèles sur des ressources d’évaluation des plongements de phrases et obtenons des résultats à l’état de l’art. En particulier sur des tâches qui devraient, par hypothèse, être plus sensibles à la structure des phrases. 

Le deuxième chapitre apprend conjointement des structures symboliques et des représentations vectorielles. Nous utilisons des réseaux neuronaux structurés en arbre, qui encodent naturellement la structure du langage. Pour chaque phrase, le réseau encode les unités de texte en suivant un arbre syntaxique, en partant des feuilles jusqu'à la racine. Cependant, ces modèles souffrent de contraintes pratiques qui limitent leur application. En particulier, les modèles structurés nécessitent non seulement du texte brut en entrée mais aussi la structure de la phrase sous la forme d'un arbre. Une telle structure nécessite des annotations dans le cadre supervisé. Nous formulons un nouveau modèle structuré en arbre qui apprend sa fonction de composition en même temps que sa structure. Le modèle comprend deux modules, un analyseur de graphe biaffine et un Tree-LSTM. Les fonctions d'analyse syntaxique et de composition sont explicitement connectées et, par conséquent, apprises conjointement. La méthode diffère des travaux précédents car la représentation n'est pas calculée à partir d’une forêt entière d'arbres potentiels. De plus, l'apprentissage du modèle ne nécessite pas de supervision directe pour la structure. Le modèle est plus performant que les modèles à base d'arbres reposant sur des structures extrinsèques. Dans certaines configurations, il est même compétitif avec Bert.

Le troisième chapitre introduit un cadre formel pour les transformers selon une structure de graphes. Les architectures de transformers ont gagné en popularité au sein de la communauté. Contrairement aux modèles basés sur des arbres, ils n'ont pas besoin de données annotées pour être entraînés. D'autre part, comme le suggèrent de nombreux résultats, ces nouveaux modèles acquièrent une sorte de structure hiérarchique. Les transformers mettent à jour chaque jeton caché simultanément à travers un nombre fixe de couches. Pourtant, le rôle de ces couches et la façon dont elles traitent l'information ne sont pas entièrement compris. Nous formulons l'hypothèse que les couches distinctes ne codent pas des fonctions spécifiques de surface, syntaxiques ou sémantiques mais plutôt que de telles informations émergent par l'application itérative des couches. Pour mieux étudier la transformation des représentations des tokens à travers les couches, nous proposons une variante du modèle ALBERT. Ce modèle implémente la spécificité clé de la liaison des poids entre les couches mais adapte aussi dynamiquement le nombre de couches appliquées à chaque jeton. Nous analysons la transformation des jetons à travers la profondeur du réseau. En particulier, nous étudions comment les itérations sont distribuées en fonction des types de dépendance des jetons. Nous montrons que les jetons ne nécessitent pas le même nombre d'itérations et que les jetons difficiles ou cruciaux pour la tâche sont soumis à plus d'itérations.

Finalement, le quatrième chapitre étudie l’impact de la structure vis-à-vis de la capacité de généralisation et de compositions des modèles. Bien que les transformers affichent des performances excellentes sur de nombreux benchmarks NLP, ils présentent également certaines limites. En particulier en ce qui concerne leur capacité à généraliser en dehors de leur domaine d'entraînement et à apprendre des règles de composition élémentaires. Le benchmark COGS par exemple, met en évidence que les modèles d'apprentissage profond ont du mal à généraliser à des séquences plus longues ou à des phrases présentant un niveau de récursion plus profond que celui vu pendant l'entraînement. Suite à nos travaux sur l'intégration de la structure dans l'architecture neuronale, nous cherchons à mieux caractériser le rôle de la structure dans les propriétés compositionnelles des modèles. Ce travail est actuellement en phase d'expérimentation. Nous construisons une méthode d'évaluation avec des expressions arithmétiques contenant des propriétés spécifiques. Nous entraînons différents modèles sur des sous-ensembles du jeu de données et observons comment les modèles généralisent en dehors de leur domaine. Nous comparons des modèles avec des contraintes structurelles variables, comme des modèles séquentiels, récursifs ou non structurés.

La thèse se termine par une mise en concurrence de ces approches avec des méthodes de passage à l’échelle. Dans cette seconde partie, on cherche à discuter les tendances actuelles qui privilégient des modèles plus gros, plus facilement parallélisables et entraînés sur plus de données, aux dépens de modélisations plus fines. Les deux chapitres de cette partie relatent l'entraînement de larges modèles de traitement automatique du langage et comparent ces approches avec celles développées dans la deuxième partie d’un point de vue qualitatif et quantitatif. 

Le premier chapitre s’interroge sur la taille des modèles. à première vue, il semble que le traitement actuel du langage naturel évolue vers des modèles de plus en plus gros aux dépends des subtilités de leur architecture. Cette tendance n’a pas directement profité aux modèles de plongements de phrases, car de nombreux encodeurs à base de transformers affichent des performances inférieures à l'état de l'art sur les bancs d'essai standard. Dans cette section, nous explorons comment nous pouvons tirer parti des performances des encodeurs de phrases de grande taille en adaptant leur pré-entraînement et en augmentant leur taille. Nous détaillons le développement, l'entraînement et le partage de modèles de plongements de phrases à l’état de l’art. Nous utilisons un objectif contrastif et entraînons les modèles sur un corpus d'un milliard de phrases. 

Le second chapitre de cette partie s’intéresse à l’entrainement d’un modèle de langue incrémental en français. Ce type de modèle peut aquérir des compétences grammaticales très impressionnantes. Par exemple, GPT-2 génère un texte correct avec un accord au pluriel et à distance, et ce, en dépit toute connaissance linguistique préalable. Ces accords sont pourtant déterminés par des structures abstraites et pas seulement par l'ordre linéaire des mots. Plus largement, les modèles peuvent apprendre de nombreux motifs linguistiques (sujet-verbe, nom-adverbe, verbe-verbe) sans aucune information préalable sur la théorie linguistique. Au sein de notre laboratoire, nous avons dirigé le projet d'entraînement du premier grand modèle de langage en français. Nous avons obtenu une subvention de calcul dédié pour le calculateur public français Jean Zay. Le modèle, équivalent à GPT-2 en anglais, contient plus d'un milliard de paramètres. Nous construisons un corpus d'entraînement dédié et parallélisons l'entraînement entre plusieurs nœuds et unités de calcul. Nous avons publié le modèle en Open-Source pour la recherche et les applications commerciales.

En conclusion, nous avons abordé plusieurs problèmes critiques des plongements de phrases, notamment le manque de robustesse vis-à-vis de la généralisation hors du domaine, le manque de propriétés de compositions, la nécessité de vastes corpus d'entraînement ou la sur paramétrisation. Étant donné ces problèmes, nous avons proposé les contributions suivantes : premièrement, nous avons donné des éléments empiriques montrant que certaines structures de réseaux neuronaux sont plus appropriées pour capturer des types d'informations spécifiques.  Deuxièmement, nous avons proposé des architectures originales pour apprendre conjointement la structure de la phrase et la fonction de composition sémantique. Troisièmement, nous avons adapté la méthode standard de pré-entraînement contrastive pour entraîner des modèles de transformers de grande taille sur un grand ensemble de données. Enfin, nous avons développé des ressources d'évaluation. 

\paragraph{Mots clefs :} Traitement automatique des langues naturelles, plongements de phrases, apprentissage profond, réseaux de neurones structurés.


\newpage

\paragraph{Title:} Sentence embeddings and their relation with sentence structures

\paragraph{Abstract:} Historically, models of human language assume that sentences have a symbolic structure and that this structure allows us to compute their meaning by composition. In recent years, deep learning models have successfully processed tasks automatically without relying on an explicit language structure, thus challenging this fundamental assumption. This thesis seeks to clearly identify the role of structure in language modeling by deep learning methods. The dissertation specifically investigates the construction of sentence embeddings—semantic representations based on vectors—by deep neural networks. Firstly, we study the integration of linguistic biases in neural network architectures to constrain their composition sequence based on a traditional tree structure. Secondly, we relax these constraints to analyze the latent structures induced by the neural networks. In both cases, we analyze the compositional properties of the models as well as the semantic properties of the sentence embeddings. This thesis begins with an overview of the main methods used to represent the meaning of sentences, either symbolically or using deep learning. The second part proposes several experiments introducing linguistic biases in neural network architectures to build sentence embeddings. The first chapter explicitly combines numerous sentence structures to build semantic representations. The second chapter jointly learns symbolic structures and vector representations. The third chapter introduces a formal framework for graph transformers. Finally, the fourth chapter studies the impact of the structure on the generalization capacity of the models and compares their compositional capabilities. The last part compares the models to larger-scale approaches. It seeks to discuss current trends favoring larger models, more easily parallelized and trained on more data, at the expense of finer modeling. The two chapters from this part report on the training of large models of automatic language processing and compare these approaches with those developed in the second part from a qualitative and quantitative point of view.

\paragraph{Keywords:} Natural language processing, sentence embeddings, deep learning, structured neural networks.

% I am of the opinion that every \LaTeX\xspace geek, at least once during 
% his life, feels the need to create his or her own class: this is what 
% happened to me and here is the result, which, however, should be seen as 
% a work still in progress. Actually, this class is not completely 
% original, but it is a blend of all the best ideas that I have found in a 
% number of guides, tutorials, blogs and tex.stackexchange.com posts. In 
% particular, the main ideas come from two sources:

% \begin{itemize}
% 	\item \href{https://3d.bk.tudelft.nl/ken/en/}{Ken Arroyo Ohori}'s 
% 	\href{https://3d.bk.tudelft.nl/ken/en/nl/ken/en/2016/04/17/a-1.5-column-layout-in-latex.html}{Doctoral 
% 	Thesis}, which served, with the author's permission, as a backbone 
% 	for the implementation of this class;
% 	\item The 
% 		\href{https://github.com/Tufte-LaTeX/tufte-latex}{Tufte-Latex 
% 			Class}, which was a model for the style.
% \end{itemize}

% The first chapter of this book is introductory and covers the most
% essential features of the class. Next, there is a bunch of chapters 
% devoted to all the commands and environments that you may use in writing 
% a book; in particular, it will be explained how to add notes, figures 
% and tables, and references. The second part deals with the page layout 
% and design, as well as additional features like coloured boxes and 
% theorem environments.

% I started writing this class as an experiment, and as such it should be 
% regarded. Since it has always been intended for my personal use, it may
% not be perfect but I find it quite satisfactory for the use I want to 
% make of it. I share this work in the hope that someone might find here 
% the inspiration for writing his or her own class.

% \begin{flushright}
% 	\textit{Federico Marotta}
% \end{flushright}

\index{preface}

%----------------------------------------------------------------------------------------
%	TABLE OF CONTENTS & LIST OF FIGURES/TABLES
%----------------------------------------------------------------------------------------

\begingroup % Local scope for the following commands

% Define the style for the TOC, LOF, and LOT
%\setstretch{1} % Uncomment to modify line spacing in the ToC
%\hypersetup{linkcolor=blue} % Uncomment to set the colour of links in the ToC
\setlength{\textheight}{230\hscale} % Manually adjust the height of the ToC pages

% Turn on compatibility mode for the etoc package
\etocstandarddisplaystyle % "toc display" as if etoc was not loaded
\etocstandardlines % "toc lines" as if etoc was not loaded

\tableofcontents % Output the table of contents

% \listoffigures % Output the list of figures

% Comment both of the following lines to have the LOF and the LOT on different pages
% \let\cleardoublepage\bigskip
% \let\clearpage\bigskip

% \listoftables % Output the list of tables

\endgroup

%----------------------------------------------------------------------------------------
%	MAIN BODY
%----------------------------------------------------------------------------------------

%TODO mettre la liste des publications et open-source (contributions)
\mainmatter % Denotes the start of the main document content, resets page numbering and uses arabic numbers
\setchapterstyle{kao} % Choose the default chapter heading style

% \setchapterpreamble[u]{\margintoc}
\chapter{Introduction}
\labch{intro}

\cleanchapterquote{Language is a process of free creation; its laws and principles are fixed, but the manner in which the principles of generation are used is free and infinitely varied. Even the interpretation and use of words involves a process of free creation.}{Noam Chomsky}{For reasons of state, 1973}
% p 402. Chapter 9, Language and Freedom
% \bcomment{}{this quote is suspect}
% QUOI

%\bcomment{Suggestion (sth along these lines)}{Language has structure. Sentences have discrete structures, such as trees. These are fundamental hypothesis of linguistic theory since its early days. Recently, in computational linguistics emerged a new family of methods that model language with vectors. First emerged word embeddings \cite{mikolov_13b}, then deep learning truly changed the field and it becomes natural to model sentences or even longer texts with vector representations. In this context, this dissertation is about\ldots}

% An axiom, postulate, or assumption is a statement that is taken to be true, to serve as a premise or starting point for further reasoning and arguments. 
% The structure of language is a fundamental axiom, serving as a premise in linguistic theory.
% The hypothesis that language has a structure have been the cornerstones of linguistic theory since its early days.
% These are fundamental hypothesis of linguistic theory since its early days.
% There is this strong hypothesis in computational linguistics that language has a recursive structure . This hypothesis 

Linguistic theory is founded on the hypothesis that language has a structure. In computational linguistics, a strong premise is that this structure is recursive \parencite{chomsky_56} and, in the specific case of sentences, this structure forms a tree. These premises are the cornerstone of linguistic theory. Recently, a new family of methods truly changed the field of computational linguistics by modeling language with vectors. First, word embeddings emerged \parencite{mikolov_13a, mikolov_13b} and, as deep learning gained momentum, it soon became natural to model sentences or even longer texts with vector representations \parencite{hochreiter_97, cho_14}. In this context, this dissertation is about creating sentence embeddings through the composition of lexical units. Text representation is at the core of natural language processing (NLP), which develops automatic methods for inferring related attributes from those representations. Attributes can take many forms: a given class in a classification problem, the answer to a question, a list of documents with similar semantic content, or a summary of the input text. In recent years, the representation methods for text have developed significantly. Contrary to formal linguistic frameworks, which derive syntactic and semantic properties from expert rules, these methods derive representations by exploiting the implicit patterns within vast corpora. These methods are grounded on two foundational hypotheses: the distributional hypothesis to build word representations given their context and the compositionality principle to combine those words into sentence representations.


% \bcomment{}{distributional hyp = words and their context only}

This dissertation focuses on sentences and the methods to encode them. Specifically, \textit{(i)} how layouts identified by distributional methods from vast corpora relate to linguistic structures and, respectively, \textit{(ii)} how we can efficiently infuse linguistic biases in neural architectures to drive the composition function learned by self-supervised sentence embedding methods.

This section introduces the study by first discussing the applications of sentence embedding methods, the background, and context, followed by the research problem, the research aims, objectives and questions, the significance, and finally, the limitations.

% POURQUOI

\section{Background to the study}
\labsec{introduction:backgroud}

Embedding a sentence consists of assigning it to a static, fixed-length, real-valued vector, which captures its meaning. It is important to emphasize the distinction between sentence embedding and any sentence vector representation, for example, intermediate representations from neural networks. The majority of modern NLP methods works end-to-end: the intermediate representations and the inference of attributes from those representations are part of a unified process. In such a case, it is impossible to separate the representations from the final model outputs. The representations, therefore, depend on the attribute we seek to predict and will most likely only capture the information relevant for this prediction. In the context of sentence embedding, the representation of the input text and the inference of related attributes are two explicitly disconnected steps. Therefore, sentence embedding should capture an exhaustive perspective of the text input meaning as we may use them to predict a large variety of attributes. Sentence embedding methods should be highly generic, and their conception should be independent of their later use.

% \bcomment{what properties should sentence embeddings have}{sentences with similar meanings should have similar embeddings ? }{this is not stated explictly in the thesis}
% As a result of this transformation, we can compare sentence semantic characteristics directly in the sentence embedding space using standard mathematical operators. For example, it is straightforward to define a notion of mathematical distance over the vector space and to characterize sentences with close meanings as sentences for which embedding vectors are close given this distance.
% It is also possible to use sentence embedding as features for more complex models, inferring relations such as entailment between sentence pairs. Sentence embeddings are, therefore, key for a number of applications such as search engines, information retrieval, text mining, and documents clustering.
We can divide the properties we expect from sentence embeddings into two categories:
\begin{itemize}
    \item First, the notion of semantic distance in the original sentence space should be reflected in the representation space. In the sentence embedding space, it is straightforward to define a notion of mathematical distance over the vector space. Therefore, we can use standard mathematical operators to compare semantic sentence characteristics directly in the sentence embedding space. Sentences with close meanings should be mapped to close embedding vectors.
    \item Second, we expect sentence embeddings to fully capture the meaning and general characteristics—such as the sentence length or the main verb tense—of the original sentence. It should be possible to extract some specific information from the embedding vector using statistical methods. 
\end{itemize}
Therefore, sentence embeddings are essential for many unsupervised applications such as search engines, information retrieval, text mining, and documents clustering. It is also possible to use sentence embeddings as features for more supervised models, inferring relationships such as entailment between sentence pairs. 

% COMMENT

\section{Research problem}

Embedding sentences is an active subject of research, including the development of self-supervised training objectives, training datasets, evaluation benchmarks, or the release of models as open-source contributions.

Building standalone sentence embeddings is specifically hard, as an infinite number of valid sentences exist. Compositional semantics state that the meaning of a phrase is determined by combining the meanings of its sub-phrases. Models, therefore, need to compose text units, given a syntactic structure, into global semantic embeddings. However, many contributions rely on standard encoder architectures and do not question the composition mechanisms transforming text units into a global sentence representation.

Thus, sentence embedding methods present pitfalls that are common to many domains in NLP: lack of robustness toward out-of-domain generalization, shallow pattern matching rather than compositional knowledge, the requirement for large training datasets, or over-parametrization. 

\section{Research aims, objectives, and questions}

This study aims at improving sentence encoder compositional abilities. We seek to leverage both the integration of linguistic biases into neural network architectures as well as the scaling of these models and their training setup.
We define the following research objectives:

\begin{enumerate}
    \item Develop efficient methods to integrate linguistic biases into neural networks;
    \item Evaluate the effectiveness of these strategies and approaches;
    \item Compare and contrast these strategies and approaches in terms of their strengths and weaknesses.
\end{enumerate}

% \bcomment{too many bullets around}{make paragraphs} The dissertation will address the following specific research questions:
% \begin{enumerate}
%     \item Can we efficiently introduce linguistic biases within neural network architectures? 
%     \item Does self-supervised training induce structure into neural architecture?
%     \item Can we induce specific compositional abilities through neural architectures?
%     \item Can we balance the lack of linguistic insights with larger models and larger training datasets?
% \end{enumerate}

The dissertation studies sentence embeddings and their relation with sentence structures. Below, we detail the research questions we investigate. The three first questions focus on the methods to efficiently induce linguistically driven insights within neural network composition functions. The last question asks about the benefits of such insights regarding the generalization power of neural networks for automatic language processing.

\paragraph{Can we efficiently introduce linguistic biases within neural network architectures?} The recursive structure of language is a strong hypothesis in computational linguistics \parencite{chomsky_56}. Thus, computing sentence semantic representations traditionally calls for a recursive compositional function whose structure is tree-shaped. In contrast, recent deep learning architectures—such as recurrent neural networks \parencite{hochreiter_97, cho_14} or transformers \parencite{vaswani_17}—encode text without explicit hierarchical composition. The first research question focuses on bridging the gap between these two paradigms: we explore the feasibility of explicitly integrating linguistic priors within neural architectures to compose semantic representations with a hierarchical structure.

\paragraph{Does self-supervised training induce structure into neural architecture?}

Our second research question has the same objective—integrating linguistic priors within neural architectures—but different means. This time we are not focusing on the architectures of neural networks but rather the training methods. We explore the possibility of inducing latent structure within the function that neural networks operates to compose lexical units into sentence representations.

\paragraph{Can we induce specific compositional abilities through neural architectures?}

We explore the possibility of exploiting the model and training dataset size to induce linguistic structure into neural networks. While the size of the language model in natural language processing is steadily increasing \parencite{devlin_19, brown_20}, we investigate how such approaches can compensate for the lack of linguistically based insights.

\paragraph{Can we balance the lack of linguistic insights with larger models and larger training datasets?}

Finally, we investigate the role of structure in building more robust neural network architectures. By influencing the semantic composition of neural networks, we aim to improve their compositional and generalization abilities.

\section{Significance}

% \bcomment{You might also remind the reader that you are funded by a company and that these real world achievements are also motivated by the desire of acheiving industrial contributions}{}

We expect this study to contribute to the body of knowledge on sentence embeddings and neural model architectures to encode text. The publications and the ongoing experiments will contribute to the academic effort in building more robust statistical models by incorporating language biases and approaches for scaling model training. 

This thesis is funded by Quantmetry\sidenote{\url{https://www.quantmetry.com/}}, a French pioneering consulting firm working on end-to-end AI projects—from strategy to industrialization. As such, this work is also motivated by the desire of achieving industrial contributions and ready-to-use tools available for real-world applications. Besides empirical research, a large portion of this work is thus released as open-source contributions. Resources include pre-trained language models for English and French\sidenote{\url{https://huggingface.co/asi/gpt-fr-cased-base}}, training and evaluation datasets, as well as associated scripts to reproduce the results. Finally, we released a code for recursive models under a library called PyTree\sidenote{\url{https://github.com/AntoineSimoulin/pytree}}. The library was distinguished and listed among the winners of the PyTorch Hackathon 2021. We hope this empirical work and the resources will provide real-world value for organizations in a field in which knowledge and methods are undergoing rapid and continuous evolution.

\section{Limitations}

%  \bcomment{and may, in some cases,}{it is unknown to us if it} extend to paragraphs
Our study is limited to sentences, but we hypothesize that it may, in some cases, extend to paragraphs. However, the major part of our work will not apply to longer chunks of text. Although we seek to propose methods applicable to various languages, the study focuses mainly on English and French, and some experiments may be difficult to reproduce in low-resource languages. Indeed, we make use of specific training and evaluation resources.

Our study proposes efficient methods to introduce linguistic biases into neural models and better characterize model compositional behavior. Such approaches appear promising to avoid unwanted behavior for real-world applications. However, our study also underlines the long road ahead to fully realize the promises of current language models.

\section{Contributions and Outline}

\subsection{Jointly learning model structure and compositional operations}

First, we focus on tree-structured neural networks, which naturally encode the structure of language. For each sentence, the network computes text units following a syntactic tree, starting from the leaf nodes up to the root. However, such models suffer from practical constraints that limit their application. In particular, tree-based models not only require raw text as input but also the sentence structure in the form of a parse tree. Such structure may be tedious because it requires manual annotations and external parsers. To overcome such limitations, we formulated a novel tree-based model that learns its composition function together with its structure. The model includes two modules, a biaffine graph parser, and a Tree-LSTM. The parsing and the composition functions are explicitly connected and, therefore, learned jointly. The method differs from previous work as the representation is not computed from the whole forest of potential trees. Moreover, training the full model directly does not require supervision from an explicit parsing objective. The model outperforms tree-based models relying on external parsers on downstream tasks. In some configurations, it is even competitive with BERT-base.

\subsection{Studying shallow structure in transformer models}

Recent transformer architectures have gained increased popularity within the community. Contrary to tree-based models, they do not need carefully hand-annotated data to be trained. On the other hand, as many results suggest, these new models acquire some sort of hierarchical structure. Transformers update each token hidden simultaneously through a fixed number of layers. Yet, the role of these layers and how they process information is not fully understood. We formulate the hypothesis that the distinct layers do not encode specific surface, syntactic nor semantic functions but rather that such information emerges through the iterative application of layers. To better study the transformation of token representations across layers, we propose a variant of ALBERT \parencite{simoulin_2021b}. This model implements the key specificity of weights tying across layers but also dynamically adapts the number of layers applied to each token. We analyze token transformation across the network depth. In particular, we study how iterations are distributed given the token dependency types. We show that tokens do not require the same amount of iterations and that difficult or crucial tokens for the task are subject to more iterations.

\subsection{Characterizing compositional properties of neural architectures}

While transformers show outstanding performances on many NLP benchmarks, they also have some linguistic limitations. In particular regarding their ability to generalize outside their training range and to learn elementary composition rules. The benchmark COGS \parencite{kim_20} for example, highlights that deep learning models struggle to generalize to longer sequences or sentences with deeper level of recursion than seen during training. Following our work on integrating the structure into neural architecture, we aim at better characterizing how the model structure may affect its degree of compositionality. This work is currently in an experimentation phase. We are building an evaluation setup with arithmetic expressions containing specific properties. We train various models on specific subsets and observe how models generalize outside their domain. Specifically, we compare models with varying structural constraints, such as sequential, recursive, or unstructured models.

\subsection{Training sentence embedding models using a discriminative objective}

Inspired by linguistic insights, we assume structure is crucial to building consistent representations. We indeed expect sentence meaning to be a function of both syntax and semantic aspects. In that regard, we propose a self-supervised method that builds sentence embeddings from the combination of diverse explicit syntactic structures of a sentence [4]. The novelty consists in jointly learning structured models in a contrastive multi-view framework that induces an explicit interaction between models during the training phase. We pre-train various models using a contrastive objective with a 40 million sentences corpus. We then evaluate our model on sentence embedding benchmarks and obtain state-of-the-art results. In particular on tasks that are expected, by hypothesis, to be more sensitive to sentence structure. We relate the development, training, and release of large, state-of-the-art, sentence embedding models. We use a similar contrastive objective and train models on a 1 billion sentences corpus. We develop specific evaluation benchmarks for sentence embeddings and obtain state-of-the-art results.

\subsection{Training the first large incremental language model for French}

As observed in \textcite{linzen_2020}, deep neural networks have exceptional grammatical competencies. For example, GPT-2 generates correct text with plural and long-distance agreement despite any prior linguistic knowledge. Such agreements are determined by abstract structures and not just the linear order of words. Surprisingly, models can learn such specific linguistic patterns (subject-verb, noun-adverb, verb-verb) with no prior information about linguistic theory. Within our laboratory, we led the project to train the first large language model in French \parencite{simoulin_2021c}. We obtained a dedicated computation grant for the public French HPC computer Jean Zay. The model, equivalent to GPT-2 in English, contains more than 1 billion parameters. We build a dedicated training corpus and parallelize the training between multiple nodes and compute units. We released the model in Open-Source for research and business application purposes. 

\subsection{Outline of the dissertation}

We organize the dissertation into three parts. 

Part I provides the necessary background in meaning representation, sentence embeddings, and neural model encoders. \refch{meaning} introduces meaning representations. \refch{training} reviews the architecture of standard encoders to compose words into sentence embeddings, the training objective, and evaluation methods.

Part II aims at improving the compositional properties of language models and their ability to generalize outside their training domain. We aim to integrate the recursive property of language within neural models and design architectures based on linguistic theory. \refch{structure-scale} proposes a self-supervised method that builds sentence embeddings from the combination of diverse explicit syntactic structures of a sentence. However, the tree-structured encoders require heavily structured data to compute the semantic representations. In \refch{tree}, we propose to overcome this limitation by proposing an architecture inducing trees from raw text and computing semantic representations along with the inferred structure. \refch{transformers} makes the parallel with transformers and sequential or tree-structured models. We interpret transformers as structured neural networks and layers as operations on fully connected graphs. We finally compare all models in \refch{arithmetics} by proposing an in-depth evaluation of their compositional properties.

Part III focuses on training and sharing models at scale. Indeed, the preparation of massive corpora, the training, and the use of large architectures are key for the performance of such models. \refch{1B} presents an attempt to train state-of-the-art sentence embedding models on a very large corpus. \refch{generative} proposes to train the first large generative pre-trained model in French.

\subsection{Publications}

This dissertation contains some contributions that we previously published and presented at conferences.

\begin{itemize}
    \item \refch{structure-scale} is an extended version of an article published in EACL Student Research Workshop 2021 \parencite{simoulin_2021a}. We  open-sourced the code developed for recursive models under a library called PyTree\sidenote{\url{https://github.com/AntoineSimoulin/pytree}}. The library was distinguished and listed among the winners of the PyTorch Hackathon 2021;
    \item \refch{tree} presents work currently under submission;
    \item \refch{transformers} is an extended version of an article published in ACL Research Student Workshop 2021 \parencite{simoulin_2021b};
    \item \refch{arithmetics} presents original unpublished work as well as work currently under submission;
    \item \refch{1B} relates the development of state-of-the-art sentence embedding models as part of the project \textit{Train the Best Sentence Embedding Model Ever with 1B Training; Pairs}. This project took place during the \textit{Community week using JAX/Flax for NLP \& CV} organized by Hugging Face. Our project was among the competition winners and received an honorable mention;
    \item \refch{generative} is an extended version of an article published in TALN 2021 \parencite{simoulin_2021c}.
\end{itemize}

%\footnote{\url{https://discuss.huggingface.co/t/open-to-the-community-community-week-using-jax-flax-for-nlp-cv/7104}}
%\footnote{\url{https://discuss.huggingface.co/t/train-the-best-sentence-embedding-model-ever-with-1b-training-pairs/7354}}

The code used for all experiments carried out for this dissertation, the pre-trained models, the evaluation and training datasets have been made publicly available as free softwares through the following repositories:

\begin{itemize}
    \item \url{https://github.com/AntoineSimoulin/pytree}
    \item \url{https://github.com/AntoineSimoulin/gpt-fr}
    \item \url{https://huggingface.co/datasets/asi/wikitext_fr}
    \item \url{https://huggingface.co/asi/gpt-fr-cased-base}
    \item \url{https://huggingface.co/asi/gpt-fr-cased-small}
\end{itemize}


% \pagelayout{wide} % No margins
% \addpart{Background}
% \pagelayout{margin} % Restore margins

% \setchapterstyle{kao}
\setchapterpreamble[u]{\margintoc}
\chapter{Defining sentence meaning}
\labch{meaning}

\section{The meaning of meaning}

\section{Formal semantic representations}

\section{Distributional semantic representations}

Distributional semantic, vector space
\setchapterstyle{kao}
\setchapterpreamble[u]{\margintoc}
\chapter{Embedding sentences}
\labch{training}

\cleanchapterquote{Give orange me give eat orange me eat orange give me eat orange give me you.}{Nim Chimpsky}{Male chimpanzee}

This chapter propose a literature review on today state-of-the art methods to train and evaluate sentence embedding models. We first expose traditional word embeddings methods (\refsec{survey:embeddings}). We then enumerate in \refsec{survey:encoding}, methods to compose word into sentence representation. We review the main training and evaluation setups in \refsec{training} and \refsec{evaluation}.

\section{Embedding words}
\labsec{survey:embeddings}

Embeddings are today the corner stone of every neural language model. In mathematics, an embedding is an injective and structure-preserving map f from one mathematical structure $X$ to another $Y$. The notion of “structure-preserving” depends on the nature of the latter structures. In natural language processing, we define words as a string (a sequence of characters) and the vocabulary as a finite set of distinct words. Embeddings e map the vocabulary $V$ to a vector space $E$ of dimension $h$. $e$ is an injective function an therefore, each word $w$ from the vocabulary is mapped to exactly one unique vector. Each vector from $E$ has a fixed length h and real values and are thus sometime called continuous vectors. 

Embeddings are convenient as we can exploit all the built-in properties from the representation space $E$. It thus provide all the mathematical tools to analyze words without relying on their surface form. It is straight-forward to define a notion of distance over the representation vector space to characterize its geometry. It is less obvious on the original vocabulary space. Embeddings methods usually rely on the distributional hypothesis: they characterize words given their distribution of co-occurrences in a given corpus. The core idea is that word with similar meaning tend to appear in similar context. As mentioned, embeddings preserve the structure from the original space. Therefore, embedding methods ensure that words with close distribution are mapped to close vectors, while words with distant distribution are mapped to distant vectors.

There exists multiple embedding frameworks. Since the 1990s, vector space models have become a popular tool in distributional semantic analysis, in particular with Latent Semantic Analysis (LSA) and Latent Dirichlet Allocation (LDA). \textcite{collobert_08} introduced a neural network architecture that formed the basis for many current methods utilizing pre-trained word embeddings. Their widespread application was enabled by \textsl{word2vec} \parencite{mikolov_13a, mikolov_13b} and GloVe \parencite{pennington_14}, efficient frameworks for the training of pre-trained embeddings. Word embeddings are characterized by their self-supervised supervision. They only need raw corpora of text to be trained. It is also possible to use embeddings layers that learn embeddings together with a given downstream task without prior training.

\section{Composing words into sentence embeddings}
\labsec{survey:encoding}

Many modern NLP systems use word embeddings as base features. Generalizing to embeddings for larger chunks of text, such as sentences, remains yet a question to be solved. Word embeddings operate on a finite vocabulary set, while we may build an infinite number of valid sentences. We can therefore not directly extend methods for embedding word to sentences. Sentence embedding methods rather aims at exploiting the compositionality principle: they compose word vector representations into sentence semantic representations.

Artificial neural networks consist of connected units called neurons. Neurons define a vector space transformation based on linear algebra operators and nonlinear activation functions. Neural networks typically contain a very large number of neurons, which may be arranged into layers. Neurons—and by extension layers—are inter-connected: they receive input from their inner connections and send their output to their outer connections. Each layer has its own inner structure and connection pattern. In this section, we present standard NLP architectures and define the notations we will reuse in all further chapters.

%TODO Il faut ajouter les références ! Vérifier les formules, ou sont les tanh ?

Structure of models, the role of structure. How do the requirement for syntax and word individual sense translate into neural models.

% The collection of layers and their connections forms a directed \textit{computational graph}.

\subsection{Bag-of-Words}
\labsec{architectures:bow}

The most straight-forward method to combine word vectors is the Bag-of-Words (BoW). We simply average all the vector from the sentences into one vector of the same size. This method does not account for the order of the word in the sentence, nor any kind of sentence structure. Yet, as analyzed in \textcite{arora_17}, this simple method may be a strong baseline for producing sentence embeddings.

\subsection{Recurrent neural networks}
\labsec{architectures:rnn}

Recurrent neural networks (RNN) \parencite{hochreiter_97, cho_14} take sequences $X = (x_1, x_2 \cdots x_T)$ as input. As illustrated in \reffig{rnn-cell-unfold}, they process the sequence iteratively, starting from the first element of the sequence, to the last. They consists in a RNN cell. For each element of the sequence $x_t$, the cell outputs an hidden state $h_t$, which depend from the current element of the sequence $x_t$ and from the previous element hidden state, $h_{t-1}$. The cell parameters are shared between each steps, and RNN can therefore process sequences of arbitrary length.

\begin{figure}[!ht]
	\includegraphics[width=9cm]{images/rnn_cell_unfold.png}
	\caption[RNN cell unfold]{We illustrate the recursive application of the RNN cell.}
	\labfig{rnn-cell-unfold}
\end{figure}

Basic recurrent neural networks suffer from practical limitations. In particular, gradient over-flow or underflow: when propagating the gradient error through the sequence, it tends to become very small or very large.  Gated mechanisms can mitigate this problem. These gates determine which information to retain for time steps.

% \begin{figure*}[!ht]
% 	\includegraphics[width=15cm]{images/gru_lstm_cell.png}
% 	\caption[GRU and LSTM cell]{GRU and LSTM cell gated connections.}
% 	\labfig{gru-lstm-cell}
% \end{figure*}

\paragraph{Gated recurrent units (GRU)} include a reset and update gate. Intuitively, the reset gate $r$ determines which information from previous step to reset (\refeq{gru-def-reset}). In \refeq{gru-def-last}, the update gate $z$, determines the amount of previous information that pass along the next step.

\begin{align}
r_t, z_t &=\sigma \Big( W^{(r, z)} x_t + U^{(r, z)} h_{t-1} + b^{(r, z)} \Big), \labeq{gru-def-first}\\
\tilde{h}_t &= \tanh(W^{(h)} x_t + U^{(h)} (r_t \odot h_{t-1}) +b^h) \labeq{gru-def-reset} \\
h_t &= (1-z_t) \odot \tilde{h}_t + z_t \odot h_{t-1} \labeq{gru-def-last}
\end{align}

\paragraph{Long short-term memory (LSTM)} integrate three gates. Besides the short memory vector $h$, it adds a long term memory vector $c$ that is passed along the steps. We detail the memory mechanism in \refeq{lstm-def-first} to \refeq{lstm-def-last}. Intuitively, the input gate $i$ determines what information to store in long term memory. The forget gate $f$ determines which information from the long term memory to forget. Finally, the output gate $o$ compute the new short term memory as a balance between the current input, the previous short term memory and the newly computed long term memory.

\begin{align}
i_t, o_t, u_t &=\sigma \Big( W^{(i, o, u)} x_t + U^{(i, o, u)} h_{t-1} + b^{(i, o, u)} \Big), \labeq{lstm-def-first}\\
f_{t} &= \sigma\left( W^{(f)} x_t + U^{(f)} h_{t-1} + b^{(f)} \right), \labeq{lstm-def-f}\\
c_t &= i_t \odot u_t + f_{t} \odot c_{t-1}, \\
h_t &= o_t \odot \tanh(c_t) \labeq{lstm-def-last}
\end{align}

\subsection{Tree-structure neural networks}
\labsec{architectures:tree}

Tree-structured neural networks generalize sequential networks to tree-structured topologies. They also consist in a cell that composes a state from an input vector $x_j$ and the hidden states of the input children, $h_k, \forall k \in C(j)$ with $C(j)$ the children of node $j$. As such, a sequential RNN is a special case of a Tree-RNN, where every node has exactly one child. We illustrate the composition process along an arbitrary tree structure in \reffig{tree-lstm}.

\begin{figure}[!ht]
	\includegraphics[width=7cm]{images/tree-lstm.png}
	\caption[Tree LSTM]{We illustrate the application of the Tree LSTM on an arbitrary branching tree. The figure takes inspiration from \url{https://arxiv.org/pdf/1503.00075.pdf}.}
	\labfig{tree-lstm}
\end{figure}

Intuitively, tree-structured network might be a better fit for language, which is supposed to follow a recursive structure. We focus on two specific frameworks describing language structure: dependency and constituency parsing. In constituent analysis, the syntactic structure of a sentence is represented as a nested multi-word constituents. The dependency tree represents the relationship between individual words. For constituents analysis, it is possible to binarize the tree, such that every node has exactly two children. It is also possible to differentiate the left and right children. Given these distinction, we define two tree-structured cell operations adapted for each of these frameworks.

\paragraph{Childsum Tree LSTM} \textcite{tai_15} compute sentence embeddings using a recursive node function derived from standard LSTM formulations but adapted for tree inputs. Each node is assigned an embedding given its dependent with a recursive function. The hidden state is computed as the sum of all children hidden states (\refeq{treelstm-def}). This model is adapted for dependency tree structured in which, words are connected through dependency edges. A word might have an arbitrary number of dependents.

\begin{align}
\tilde{h}_j &= \sum_{k \in C(j)} h_k, \labeq{treelstm-def} \\
i_j, o_j &=\sigma \Big( W^{(i, o)} x_j + U^{(i, o)} \tilde{h}_j + b^{(i, o)} \Big), \\
u_j &=\tanh \Big( W^{(u)} x_j + U^{(u)} \tilde{h}_j + b^{(u)} \Big), \\
f_{jk} &= \sigma\left( W^{(f)} x_j + U^{(f)} h_k + b^{(f)} \right), \labeq{treelstm-def-f}\\
c_j &= i_j \odot u_j + \sum_{k\in C(j)} f_{jk} \odot c_{k}, \\
h_j &= o_j \odot \tanh(c_j) \labeq{treelstm-def-last}
\end{align}

\paragraph{N-ary Tree LSTM} is also defined in \textcite{tai_15}. It is a tree structured-model designed for constituency parsed inputs, which describes the sentence as a nested multi-word structure. In this framework, words are grouped recursively in constituents. In the resulting tree, only leaf nodes correspond to words, while internal nodes encode recursively word sequences. It is possible to binarize the trees to ensure that every node has exactly two dependents. Again the representation is computed bottom-up and the embedding of the tree root node is used as sentence embedding. The equations make the distinction between right and left nodes.

\begin{align}
i_j, o_j &=\sigma \left( W^{(i, o)} x_j + \sum_{\ell=1}^N U^{(i, o)}_\ell h_{j\ell} + b^{(i, o)} \right), \labeq{nary-treelstm-def-first}\\
u_j &= \tanh\left( W^{(u)} x_j + \sum_{\ell=1}^N U^{(u)}_\ell h_{j\ell}  + b^{(u)} \right), \\
f_{jk} &= \sigma\left( W^{(f)} x_j + \sum_{\ell=1}^N U^{(f)}_{k\ell} h_{j\ell} + b^{(f)} \right), \labeq{nary-treelstm-def-f}\\
c_j &= i_j \odot u_j + \sum_{\ell=1}^N f_{j\ell} \odot c_{j\ell}, \\
h_j &= o_j \odot \tanh(c_j), \labeq{nary-treelstm-def-last}
\end{align}

\subsection{Transformer neural networks}
\labsec{architectures:transformers}

Introduced in \textcite{vaswani_17}, transformers originally consist in an encoder-decoder framework relying almost exclusively on attention and completely discarding any recurrent operation. By extension, the encoder or decoder taken separately may also be called transformers and we focus here on the encoder part.

\begin{figure}[!ht]
	\includegraphics[width=10cm]{images/transformer_layer_unfold.png}
	\caption[RNN cell unfold]{We illustrate the iterative application of transformer layers.}
	\labfig{transformer-layer-unfold}
\end{figure}

Transformers are composed of a series of layers. Each layer acts as a many-to-many encoder, mapping a set of vectors to a set of so-called contextualized vectors. Each layer is composed of a multi-head attention layer that map each input vector to a weighted average from the input set, followed by a feed-forward network (\refeq{transformers-layer}). As usual, the first layer (\refeq{transformers-emb}) is an encoding layer that map each word to a corresponding embedding. Additionally, the embedding layer encode each word position with dedicated positional embedding weights.

%TODO Attnetion manque un skip connection ici

\begin{align}
    h^0_t &= W_eu_t + W_p, \labeq{transformers-emb}\\
    h^{(k+1)}_u &= \text{FFN}^{(k)}\left(\text{MHA}^{(k)}\left( \{h_v^{(k)}), \forall v \in \mathcal{N}(u) \cup \{u\}\}\right) + h_v^{(k)}\right) \quad \forall n \in [1, L] \labeq{transformers-layer}
\end{align}

Transformer implementations may easily be parallelized since layers compose token contextualized representations simultaneously.

% \begin{figure}[!ht]
% 	\includegraphics[width=7cm]{images/transformer_layer.png}
% 	\caption[Transformer layer]{Transformer layer inner layers.}
% 	\labfig{transformer-layer}
% \end{figure}

\section{Training sentence embeddings}
\labsec{training}

% Dire aussi l'avantage des encoring de phrases versus juste les états cachés. Plus générique, peut êter utilisé sur plus de taches et de manière non supervisée : semantic similarity, clsutering, search engine ...

% Dire aussi que plusieurs contraintes : d'efficacité, de performances, de besoin en données labélisées.

All of the encoders discussed in \refsec{survey:encoding} produce sentence representations. By using them as input for a \textit{supervised task} and after backpropagating through the entire architecture, we can update the encoder’s composition weights (\reffig{training-sentence-embeddings}). On the one hand, we learn high-quality sentence representations, specific to each task. On the other hand, sentence embeddings intend to provide generic, general-purpose sentence representations. They should provide generic input features that can be applied to many different tasks, rather than being specific to a single task.

In this regard, we follow a two-step procedure, illustrated in \reffig{training-sentence-embeddings}. First, we train the encoder on a \textit{proxy task}. We only perform this step once. We then use the encoder's sentence representations as input for a \textit{downstream task}. We do not update the encoder weights for downstream tasks. As a result, any downstream task will use the same representation of a given sentence\sidenote{This procedure presents similarities with the \textit{pre-training}/\textit{fine-tuning} modus operandi. In contrast to fine-tuning, we do not update encoder weights when we train the full architecture on the downstream task.}.

Certainly, the nature of the proxy task is crucial to the procedure. Ideally, the task should produce sentence embeddings containing a variety of generic and rich features that can be used to resolve any downstream task. In this section, we make a literature review of the common task used as a proxy objective to train sentence embeddings.

\begin{figure*}[!ht]
	\includegraphics[width=15cm]{images/supervised-self-supervised.png}
    \caption{Training sentence embeddings in supervised and transfer learning setting. Expliquer pour la figure, ce qu'on entraine, ce qui set freeze. Pourquoi FFN ... C'est quoi L et L hat.} 
    \labfig{training-sentence-embeddings}
\end{figure*}

The majority of these proxy tasks involve predicting a relationship between two or more sentences. In theory, the model cannot predict the relationship without fully capturing the meaning of the considered sentences. All proxy objectives may have an impact on the model's capacity to capture aspects of meaning or the amount of data necessary to train a model. In \refsec{training:supervised}, we obtain the relation between the sentences by labelling data. It is also possible to use weaker signal in a self-supervised setting (\refsec{training:self-supervised}). Finally, it is possible to mix multiple training paradigms in a multi-task setup (\refsec{training:multi-task}).

% \begin{figure}[htb!]
%     \centering
%     \begin{subfigure}[b]{0.29\textwidth}
%         \centering
%         \includegraphics[width=\columnwidth]{images/supervised.png}
%         \caption[Training sentence embeddings with supervised setting]{Training sentence embeddings in a supervised setting.}
%     \end{subfigure}
%     \hfill
%     \begin{subfigure}[b]{0.69\textwidth}  
%         \centering 
%         \includegraphics[width=\columnwidth]{images/self-supervised.png}
%         \caption[Training sentence embeddings in transfer learning setting]{Training sentence embeddings in transfer learning setting}
%     \end{subfigure}
%     \caption{Training sentence embeddings in supervised and transfer learning setting} 
%     \labfig{training-sentence-embeddings}
% \end{figure}

% \begin{figure}[!ht]
% 	\includegraphics[width=9cm]{images/sentence-embeddings-mindmap.png}
% 	\caption[Training sentence embeddings]{Sentence embeddings training methods.}
% 	\labfig{sentence-embeddings-mindmap}
% \end{figure}

\subsection{Supervised learning}
\labsec{training:supervised}

\paragraph{Infersent} In keeping with the definition of meaning discussed in the last chapter, a model that captures the meaning of a sentence could infer the entailment relation between sentence pairs. Thus, training a model to predict the entailment relationship between two sentences seems reasonable to build efficient sentence embeddings. In this setup, the proxy task is therefore a natural language inference task (NLI). NLI consists in a supervised classification task. The model takes a input a sentence pair: a premise and an hypothesis. It should then predict whether the first entail, contradict or is neutral to the second. Large datasets exist for English like Stanford Natural Language Inference (SNLI)\sidenote{The dataset includes 570k pairs of sentences, distributed in a 550k/10k/10k train/dev/test split} \parencite{bowman_15} and MultiNLI\sidenote{The MultiNLI includes 433k sentence pairs. We refer to the concatenation of the SNLI and MultiNLI as AllNLI.} \parencite{williams_18b} or other languages including French with the XNLI corpus \parencite{conneau_18b}. We present some examples from the SNLI task in \reftab{snli-examples}.

\begin{table}[!htb]
\centering
\small
\begin{tabularx}{\textwidth}{@{}YYY@{} }
  \toprule
Premise & Hypothesis & label \\
\midrule
\midrule 
A man inspects the uniform of a figure in some East Asian country. & The man is sleeping & contradiction\\
\rule{0pt}{3ex}An older and younger man smiling. & Two men are smiling and laughing at the cats playing on the floor. & neutral\\
\rule{0pt}{3ex}A black race car starts up in front of a crowd of people. & A man is driving down a lonely road. & contradiction\\
\rule{0pt}{3ex}A soccer game with multiple males playing. & Some men are playing a sport. & entailment\\
\rule{0pt}{3ex}A smiling costumed woman is holding an umbrella. & A happy woman in a fairy costume holds an umbrella. & neutral\\
    \bottomrule
% From 1.0rc3
  \end{tabularx}
  \caption{\labtab{snli-examples}Examples presented in the original paper and extracted from the development section of the corpus.}
\end{table}


\textcite{conneau_17} propose a siamese framework to train model on NLI data. First a sentence encoder encode separately encodes the premise $h_L$ and the hypothesis $h_R$. The encoder weights are shared for the encoding of both parts but the two sentences are not encoded jointly (as it is the case when using cross-features or attention architectures). Then, a dedicated architecture is used to predict the similarity distribution from the pair of sentences. The similarity module takes as input a pair of sentence vectors $h_{L} $ and $h_{R}$ and computes their component\-wise product $h_{L} \odot h_{R}$ and their absolute difference $|h_{L} - h_{R}|$\sidenote{Their exists multiple variation of the similarity module which differs given the aggregation function of  $h_{L} $ and $h_{R}$, the number of fully-connected layers and their hidden dimensions \parencite{conneau_17, choi_18, reimers_19}.}. Given these features, we compute the probability distribution  $\hat{p}_{\theta}$ using a three layers perceptron network (MLP):

\begin{align}
\begin{split}
&h_{\times}=h_L\odot h_R, ~~~~~h_{+} = |h_L - h_R |, \\
&h_s = \textsc{Relu}(W^{(1)}[h_{\times}, h_{+}, h_L, h_R] + b^{(1)}), \\
&h_s = \textsc{Relu}(W^{(2)}h_s + b^{(2)}), \\
&\hat{p}_{\theta} = \text{softmax}(W^{(p)}h_s + b^{(p)}),\\
\end{split}
\end{align}

% We use the cross entropy loss between the prediction $\hat{p}_{\theta}$ and the ground truth $p$ as training objective:
% \begin{equation}
% J(\theta) = -\frac{1}{N}\sum_{k=1}^{N}p^{(k)} log \hat{p}_{\theta}^{(k)} + \lambda||\theta||_{2}^{2}
% \end{equation}

\textcite{conneau_17} proposes multiple sentence encoder to build $h_{L} $ and $h_{R}$, including LSTM and GRU, BiLSTM with mean/max pooling, Self-attentive network or Hierarchical ConvNet. SentenceBert later adapted the setup to use \textsc{Bert} as sentence encoder \parencite{reimers_19}. \textcite{reimers_19} uses the same supervised training method but with a pre-trained \textsc{Bert} as encoder.

% , we consider training the model on natural language inference task (NLI). This task consists in predicting the relation between two sentences. The possible relations are \textit{entailment}, \textit{contradiction} or \textit{neutral}.

% Supervised learning has been shown to yield general-purpose representations of meaning, training on semantic relation tasks like Stanford Natural Language Inference (SNLI) and MultiNLI


% Objective to learn general-purpose sentence embeddings.
% Proxy objective: classifying sentence pairs. Like the formal definition of meaning. Entailment, relatedness, sentence order, discourse relations, paraphrase identification.
% \paragraph{Task-supervised learning}

% Et de dissent

% Et paraphrase dans plusieurs langues: Multilingual Universal Sentence Encoder

\paragraph{DisSent} \textcite{nie_19} propose a weaker signal to train sentence embeddings: the discourse relations between sentences. The task is positioned as an intermediary between a fully supervised and self-supervised approach. Given two sentence embeddings, a classifier aims a identifying which discourse marker was used to link the sentences. As for infersent, the setup can accommodate any sentence encoder such can be used to train sequential LSTM or fine-tune larger pre-trained models such as \textsc{Bert}. We present some examples of the training data in \reftab{dissent-task_examples}.

\begin{table}[!htb]
\centering
\small
\begin{tabularx}{\textwidth}{@{}YYY@{}}
\toprule
S1  & S2  & marker\\
\midrule
\midrule 
Her eyes flew up to his face.
&Suddenly she realized why he looked so different.&
and\\
The concept is simple.
&The execution will be incredibly dangerous.&
but \\
You used to feel pride.
&You defended innocent people.&
because \\
Ill tell you about it.
&You give me your number.&
if \\
Belter was still hard at work.
&Drade and barney strolled in.&
when \\
We plugged bulky headsets into the dashboard.
&We could hear each other when we spoke into the microphones.&
so \\
It was mere minutes or hours.
&He finally fell into unconsciousness.&
before \\
And then the cloudy darkness lifted.
&The lifeboat did not slow down.&
though \\
\bottomrule
\end{tabularx}
\caption{Example pairs presented in the original paper.}
\labtab{dissent-task_examples}
\end{table}

The training dataset is build upon the BookCorpus dataset \parencite{zhu_15}\sidenote{This model requires a training corpus of contiguous text. The BookCorpus dataset is a collection of 11,038 free books written by yet unpublished authors. It contains books in 16 different genres. It contains 74,004,228 sentences and 984,846,357 words.}. The training pairs are collected using a semi-automated procedure. The authors used the Stanford CoreNLP dependency parser \parencite{schuster_16} to identify discourse markers between two sentences $S_1$ and $S_2$. They collected a curated a dataset of 4,706,292 pairs of sentences for 15 discourse markers. The training procedure is close from infersent. Given a sentence pair, a sentence encoder model produces sentences embeddings ($s_1$, $s_2$). A similarity modules then compute pair-wise vector operations and outputs probability distribution over discourse relations.

\begin{equation}
\begin{split}
    &s_{\text{avg}} = \frac{1}{2} (s_1 + s_2), ~~~s_{\text{sub}} = s_1 - s_2, ~~~s_{\text{mul}} = s_1 * s_2 \\
    &S = [s_1, s_2, s_{\text{avg}}, s_{\text{sub}}, s_{\text{mul}}]\\
    &h_s = \textsc{Relu}(W^{(2)}h_s + b^{(2)}), \\
    &\hat{p}_{\theta} = \text{softmax}(W^{(p)}h_s + b^{(p)}),\\
\end{split}
\label{eq:vec_op}
\end{equation}

\paragraph{Mining sentence pairs} 

It is also possible to use other sentence pairs as signal to train sentence embedding models. To only cite a few, \textcite{gimpel_18} produce the PARANMT-50M, a dataset of more than 50 million English-English sentential paraphrase pairs. The dataset was generated automatically by using neural machine translation on a parallel corpus. \textcite{yang_20} train a multilingual sentence embedding model by using training QA pairs mined from online forums and QA websites, including Reddit, StackOverflow, and YahooAnswers.

% \paragraph{Transfer learning}

\subsection{Self-supervised learning}
\labsec{training:self-supervised}

Previous methods rely on annotated data or semi-automatically constructed corpora. However, such resources may be hard to find in other languages than English or in specific domains. In this section, we review methods that rely only on the structure of raw text and that may be trained in a self-supervised manner. 
%TODO Il faut aussi homogénéiser les notations

\paragraph{ParagraphVector (\textsl{doc2vec})} \textcite{le_14} proposed two log-linear models of sentence representation. The DBOW model is trained to predict the words $w$ from a given sentence $s$ by taking as input the sentence embedding $h$. Each sentence (or document) is assigned a unique embedding vector while the word embeddings $v_w$ are shared across the corpus. The DM model, concatenate the sentence embedding $h$ with the embedding of $k$ consecutive words $w_i \cdots w_{i+k}$ in $s$ to predict the next word $w_{i+k+1}$.

\paragraph{Skip-thought (ST)} \textcite{kiros_15} aims at translating the skip-gram function to the sentence level. Instead of predicting a word's context, it predicts whole sentences. Skip-thought works as a sequence-to-sequence framework. Given a tuple of consecutive sequences $(s_{i-1}, s_i, s_{i+1})$ as input, it encodes the context sentence using a sentence encoder $SE$ in a fixed length vector $h_i = SE(s_i)$. Given the sentence vector, a sentence decoder $DE_p$ aims to generate the previous sentence $DE_p(h_i) = s_{i-1}$ and the next sentence $DE_n(h_i) = s_{i+1}$.

Both the encoder and decoder are trained to minimize the sum of the log-probabilities for the forward and backward sentences conditioned on the encoder representation:

\begin{equation*}
    \mathcal{L} = \sum_t \text{log} P(w_{i+1}^t | w_{i+1}^{<t}, \textbf{h}_i) + \sum_t \text{log} P(w_{i-1}^t | w_{i-1}^{<t}, \textbf{h}_i)
\end{equation*}

The model is trained on the BookCorpus dataset. The original implementation uses Recurrent Neural Network, with Gated Recurrent Units \parencite{cho_14} for the encoder and decoder. Skip-thought method have become popular as the method is fully self-supervised and do not require any labelled data. Moreover, the original paper trained models at scaled and released them in open-source\sidenote{\url{https://github.com/ryankiros/skip-thoughts}. \textcite{ba_16} also proposed an upgrade of the model by adding layer normalization.}. 

The methods yet suffer from practical limitations. First, the decoding part is trained to reconstruct the surface form of the sentence while we are only interested in the semantic aspects. Second, the decoding part—although the decoders are not used during inference—is computationally costly. The approach must sequentially decode the words of target sentences. Moreover, predicting each output word prediction on a heavy softmax operation over the entire vocabulary. Overall, it takes two weeks to train the original model. 

\paragraph{Sequential Denoising Autoencoder (SDAE)} \textcite{hill_16} propose a model based on denoising autoencoders (DAEs) for text. The model uses an encoder-decoder framework to reconstruct a corrupted version of the current sentence. As with Skip-thought, the model has an unsupervised objective, but does not require that the training corpus maintains the narrative order of the sequences. The input sentence $s$ is corrupted using a noise function $N(s|p_o, p_x)$ which acts as follows: for each word $w \in s$, $N$ deletes $w$ with (independent) probability $p_o$. Then, for each non-overlapping bigram $w_iw_{i+1} \in s$ , $N$ swaps $w_i$ and $w_{i+1}$ with probability $p_x$. The encoder-decoder architecture is based on LSTMs and is also trained on the BookCorpus dataset to optimize the following loss function:

\begin{equation*}
    \mathcal{L} = \sum_t \text{log} P(w_{i+1}^t | w_{i+1}^{<t}, \textbf{h}_i)
\end{equation*}

\paragraph{FastSent} The method, also introduced in \textcite{hill_16}, is a additive (log-linear) version of Skip-thought, which aims to lower its computational expense. The model is trained to predict the words appearing in context (and optionally, the anchor) sentences given a BoW representation of the anchor.

FastSent learns a source $u_w$ and target $v_w$ embedding for each word in the model vocabulary. Given a tuple of consecutive sequences $(s_{i-1}, s_i, s_{i+1})$ as input, it encodes the anchor sentence as the sum of its word embeddings $h_i = \sum_{w \in s_i}u_w$. Given the representation of the anchor sentence, it aims at predicting the word of the context sentences. 

\begin{equation*}
    \mathcal{L} = \sum_{w \in s_{i-1} \cup s_{i+1}} \text{log} P(w | h_i)
\end{equation*}

With $P(w | h_i) = \frac{e^{h_i u_w}}{\sum_{v \in V}e^{h_i u_v}}$. A variant include the prediction of the words from the anchor sentence in addition to those of adjacent sentences. The objective function thus becomes:

\begin{equation*}
    \mathcal{L} = \sum_{w \in s_{i-1} \cup s_{i} \cup s_{i+1}} \text{log} P(w | h_i)
\end{equation*}

\paragraph{Quick-thoughts (QT)} \textcite{logeswaran_18} circumvents some practical limits of Skip-thoughts by directly operating in the space of sentence embeddings. It uses a discriminative rather than a generative objective\sidenote{More broadly, the approach relates to contrastive learning, which is successfully applied in a variety of domains including audio \parencite{oord_18}, image \textcite{wu_18, tian_19}, video or word with the negative sampling methods from \textsl{word2vec} \parencite{mikolov_13a, mikolov_13b}. Some mathematical foundations are detailed in \textcite{saunshi_19}}. A classifier aims at distinguishing the correct embedding of a target sentence given a set of candidate sentences. The method thus avoids to reconstruct the surface form of the input sentence or its neighbors.

The method takes inspiration from the distributional hypothesis successfully applied for word, but this time, to identify context sentences. Given a sentence $s$, a corresponding context sentence $s^+$ and a set of $K$ negative samples $s^-_1 \cdots s^-_K$, the training objective is to maximize the probability of discriminate the correct sentence among negative samples: $p(s^+ | s, s^-_1 \cdots s^-_K)$. The algorithm architecture used to estimate $p$ is close to \textsl{word2vec}. Two sentences encoders $f$ and $g$ are defined and the conditional probability is estimated as follow:

Peut être remplacer les + et - par des i et i+ 1...

\begin{equation*}
    p(s^+ | s, s^-_1 \cdots s^-_K) = \frac{e^{f(s)^Tg(s^+)}}{e^{f(s)^Tg(s^+)}+\sum_{i=1}^Ne^{f(s)^Tg(s^-_i)}}    
\end{equation*}

The parameters from $f$ anf $g$ are trained to maximize the probability of identifying the correct context sentences for
each sentence in the training data D:

\begin{equation*}
    \mathcal{L} = \sum_{s \in D} \text{log} P(s^+ | s, s^-_1 \cdots s^-_K)
\end{equation*}

% \marginpar{It is not entirely clear from text and from picture if each sentence is encoded by each view or if a sentence is encoded randomly by one view or another. I guess the figure is misleading.}

The model is also trained on the BookCorpus dataset. Each batch is composed of contiguous sentences from the corpus. For each sentence, all the sentences in the batch constitute the candidate for classification. The pre-train model is also available in open-source\sidenote{\url{https://github.com/lajanugen/S2V}}. At inference time, the sentence representation is obtained as the concatenation of the two encoders $f$ and $g$ such as $s \rightarrow [f(s);g(s)]$. $f$ and $g$ are chosen identical and consist in two LSTM. 

%TODO Mettre des images pour chaque méthode.

\subsection{Multi-task learning}
\labsec{training:multi-task}

Some frameworks propose to combines the training objective mentioned above in a multi-task setup. We expect the model to encode complementary properties and inductive biases required for each sub-task. Thus, training on many weakly related tasks is expected to improve generalization to novel ones.

The universal sentence encoder (USE) \parencite{cer_18} train a transformer and Deep Averaging Network (DAN) on a multi-task setup: a skip-thought objective \parencite{kiros_15}, a conversational response prediction and a supervised natural language inference classification task on the SNLI dataset \parencite{bowman_15, conneau_17}.

\textcite{subramanian_18} also propose a multitask learning framework that trains a single model with six distinct objectives: context sentences generation (\refsec{training:self-supervised}), neural machine translation, constituency parsing and natural language inference (\refsec{training:supervised}).

\section{Evaluating sentence embeddings}
\labsec{evaluation}

Evaluation of sentence embedding is not a straightforward process. As for the training step, we do not have access to \textit{gold} labels to evaluate our embeddings. We must therefore rely on indirect evaluation methods. The first set of methods in \refsec{survey:downstream} characterizes the quality of the sentence representations given the performances they allow on a task of interest. The second set of methods probe for controlled and targeted linguistic characteristics by the mean or indirect classification task on dedicated artificial datasets (\refsec{survey:probing}). Finally, we enumerate in \refsec{survey:analysis} methods to directly analyze the underlying dynamics and mechanisms within the connections of model layers.

\subsection{Downstream tasks}
\labsec{survey:downstream}
% Ok, ça marche bien
% Mais dire que Bert résout pas tous les problèmes : mauvaise représentation des phrases

The SentEval benchmark \parencite{conneau_18}\sidenote{Senteval is posterior to most of the references. However, these studies do evaluate on tasks later included in the benchmark.} is specifically designed to assess the quality of the embeddings. Each task is formatted as a classification task that take a sentence embedding as input features. The downstream model usually consists in a simple multi layers perceptron or logistic regression. It is kept as minimal as possible to avoid the case where uninformative embeddings are compensated by an excellent classifier\sidenote{It is important to make the distinction between the training of the sentence embedding method (detailed in \refsec{training:self-supervised} and the training of the downstream classifier which use the sentence embedding as input but doesn't not further trained them.}. An other important reason for using simple downstream classifiers is to assess for straightforward extractability of information from embeddings. Our goal is to identify whether the information is captured in the embedding vectors rather than assessing whether we can reconstruct the information from the embeddings. The downstream evaluation methods is completely agnostic with respect to the sentence embedding method\sidenote{Contrary to GLUE and SuperGLUE benchmarks \parencite{wang_19_glue, wang_19_superglue}, the sentence embedding model is not fine-tuned during the evaluation. We specifically evaluate the information whithin sentence embeddings and not the model used to produce them.}. For each task, the development set is used for choosing the regularization parameter and results are reported on the test set. The tasks include sentiment and subjectivity analysis (\textbf{MR, CR, SUBJ, MPQA}), question type classification (\textbf{TREC}), paraphrase identification (\textbf{MRPC}) and semantic relatedness (\textbf{SICK-R}). We report in \reftab{senteval} the downstream results using the training methods enumerated in \refsec{training}.

\begin{table*}[!htb]
\footnotesize
% \begin{minipage}{\textwidth}
\centering
\begin{tabularx}{16cm}{@{}c c Y | Y Y Y Y Y Y Y Y Y Y @{}}
\toprule
\multirow{2}{*}{\textbf{Model}} & \multirow{2}{*}{\textbf{Dim}} & \multirow{2}{*}{\textbf{Hrs}} & \multirow{2}{*}{\textbf{MR}} & \multirow{2}{*}{\textbf{CR}} & \multirow{2}{*}{\textbf{SUBJ}} & \multirow{2}{*}{\textbf{MPQA}} & \multirow{2}{*}{\textbf{TREC}} &  \multicolumn{2}{c}{\textbf{MRPC}} &  \multicolumn{3}{c}{\textbf{SICK-R}}\\%\cmidrule(r){9-10} \cmidrule(r){11-13}
 &  &  &  &  &  &  &  & \textbf{Acc} & \textbf{F1} & \textbf{$r$} & \textbf{$\rho$} & \textbf{MSE}\\\midrule
\multicolumn{13}{c}\textit{Context sentences prediction} \\\midrule
FastSent & $\leq500$ & 2 & 70.8 & 78.4 & 88.7 & 80.6 & 76.8 & 72.2 & 80.3 & --- & --- & ---\\
FastSent + AE & $\leq500$ & 2 & 71.8 & 76.7 & 88.8 & 81.5 & 80.4 & 71.2 & 79.1 & --- & --- & ---\\
Skipthought & $4800$ & 336 & 76.5 & 80.1 & 93.6 & 87.1 & 92.2 & 73.0 & 82.0 & 85.8 & 79.2 & 26.9\\
Skipthought + LN & $4800$ & 672 & 79.4 & 83.1 & 93.7 & 89.3 & --- & --- & --- & 85.8 & 78.8 & 27.0\\
Quickthoughts & $4800$ & 11 & \textbf{80.4} & \textbf{85.2} & \textbf{93.9} & \textbf{89.4} & \textbf{92.8} & \textbf{76.9} & \textbf{84.0} & \textbf{86.8} & \textbf{80.1} & \textbf{25.6}\\
\midrule\multicolumn{13}{c}\textit{Sentence relations prediction} \\\midrule
InferSent & $4096$ & --- & \textbf{81.1} & \textbf{86.3} & 92.4 & 90.2 & 88.2 & \textbf{76.2} & \textbf{83.1} & \textbf{\underline{88.4}} & --- & ---\\
DisSent Books 5 & $4096$ & --- & 80.2 & 85.4 & 93.2 & 90.2 & 91.2 & 76.1 & --- & 84.5 & --- & ---\\
DisSent Books 8 & $4096$ & --- & 79.8 & 85.0 & \textbf{93.4} & \textbf{\underline{90.5}} & \textbf{\underline{93.0}} & 76.1 & --- & 85.4 & --- & ---\\
\midrule\multicolumn{13}{c}\textit{Pre-trained transformers} \\\midrule
\textsc{Bert}-base [CLS] & $768$ & 96 & 78.7 & 84.9 & 94.2 & 88.2 & \textbf{91.4} & 71.1 & --- & 75.7$^\dagger$ & --- & ---\\
\textscsc{Bert}-base [NLI] & $768$ & 96 & \textbf{\underline{83.6}} & \textbf{\underline{89.4}} & \textbf{94.4} & \textbf{89.9} & 89.6 & \textbf{76.0} & --- & \textbf{84.4}$^\dagger$ & --- & ---\\
\bottomrule
\end{tabularx}
\caption{entEval Task Results Using Fixed Sentence Encoder.
We divided the table into sections. The first range of models uses self-supervised training objective. The second section present models trained on labelled or semi-automatically labeled data. The third section reports pre-trained transformers based-models. FastSent is reported from \textcite{hill_16}. Skipthoughts results from \parencite{kiros_15} Skipthoughts + LN which includes layer normalization method from \textcite{ba_16}. We considered the Quickthoughts results \parencite{logeswaran_18}. DisSent and Infersent are reported from \textcite{nie_19} and \textcite{conneau_17} respectively. Pre-trained transformers results are reported from \textcite{reimers_19}. The \textbf{Hrs} column indicates indicative training time, the \textbf{Dim} column corresponds to the sentence embedding dimension. $^\dagger$\, indicates models that we had to re-train. Best results in each section are shown in \textbf{bold}, best results overall are \underline{underlined}. Performance for \textbf{SICK-R} results are reported by convention as $\rho \text{ and } r \times 100$.}
\labtab{senteval}
\end{table*}

Yet, downstream task may suffer from empirical limitations. The downstream performances may not necessarily reflect the quality of the representations. First, the complexity of the tasks makes it difficult to determine what information is captured in the representations. Then, Uncontrolled effects can inflate the perception of success on downstream tasks: certain hyper-parameters such as the embedding dimensions may have an impact on downstream performance ; models may also exploit superficial cues or structural biases from the evaluation datasets.

In that regard, \textcite{wieting_19} propose a rather disturbing study in which they test randomly initialized encoders on downstream task and still obtain competitive results. They show that above and beyond the encoder structure, many parameters impacts downstream performances. In particular, the quality of the word embeddings being composed by the encoder, or the dimension of the word and sentence embeddings. Similarly, \textcite{adi_17} demonstrate that a BoW composition model is 70\% accurate on a binary word orders prediction task. Since BoW model do not preserve word order information, \textcite{ettinger_18} interpret that the above-chance performance appears to rely on statistical regularities of word ordering in the train and test sets.

\subsection{Probing tasks}
\labsec{survey:probing}
% Mais dans quelles limites, sur quel périmètre, et comment

%reformuler phrase de ettinger
% citer random sent et Annotation artifacts in natural language inference data.

% Probing tasks evaluate representation on a per-phenomenon basis, by discriminating representations containing or not a particular semantic, syntactic, lexical or surface information. They typically consist of simple classification tasks that require access to the relevant information to achieve high accuracy. 

% Probing tasks evaluate representation on a per-phenomenon basis, to discriminate representations containing or not a particular semantic, syntactic, lexical or surface information. They typically consist of simple classification contingent on a precise linguistic property.	

% They typically consist of simple classification tasks that require access to the relevant information to achieve high accuracy. 

% that target for
Probing tasks evaluate representations on a per-phenomenon basis. They aim at determining  which precise semantic, syntactic, lexical or surface information of the input sentence is captured in its embeddings. They typically consist of simple classification contingent on a precise linguistic property. This targeted approach simplifies interpretations. Probing and downstream evaluation follow the same protocole: the probing classifier takes as input feature the sentence embeddings produced by a given encoder (as for the downstream evaluation, the embeddings are not further tuned in that phase). A high accuracy on the task should therefore indicate that the information is encoded in the input embeddings.

Probing tasks require to maintain access to the detailed labeling of the linguistic phenomenon of interest. Given this information, it is possible to partition the dataset and decompose the model performances given this specific phenomenon. This partition should also maintain the dataset distribution unchanged regarding any other linguistic phenomena and remove any uncontrolled bias toward this specific aspect\sidenote{For example, \textcite{lai_14, bentivogli_16} observed structural biases in the SICK \parencite{marelli_14} dataset distribution. As a consequence, a simple heuristic detecting negation is sufficient to achieve high accuracy for the textual entailment task.}. Last but not least, it should retain a variety of sentences that will be encountered in natural-occurring text.

Probing tasks must therefore be constructed in a rigorous and controlled manner. The dataset is usually created either by labelling natural occurring sentences or by semi-automatically generating sentences that follow specific properties. The first method facilitates the access to wide variety of syntactic structures and configurations. On the other hand, semi-automatically sentence generation allow for the precise controlled of their targeted characteristics.

%TODO Vérifier que baroni inclue bien adi

Probing task is an active subject of research and have been adapted for many linguistic properties.\textcite{baroni_18} aggregate 10 tasks—including those introduced in \textcite{adi_17}—in a benchmark. The tasks tests for surface, semantic and syntactic information. The sentence length (\textbf{SentLen}) task aims at predicting for the length of sentences in terms of number of words. The word content (\textbf{WC}) task determines the ability to recover the original words in a sentence from the embedding. The bigram shift (\textbf{BShift}) tests the sensitivity to original word orders. The tree depth (\textbf{TreeDepth}) aims at predicting the depth from a sentence hierarchical structure. The top constituent task (\textbf{TopConst}), aims at predicting the top constituent immediately below the sentence root node. the \textbf{Tense} task aims at predicting for the tense of the main-clause verb. The subject and object number (respectively \textbf{SubjNum} and \textbf{ObjNum}) tasks focuses on the number of respectively the subject and object of the main clause. The semantic odd man out (\textbf{SOMO}) task aims at identifying sentences for which a random nouns or verbs was replaced. Finally, the coordination inversion (\textbf{CoordInv}) aims at identifying sentences for which the order of the clauses was or not modified.

% On peut citer \parencite{baroni_18}, \parencite{ettinger_18}, \parencite{adi_17} avec des exemples de probing tasks.


% Analysis with small synthetic datasets
% Semantic textual similarity

% Compositional properties
% parler de adi et al
% et ettinger

% Comparison with formal and distributional semantic representations. Very efficient. Yet not like humans.

\subsection{Analysis of the internal dynamics underlying NLP models}
\labsec{survey:analysis}
% Que ce passe t-il dans le réseau en pratique ?

Finally, some alternative approaches propose intuitive visualization techniques that allow to interpret neural network mechanisms when processing specific examples. \textcite{li_16} propose to analyze compositional model properties with some specific plots. Using dimensionality reduction methods, they project words or phrases before and after modifying, negating, or composing clauses. Additionally, they display the saliency of individual tokens with respect to their prediction. Other methods propose visualization of neural model hidden states. \textcite{strobelt_18} represent recurrent models hidden states. While \textcite{hoover_19} propose a similar tool for the analysis of transformers.

% \setchapterstyle{kao}
\setchapterpreamble[u]{\margintoc}
\chapter{Evaluating sentence embeddings}
\labch{evaluating}

\section{Downstream tasks}

Paraphrase, entailment, sentiment analysis

\section{Representation space topology}

Semantic textual similarity

\section{Compositional properties}

Comparison with formal and distributional semantic representations. Very efficient. Yet not like humans.


% \pagelayout{wide} % No margins
% \addpart{Toward integrating linguistic biases into neural networks}
% \pagelayout{margin} % Restore margins

% \setchapterstyle{kao}
\setchapterpreamble[u]{\margintoc}
\chapter{Structured sentence encoders at scale}
% using a generative objective
\labch{structure-scale}

% \section{Training efficient sentence embedding models using structured encoders}
% \labsec{structure-scale}

\cleanchapterquote{When people say AI has “learned x” what they usually mean is that a deep learning model has learned a dataset well enough to find the pattern you asked for. It has no symbolic or logical abstraction. It is kahnemann system 1. It looks smart. It isn’t.}{Mark Madsen}{Twitter post, 2019}

The previous sections proposed to train sentence encoders at scale, In \refch{1B} by augmenting the pre-training corpus size and in \refch{generative} by proposing alternative pre-training objectives. In this section, we focus on the encoder architecture. We train various encoders and combine them in an original multi-view setup.

The section is derived from \textcite{simoulin_2021a}, in which we propose a self-supervised embeddings method that combines diverse explicit syntactic structures of a sentence. As detailed in \refsec{structure:motivation}, the novelty consists in jointly learning structured models in a contrastive multi-view framework that induces an explicit interaction between models during the training phase. We pre-trained various models using a contrastive objective with a 40 million sentences corpus (\refsec{structure:method}). We then evaluate the models on sentence embedding benchmarks and obtain state-of-the-art results (\refsec{structure:experiments}). In particular on tasks that are expected, by hypothesis, to be more sensitive to sentence structure.

\section{Motivation}
\labsec{structure:motivation}

% Inspired from linguistic insights, I assume structure is crucial to building consistent representations. I indeed expect sentence meaning to be a function of both syntax and semantic aspects. 
We hypothesize structure is a crucial element to perform compositional knowledge. In particular, the heterogeneity of performances across models and tasks makes us assume that some structures may be better adapted for a given example or task. Therefore, combining diverse structures should be more robust for tasks requiring complex word composition to derive their meaning. Hence, we aim here to evaluate the potential benefit from interactions between pairs of encoders. In particular, we propose a training method for which distinct encoders are learned jointly. We conjecture this association might improve our embeddings' power of generalization and propose an experimental setup to corroborate our hypothesis.

We take inspiration from multi-view learning, which is successfully applied in a variety of domains. In such a framework, the model learns representations by aligning separate observations of the same object. Such observations are referred to as \textit{views}. In our case, we consider a view for a given sentence as the association of the plain sentence with a syntactic structure. 

Combining different structural views has already been proven to be successful in many NLP applications. \textcite{kong_11} provide a heuristic to combine dependency and constituency analysis for coreference resolution. \textcite{zhou_16, ahmed_19_2} combine Tree LSTM and standard sequential LSTM with a cross-attention method and observe improvements on a semantic textual similarity task. \textcite{chen_liu_17} combine CNN and Tree LSTM using attention methods and outperform both models taken separately on a sentiment classification task. Finally, \textcite{chen_17} combine sequential LSTM and Tree LSTM for natural language inference tasks. 

The novelty here is to combine distinct structured models to build standalone sentence embeddings, which has not yet been explored. This paradigm benefits from several structural advantages. It pairs nicely with contrastive learning, as already mentioned. It might thus be trained in a self-supervised manner that does not require data annotation. Moreover, contrary to models presented in Section \ref{sec:multi-views}, our method is not specific to a certain kind of encoder architecture. It does not require, for example, the use of attention layers or tree-structured models.

Our setup could therefore be extended with any encoding function. Finally, our training method induces an interaction between models during inference and, paramountly, during the training phase.

\section{Method}
\labsec{structure:method}

\subsection{Contrastive learning}

\begin{figure*}[!htb]
\begin{center}
\includegraphics[width=15cm]{images/contrastive-4.png}
\end{center}
\caption{\textbf{Contrastive training method}. The objective is to reconstruct the storyline. Sentences are presented in their original order. Given an anchor sentence $x$, the model should identify the context sentence $x^+$ out of negative samples $x_1^-, x_2^-$. Sentences are encoded using separate views, which are composed within a pairwise distance matrix. }
\labfig{contrastive}
\end{figure*}

We train our model using the contrastive objective from \textcite{logeswaran_18}, detailed in \refsec{training}. The method takes inspiration from the distributional hypothesis successfully applied for word, but this time, to identify context sentences. Given a sentence $s$, a corresponding context sentence $s^+$ and a set of $K$ negative samples $s^-_1 \cdots s^-_K$, the training objective is to maximize the probability of discriminate the correct sentence among negative samples: $p(s^+ | S)$ with $S = \{s, s^+, s^-_1 \cdots s^-_K\}$. As illustrated in \reffig{contrastive}, two sentences encoders $f$ and $g$ are defined and the conditional probability is estimated as follow\footnote{\citet{logeswaran_18} simply use an inner product for $c$ such as $c\left(x, y\right) = x^Ty$. In our case, as the encoders $f$ and $g$ are distincts, we choose a bilinear function defined as $c\left(x, y\right) = x^TWy$ \cite{tschannen_20}.}:

\[
p(s^+ | S) = \frac{e^{c\left(f(s), g(s^+)\right)}}{e^{c\left(f(s), g(s^+)\right)}+\sum_{i=1}^Ne^{c\left(f(s), g(s^-_i)\right)}}
\]

At inference time, the sentence representation is obtained as the concatenation of the two encoders $f$ and $g$ such as $s \rightarrow [f(s);g(s)]$, as illustrated in \reffig{inference}. In \textcite{logeswaran_18}, $f$ and $g$ use the same RNN encoder. However, the authors observe that the encoders might learn redundant features. To limit this effect, they propose to use a distinct set of embeddings for each encoder. 

We propose addressing this aspect by enhancing the method with a multi-view framework and using a distinct structured model for the encoders $f$ and $g$. We hypothesize that some structures may be better adapted for a given example or task. 
For example, dependency parsing usually sets the verb as the root node. Whereas in constituency parsing, subject and verb are often the right and left child from the root node. Therefore, the combination of different structures should be more robust for tasks requiring complex word composition and be less sensitive to lexical variations. Consequently, we propose a training procedure that allows the model to benefit from the interaction of various syntactic structures.

\subsection{Language views}

Multi-view aims as learning representations from data represented by multiple independent sets of features. We generalize the notion of view for a sentence as the application of a specific syntactic framework. For each view, we use an \textit{ad-hoc} algorithm that maps the structured sentence into an embedding space.

We consider structures exposed in \refsec{survey:encoding}: Vanilla GRU (\textsc{Seq}), dependency tree combined with the Child-Sum Tree LSTM (\textsc{Dep}), Constituency tree combined with N-Ary Tree LSTM (\textsc{Const}). Although equivalences might be derived between the two representations schemes, we hypothesize that, in our context, the corresponding sequence of operations might allow capturing rather distinct linguistic properties. The various models may, therefore, be complementary and their combination allows for more fine-grained analysis. 

For the \textsc{Dep} view, the dependency tree is obtained using the deep biaffine parser from \textcite{dozat_17}. We use an open-source implementation\sidenote{\url{https://github.com/yzhangcs/biaffine-parser}} of the dependency parser \parencite{dozat_17} and replace the pos-tags features with features obtained with \textsc{bert}. Therefore we do not need pos-tags annotations to parse our corpus. Regarding the inference speed, The constituency parser is the bottleneck in this case and parse around 500 sentences/second. In our case, the parsing of the entire corpus (40M sentences) take about a day to complete. 

For the \textsc{Const} view, the structure is obtained using the constituency neural parser from \textcite{klein_18}. We binarize the trees to ensure that every node has exactly two dependents. The binarization is performed using a left markovization and unary productions are collapsed in a single node. 

\begin{figure}[!htb]
\begin{center}
\includegraphics[width=10cm]{images/contrastive-inf.png}
\end{center}
\caption{\textbf{Multi-view sentence embedding}. At inference, embeddings are the concatenation from both views.}
\labfig{inference}
\end{figure}

\section{Experiments}
\labsec{structure:experiments}

We train our models on the UMBC dataset \footnote{\url{https://ebiquity.umbc.edu/blogger/2013/05/01/umbc-webbase-corpus-of-3b-english-words/}}\sidenote{The bookcorpus introduced in \textcite{zhu_15} and traditionally used for sentence embedding is no longer distributed for copyright reasons. Therefore, we prefer a corpus freely available. The impact of the training dataset choice is analyzed in Appendix \ref{appendix:training-dataset}.} \parencite{han_13}. We limited our corpus to the first 40M sentences from the tokenized corpus. Indeed, \textcite{logeswaran_18} already analyze the effect of the corpus size, and we focus here on the impact of our multi-view setting. We build batches from successive sentences. Given a sentence in a batch, other sentences not in the context are considered as negatives samples. 

Model hyper parameters are fixed given literature on comparable work \parencite{tai_15, logeswaran_18}. All models are trained using a batch size of 400 and the Adam optimizer with a $5e^{-4}$ learning rate. Regarding the infrastructure, we use a Nvidia GTX 1080 Ti GPU. All model weights are initialized with a Xavier distribution and biases set to 0. We do not apply any dropout. 

For the vocabulary, we follow the setup proposed in \textcite{kiros_15, logeswaran_18} and we train two models in each configuration. One initialized with pre-trained embedding vectors. The vectors are not updated during training and the vocabulary includes the top 2M cased words from the 300-dimensional GloVe vectors\sidenote{\url{https://nlp.stanford.edu/projects/glove/}} \parencite{pennington_14}. The other is limited to 50K words initialized with a Xavier distribution and updated during training. For inference, the vocabulary is expanded to 2M words using a linear projection.

\subsection{Evaluation on SentEval}

\begin{table*}[!htb]
\footnotesize
% \begin{minipage}{\textwidth}
\centering {
\begin{tabularx}{16cm}{@{}c c Y | Y Y Y Y Y Y Y Y Y Y @{}}
\toprule
\multirow{2}{*}{\textbf{Model}} & \multirow{2}{*}{\textbf{Dim}} & \multirow{2}{*}{\textbf{Hrs}} & \multirow{2}{*}{\textbf{MR}} & \multirow{2}{*}{\textbf{CR}} & \multirow{2}{*}{\textbf{SUBJ}} & \multirow{2}{*}{\textbf{MPQA}} & \multirow{2}{*}{\textbf{TREC}} &  \multicolumn{2}{c}{\textbf{MRPC}} &  \multicolumn{3}{c}{\textbf{SICK-R}}\\%\cmidrule(r){9-10} \cmidrule(r){11-13}
 &  &  &  &  &  &  &  & \textbf{Acc} & \textbf{F1} & \textbf{$r$} & \textbf{$\rho$} & \textbf{MSE}\\\midrule
\multicolumn{13}{c}{\textit{Context sentences prediction}} \\\midrule
FastSent & $\leq500$ & 2 & 70.8 & 78.4 & 88.7 & 80.6 & 76.8 & 72.2 & 80.3 & --- & --- & ---\\
FastSent + AE & $\leq500$ & 2 & 71.8 & 76.7 & 88.8 & 81.5 & 80.4 & 71.2 & 79.1 & --- & --- & ---\\
Skipthought & $4800$ & 336 & 76.5 & 80.1 & 93.6 & 87.1 & 92.2 & 73.0 & 82.0 & 85.8 & 79.2 & 26.9\\
Skipthought + LN & $4800$ & 672 & 79.4 & 83.1 & 93.7 & 89.3 & --- & --- & --- & 85.8 & 78.8 & 27.0\\
Quickthoughts & $4800$ & 11 & \textbf{80.4} & \textbf{85.2} & \textbf{93.9} & \textbf{89.4} & \textbf{92.8} & \textbf{76.9} & \textbf{84.0} & \textbf{86.8} & \textbf{80.1} & \textbf{25.6}\\
\midrule\multicolumn{13}{c}{\textit{Sentence relations prediction}} \\\midrule
InferSent & $4096$ & --- & \textbf{81.1} & \textbf{86.3} & 92.4 & 90.2 & 88.2 & \textbf{76.2} & \textbf{83.1} & \textbf{\underline{88.4}} & --- & ---\\
DisSent Books 5 & $4096$ & --- & 80.2 & 85.4 & 93.2 & 90.2 & 91.2 & 76.1 & --- & 84.5 & --- & ---\\
DisSent Books 8 & $4096$ & --- & 79.8 & 85.0 & \textbf{93.4} & \textbf{\underline{90.5}} & \textbf{\underline{93.0}} & 76.1 & --- & 85.4 & --- & ---\\
\midrule\multicolumn{13}{c}{\textit{Pre-trained transformers}} \\\midrule
\textsc{Bert}-base [CLS] & $768$ & 96 & 78.7 & 84.9 & 94.2 & 88.2 & \textbf{91.4} & 71.1 & --- & 75.7$^\dagger$ & --- & ---\\
\textsc{Bert}-base [NLI] & $768$ & 96 & \textbf{\underline{83.6}} & \textbf{\underline{89.4}} & \textbf{94.4} & \textbf{89.9} & 89.6 & \textbf{76.0} & --- & \textbf{84.4}$^\dagger$ & --- & ---\\
\midrule\multicolumn{13}{c}{\textit{\textbf{Our models (GloVe \& Pretrained Embeddings)}}} \\\midrule
\textsc{Seq}, \textsc{Const}$^\dagger$ & $4800$ & 41 & 79.8 & 82.9 & 94.6 & 88.5 & 90.4 & 76.4 & 83.7 & 86.1 & 78.9 & 26.3\\
\textsc{Dep}, \textsc{Seq}$^\dagger$ & $4800$ & 27 & 79.7 & 82.2 & 94.4 & 88.6 & 91.0 & \textbf{\underline{77.9}} & \textbf{\underline{84.4}} & 86.6 & 79.8 & 25.5\\
\textsc{Dep}, \textsc{Const}$^\dagger$ & $4800$ & 39 & \textbf{80.7} & \textbf{83.6} & \textbf{\underline{94.9}} & \textbf{89.2} & \textbf{92.6} & 76.8 & 83.6 & \textbf{87.0} & \textbf{\underline{80.3}} & \textbf{\underline{24.8}}\\
\bottomrule
\end{tabularx}}
\caption{\textbf{SentEval Task Results Using Fixed Sentence Encoder.} 
We divided the table into sections. The first range of models is directly comparable to our model as the training objective is to identify context sentences. The second section objective is to identify the correct relationship between a pair of sentences. The third section reports pre-trained transformers based-models. The last section reports the results from our models. FastSent is reported from \textcite{hill_16}. Skipthoughts results from \textcite{kiros_15} Skipthoughts + LN which includes layer normalization method from \textcite{ba_16}. We considered the Quickthoughts results \cite{logeswaran_18} with a pre-training on the bookcorpus dataset. DisSent and Infersent are reported from \textcite{nie_19} and \textcite{conneau_17} respectively. Pre-trained transformers results are reported from \textcite{reimers_19}. The \textbf{Hrs} column indicates indicative training time, the \textbf{Dim} column corresponds to the sentence embedding dimension. $^\dagger$\, indicates models that we had to re-train. Best results in each section are shown in \textbf{bold}, best results overall are \underline{underlined}. Performance for \textbf{SICK-R} results are reported by convention as $\rho \text{ and } r \times 100$.}
\labtab{contrastive-soa}
\end{table*}

As usual for models aiming to build generic sentence embeddings \textcite{kiros_15, hill_16, arora_17, conneau_17, logeswaran_18, nie_19}, we use tasks from the SentEval benchmark \parencite{conneau_18}\sidenote{Senteval is posterior to most of the references. However, these studies do evaluate on tasks later included in the benchmark.}. SentEval is specifically designed to assess the quality of the embeddings themselves rather than the quality of a model specifically targeting a downstream task, as is the case for the GLUE and SuperGLUE benchmarks \parencite{wang_19_glue, wang_19_superglue}. Indeed, the evaluation protocol prevents for fine-tuning the model during inference and the architecture to tackle the downstream tasks is kept minimal. Moreover, the embedding is kept identical for all tasks, thus assessing their properties of generalization. 

Therefore, classification tasks from the SentEval benchmark are usually used for evaluation of sentence representations \parencite{conneau_18}: the tasks (presented in \refsec{survey:downstream}) include sentiment and subjectivity analysis (\textbf{MR, CR, SUBJ, MPQA}), question type classification (\textbf{TREC}), paraphrase identification (\textbf{MRPC}) and semantic relatedness (\textbf{SICK-R}). Contrasting the results of our model on this set of tasks will help to better understand its properties.  The MR, CR, SUBJ, MPQA tasks are binary classification tasks with no pre-defined train-test split. We therefore use a 10-fold cross validation. For the other tasks we use the proposed train/dev/test splits.
%We use either the pre-defined train/dev/test splits or perform a 10-fold cross-validation. 
We follow the linear evaluation protocol of \textcite{kiros_15}, where a logistic regression or softmax classifier is trained on top of sentence representations. The dev set is used for choosing the regularization parameter and results are reported on the test set. 

% \paragraph{Results analysis} 

We compare the properties of distinct views combination on downstream tasks. Results are compared with state of the art methods in \reftab{contrastive-soa}. The first set of methods (\textsl{Context sentences prediction}) are trained to reconstruct books storyline. % relies on a distributional hypothesis: models 
The second set of models (\textsl{Sentence relations prediction}) is pre-trained on a supervised task. Infersent \parencite{conneau_17} is trained on the SNLI dataset, which proposes to predict the entailment relation between two sentences. DisSent \parencite{nie_19} proposes a generalization of the method and builds a corpus of sentence pairs with more possible relations between them. Finally, we include models relying on transformer architectures (Pre-trained transformers) for comparison. In particular, \textsc{Bert}-base model and a \textsc{Bert}-model fine-tuned on the SNLI dataset \cite{reimers_19}. 
In \reftab{contrastive-soa}, we observe that our models expressing a combination of views such as (\textsc{Dep}, \textsc{Seq} or (\textsc{Dep}, \textbf{const}) give better results than the use of the same view (\textbf{seq}, \textbf{seq}) used in Quick-Thought model. It seems that the entanglement of views benefits the sentence embedding properties. In particular, we obtain state-of-the-art results for almost every metric from \textbf{MRPC} and \textbf{SICK-R} tasks, which focus on paraphrase identification. For the \textbf{MRPC} task, we gain a full point in accuracy and outperform \textsc{Bert} models. We hypothesize structure is important for achieving this task, especially as the dataset is composed of rather long sentences. The \textbf{SICK-R} dataset is structurally designed to discriminate models that rely on compositional operations. 

This also explains the score improvement on this task. Tasks such as \textbf{MR}, \textbf{CR} or \textbf{MPQA} consist in sentiment or subjectivity analysis. We hypothesize that our models are less relevant in this case: such tasks are less sensitive to structure and depend more on individual word or lexical variation.

\subsection{Impact of the multi-view}
\labsec{impact-multi-view}

We aim to measure the impact of multi-view specifically. Table~\ref{table:multi-view} compares all possible view pairs out of \textsc{Dep}, \textsc{Const} and \textsc{Seq} views. For each multi-view model, we report the average score from SentEval tasks\sidenote{We scale all metrics as percentages. In particular, we use 100 - MSE for the \textbf{SICK-R} task. The final score corresponds to the average of all tasks. We average the scores for tasks with multiple metrics (\textbf{MRPC} and \textbf{SICK-R}).}. The first section of the Table corresponds to \textit{single-view} models, for which both views from the pair are identical. The second section reports multi-view models. 

Multi-view models outperform those using a single view. Given our experiment, it is advantageous to use multiple views instead of one. It also confirms our hypothesis that combining multiple structured models or views yield richer sentence embeddings.

\begin{table}[!htb]
\footnotesize
% \begin{minipage}{\textwidth}
\centering {
\begin{tabularx}{\textwidth}{@{}c | Y@{}}
\toprule
Model & Avg. SentEval Score \\\midrule
\multicolumn{2}{c}{\textit{Single-view models}} \\\midrule
\textsc{Const}, \textsc{Const} & 84.4\\
\textsc{Dep}, \textsc{Dep} & 84.6\\
\textsc{Seq}, \textsc{Seq} & 84.9\\\midrule
\multicolumn{2}{c}{\textit{Multi-view models}} \\\midrule
\textsc{Seq}, \textsc{Const} & 85.1\\
\textsc{Seq}, \textsc{Dep} & 85.3\\
\textsc{Dep}, \textsc{Const} & \textbf{86.0}\\
\bottomrule
\end{tabularx}}
\caption{\textbf{Impact of the multi-view.} The first section corresponds to single-view setups for which $f$ and $g$ are the same views. The second section reports multi-view models. For each model, we report the average score on the SentEval benchmark.}
\label{table:multi-view}
\end{table}

\subsection{Impact of the corpus choice}
\labsec{impact-corpus}

We have chosen to make use of a distinct corpus as the BookCorpus dataset is no longer distributed for copyright reasons. We have run QuickThought scripts \parencite{logeswaran_18} using our dataset based on the UMBC corpus to compare both setups. Results are detailed in the first Section from \reftab{corpus} and are rather close in both configurations. Indeed, except for the \textbf{SUBJ} and \textbf{MR} task, the use of our dataset penalizes the results. Our corpus is indeed restricted to 40M sentences, in comparison with 74M for the Bookcorpus. Regarding the dataset size and the SentEval results, we have considered the comparison holds.

\begin{table*}[!htb]
\footnotesize
% \begin{minipage}{\textwidth}
\centering {
\begin{tabularx}{16cm}{@{}c| Y Y Y Y Y Y Y Y Y Y @{}}
\toprule
\multirow{2}{*}{\textbf{Model}} & \multirow{2}{*}{\textbf{MR}} & \multirow{2}{*}{\textbf{CR}} & \multirow{2}{*}{\textbf{SUBJ}} & \multirow{2}{*}{\textbf{MPQA}} & \multirow{2}{*}{\textbf{TREC}} &  \multicolumn{2}{c}{\textbf{MRPC}} &  \multicolumn{3}{c}{\textbf{SICK-R}}\\%\cmidrule(r){9-10} \cmidrule(r){11-13}
 &  &  &  &  &  & \textbf{Acc} & \textbf{F1} & \textbf{$r$} & \textbf{$\rho$} & \textbf{MSE}\\\midrule
 \multicolumn{11}{c}{\it Impact of the pretraining corpus on QuickThought} \\\midrule
Quickthoughts (results from paper) & 80.4 & \textbf{85.2} & 93.9 & \textbf{89.4} & \textbf{92.8} & \textbf{76.9} & 84.0 & \textbf{86.8} & \textbf{80.1} & \textbf{25.6}\\
Quickthoughts (UMCB 40M)$^\dagger$ & \textbf{80.9} & 84.4 & \textbf{95.1} & 88.9 & 92.2 & 75.8 & --- & 86.0 & --- & ---\\\midrule
 \multicolumn{11}{c}{\it Impact of the embedding size} \\\midrule
{\sc Bert}-base [CLS] $^\dagger$& \textbf{77.3} & 81.3 & 92.7 & 85.0 & 80.2 & 69.9 & --- & 61.0 & --- & ---\\
{\sc Bert}-base [CLS] /w random projection $^\dagger$& 77.1 & \textbf{82.6} & \textbf{93.1} & \textbf{85.9} & \textbf{80.8} & \textbf{71.3} & --- & \textbf{71.0} & --- & ---\\\midrule
 \multicolumn{11}{c}{\it Impact of pre-training}\\\midrule
\dep{}, \const{}$^\dagger$ & \textbf{80.7} & \textbf{83.6} & \textbf{94.9} & \textbf{89.2} & \textbf{92.6} & \textbf{76.8} & \textbf{83.6} & \textbf{87.0} & \textbf{80.3} & \textbf{24.8}\\
Rand \sc{LSTM} & 77.2 & 78.7 & 91.9 & 87.9 & 86.5 & 74.1 & --- & 86.0 & --- & ---\\
\bottomrule
\end{tabularx}}
\caption{\labtab{corpus}Study on SentEval task results: the first section compares the impact of the training dataset for QuickToughts. The next section focuses on the impact of the embedding size. To this end, hidden representations are projected into a larger embedding space using a random, fully connected layer. The final Section compares models randomly initialized with those pre-trained on our self-supervised task. $^\dagger$ indicates models that we had to re-train.}
\end{table*}

\subsection{Biases toward embedding size}
\labsec{impact:embedding-size}

As exposed in \refsec{training:supervised}, SentEval evaluation framework is suspected to suffers from biases toward the embedding size \parencite{eger_19}. Moreover, some works on sentence embedding evaluation methods points surprising good results may be achieved using randomly initialized encoders \parencite{wieting_19}. We provide extra analysis to discuss these potential pitfalls.

Regarding the dependency on the embedding size, we run experiments to analyze if such bias could explain \textsc{Bert} low performances on SentEval since the output hidden size is only of $768$. Following the protocol from \textcite{wieting_19}, we project the embedding from the \textsc{CLS} token using a random matrix initialized with a glorot distribution. This setup expands \textsc{Bert} embedding into 4096 dimensions. We reported the results in \reftab{corpus}. We observe expanding the embedding size seems to slightly improve the results. However, the results are still below Quickthought vectors by a large margin.

Regarding the effect of randomly initialized encoders \parencite{wieting_19}, we reported the results in \reftab{corpus}. Although randomly initialized encoders achieve surprisingly good results, they are still below our results obtained with pre-training.

\section{Conclusion and future work}

Inspired from linguistic insights and supervised learning, we hypothesize that structure is a central element to build sentence embeddings. The novelty here is detailed in \refsec{structure:method} and consists in jointly learning structured models in a contrastive framework. In \refsec{structure:experiments} we evaluate the standalone sentence embeddings and use them as a feature for the dedicated SentEval benchmark. We obtain state-of-the-art results on tasks which are expected, by hypothesis, to be more sensitive to sentence structure. We show in \refsec{impact-multi-view} that multi-view embeddings yield better downstream task results. Our setup confirms our hypothesis that combining diverse structures should be more robust for tasks requiring to perform complex compositional knowledge.
% \setchapterstyle{kao}
\setchapterpreamble[u]{\margintoc}
\chapter{Jointly learning model structure and compositional operations}
\labch{tree}

% \newcommand{\bcomment}[2]{{\bfseries\color{red} #1} {\bfseries\color{blue}#2}}

\cleanchapterquote{Give orange me give eat orange me eat orange give me eat orange give me you.}{Nim Chimpsky}{Male chimpanzee, 1979}

% \bcomment{}{Yep. the neural models cannot handle quantification naturally. Change the example to sentiment analysis and maybe refer to previous chapters on differences between logical representations and neural representations. Problematize : You might also state what gains you expect from structuring a neural model with a tree shape rather than using say an LSTM.}

This chapter examines the possibility of including tree-like structural bias in neural models while minimizing or eliminating direct structure supervision. 

There is this strong hypothesis in computational linguistics that language has a recursive structure \parencite{chomsky_56}. Thus, computing sentence semantic representations traditionally calls for a recursive compositional function whose structure is tree-shaped. As illustrated in \reffig{tree:tree-expl}, we can use a sentence structure as support to compute semantic representations—in this case, FOL statements, as introduced in \refsec{meaning:formal}. When using vector representations, we can also use the structure as support to encode the sentence. Following this direction, \textcite{socher_13} introduce the Stanford sentiment treebank: a corpus with fully labeled parse trees that can be used to analyze the compositional effects of sentiment in language. The dataset provides fine-grained information about lexical units carrying positive or negative sentiment. As illustrated in \reffig{tree:tree-expl}, a sentiment prediction system cannot predict the sentiment of a given sentence by simply averaging the sentiment carried by each word. We can only infer this sentiment by analyzing the sentence's structure together with individual words.

\begin{figure*}[htb!]
    \centering
    \begin{subfigure}[b]{8cm}
        \centering
        \includegraphics[width=7.5cm]{images/lambda_form.png}
    \end{subfigure}
    \hfill
    \begin{subfigure}[b]{8cm}  
        \centering 
        \includegraphics[width=7.5cm]{images/socher-tree-5.png}
\end{subfigure}
\caption{\labfig{tree:tree-expl}\textit{(left)} We illustrate a Montague-style \parencite{montague_1973} derivation of a semantic representation for the sentence "Every man loves a woman." We extracted the figure from \url{https://www.coli.uni-saarland.de/~koller/papers/sem-handbook.pdf}. \textit{(right)} We extracted a sentence from the Stanford sentiment treebank. The sample provides annotation for each word contribution to the final sentence sentiment. There are two parts in the sentence "There are slow and repetitive parts, but it has just enough spice to keep it interesting.", respectively negative and positive. The final sentence ends up positive. We adapted the figure from \url{http://nlp.stanford.edu:8080/sentiment/rntnDemo.html}.}
\end{figure*}

\textcite{socher_13} propose to combine the Stanford sentiment treebank with recursive neural networks. We already introduced such architectures in \refsec{architectures:tree} and successfully used them in \refch{structure-scale}. Recursive neural networks represent a phrase using word vectors and a parse tree. They compute parent vectors in a bottom-up fashion using the children as input arguments from a composition function. The composition function is shared across all node computations. \textcite{socher_13} show that, unlike bag of words, recursive networks can capture the scope of negation and sentiment change induced by contrastive conjunctions such as "but".

However, not everyone has access to resources as rich as the Stanford sentiment treebank. The corpus is only available in English and requires precise annotations from experts in linguistic. This chapter investigates the possibility of incorporating tree structural biases with minimal explicit supervision. To this end, we propose a model that jointly parses sentences into discrete trees and composes a semantic vector along with these trees.

We organize our chapter as follows: \refsec{tree:related-work} reviews related models, learning the composition function together with the sentence structure. \refsec{sec:model} introduces our model, which is based on well-known components and could therefore accommodate a variety of parsing architectures such as graph parsers or attention matrices from \textsc{Bert}. In \refsec{sec:eval}, we train and evaluate the full model with distant downstream supervision on textual entailment and semantic similarity tasks. Finally, in \refsec{sec:parser-init}, we analyze how the initial parser supervision impacts the learned structures and the performance on downstream tasks.


% Using syntax driven semantic analysis \parencite{jurafsky_2009}, we can map a sentence to a logical form that reflects its meaning—following the definition from \refsec{meaning:formal}. In this perspective, \bcomment{nonsense}{} we compute the sentence meaning using the compositionality principle. In its essence, the principle states that we can derive the meaning of a sentence by composing the meaning of its parts. \bcomment{That's an operational statement}{} We first infer the sentence structure\sidenote{The structure may consist in various trees such as constituency, dependency or binary parses.}. We then recursively compute the semantic representation along the structure, by \bcomment{rephrase the sentence}{} augmenting and combining each node with semantic attachments. These attachments follow explicit combination rules.





% We propose a framework relying explicitly on syntactic trees rather than analyzing the patterns learned by statistical models. 

% To compute semantic representations, tree structured models rely on an explicit and discrete structure. It favors the integration of linguistic information and inductive biases. Moreover, it favors the analysis of the composition since it explicitly relies on syntactic trees. Yet, such models require \bcomment{raw text and linguistic structure}{not only raw text but also linguistic structure} in the form of parse trees to calculate the semantic representation. This prerequisite limits their use in practice \bcomment{}{because it requires annotations in the supervised case}. By using tree structure neural models without the need for labeled parses, we hope to overcome this limitation.

% We evaluate our model on textual entailment and semantic similarity tasks and outperform sequential models and tree-structured models relying on external parsers. Moreover, when initialized on human-annotated structures, our model improves inference close to \textsc{Bert} base performances on the semantic similarity task. 

% We then conduct an ablation study to quantify the impact of the parser initialization on the resulting structures and downstream performances. We corroborate that the sole use of downstream supervision is insufficient to produce parses that are easy to interpret. To encourage convergence towards readable linguistic structures, we examine a number of initialization setups. We observe that our structures often converge toward trivial branching patterns, which have few in common with gold linguistic parses. However, with respect to the downstream performances, linguistic insights appear as a relevant initialization.

% \section{Principle of compositionality for building semantic representations}

% \begin{figure}[htb!]
% 	\includegraphics[width=10cm]{images/lambda_form.png}
% 	\caption[Lambda form]{A Montague-style \parencite{montague_1973} derivation of a semantic representation for the sentence “Every man loves a woman.” Credits: by Alexander Koller and Manfred Pinkal \url{https://www.coli.uni-saarland.de/~koller/papers/sem-handbook.pdf}}
% 	\labfig{lambda-form}
% \end{figure}

% There is this strong \bcomment{intuition}{hypothesis} \bcomment{in natural language processing}{computational linguistics} that language has a recursive structure \parencite{chomsky_56, shen_19}. As illustrated in \reffig{lambda-form} computing sentence semantic representations traditionally calls for a recursive compositional function whose structure is tree-shaped. Using syntax driven semantic analysis \parencite{jurafsky_2009}, we can map a sentence to a logical form that reflects its meaning—following the definition from \refsec{meaning:formal}. In this perspective, \bcomment{nonsense}{} we compute the sentence meaning using the compositionality principle. In its essence, the principle states that we can derive the meaning of a sentence by composing the meaning of its parts.
% \bcomment{That's an operational statement}{} We first infer the sentence structure\sidenote{The structure may consist in various trees such as constituency, dependency or binary parses.}. We then recursively compute the semantic representation along the structure, by \bcomment{rephrase the sentence}{} augmenting and combining each node with semantic attachments. These attachments follow explicit combination rules.

% \begin{figure}[htb!]
% 	\includegraphics[width=10cm]{images/tree_rnn.png}
% 	\caption[Tree LSTM]{Semantic encoding of a sentence using a tree-structured neural model.}
% 	\labfig{tree-rnn}
% \end{figure}

% As detailed in \refsec{training:architectures}, computing sentence semantic representations traditionally calls for a recursive compositional function whose structure is tree-shaped. 
% \bcomment{}{Yep. the neural models cannot handle quantification naturally. Change the example to sentiment analysis and maybe refer to previous chapters on differences between logical representations and neural representations. Problematize : You might also state what gains you expect from structuring a neural model with a tree shape rather than using say an LSTM.} Neural models can emulate this composition procedure using tree-structured networks. As for the logical form, such models proceed in two stages: first we infer the syntactic structure of the sentence, then we recursively compose the semantic representation along the structure. We illustrate the composition procedure in \reffig{tree-rnn}. Here, the semantic representations consist of vectors composed with linear algebra operations—as detailed in \refsec{meaning:distributional}. The parameters of the semantic algebraic operators are \textit{learned} by optimizing a loss function given a dataset. Regarding the sentence structure, we usually infer it using a dedicated parser, learned on labeled data.

% and enable better generalization and abstraction properties.

%%%%%%%%%%%%%%%%%%%%%%%%%%%%%%%%%%%%%%%%%%%%%%%%%%%%%%%%%%%%%%%%%%%%%%%%%%%%%%%%
\section{Latent tree learning}
\labsec{tree:related-work}
%%%%%%%%%%%%%%%%%%%%%%%%%%%%%%%%%%%%%%%%%%%%%%%%%%%%%%%%%%%%%%%%%%%%%%%%%%%%%%%%

Tree-structured models rely on an explicit and discrete structure to compute semantic representations. It favors the integration of linguistic information and inductive biases. Moreover, it favors compositional analysis since it explicitly relies on syntactic trees. However, such models require not only raw text but also linguistic structure in the form of parse trees to calculate the semantic representations. This prerequisite limits their use in practice because it requires annotations in the supervised case. 
% By using tree structure neural models without the need for labeled parses, we hope to overcome this limitation.

% \bcomment{not really true}{in the supervised setting} Tree-based models need carefully hand-annotated data to be trained. 
One method of overcoming this limitation is to induce trees from raw text and computes semantic representations along with the inferred structure. Such a method preserves explicit recursive computations and produces intelligible tree structures. Known as \textit{latent tree learning}, these methods generally consist of two components: a parser and a composition function that uses the parses. The parser and composition function are learned jointly and are specific to a given task or domain. 
% The learning method is called \textit{latent tree learning}, and it \bcomment{usually}{that's not usual} consists of two components: a parser and a \textsc{TreeLSTM} that uses those parses. 

% All models differ from the syntactic formalism used and the training method.

The first set of latent tree models introduces an intermediate objective to train the parser component. \textcite{socher_11c} parse the sentence by selecting and merging adjacent nodes. The parser model is trained using an auxiliary reconstruction task. The Shift reduce Parser-Interpreter Neural Network (SPINN) model from \textcite{bowman_16} obtains the structure using a shift-reduce parser. The parser component also uses an intermediate objective that compares parses with gold-standard trees.

% defines a relaxation of a CYK-like chart parser 
\textcite{maillard_19} explicitly compute a whole forest of
%$\mathcal{O}(N^2)$ \bcomment{incorrect combinatorics}{}
potential binary parse trees for a sentence of $N$ words. 
%\sidenote{The number of possible binary trees with $n-1$ leafs is given by the nth Catalan number, $C_n = \frac{(2n)!}{(n+1)!n!} \sim \frac{4^n}{\sqrt{\pi}n^{\frac{3}{2}}$ \parencite{vardi1991computational}.}
All possible partial trees of a sentence are stored in a chart data structure inspired by the CYK parser. The final tree is constructed as a soft combination of the constituents available in each chart cell, thus approximating discrete candidate selection and making the model entirely trainable using backpropagation. However, the linear increase in candidates with depth makes this algorithm memory intensive.


% \textcite{maillard_19} introduce a model entirely trainable using backpropagation. However, the model does not rely on a soft structure and does not maintain the discreteness of the tree composition process\bcomment{a bit elliptic. what do they do really ?}{more details would be welcome}. It is, therefore, memory intensive as it explicitly computes a whole forest of 
% % $\smallO{N^2}$ 
% potential trees. 
% %for $N$ words

% \citet{yogatama_17} is the first model to jointly train its parsing and sentence embedding components. They base their model on shiftreduce parsing. Their parser is not differentiable, so they rely on reinforcement learning for training.

% \citet{yogatama_17} adapt the Spinn model with the REINFORCE algorithm. \citet{yogatama_17} introduce reinforcement learning to achieve the desired effect of discretization. However, slow convergence due to the reinforcement learning setting is one of its draw- backs, according to the authors.

% Our main idea is to use reinforcement learning (policy gradient methods) to discover the best tree structures for the task that we are interested in. We do not place any kind of restrictions when learning these structures other than that they have to be valid binary parse trees, so it may result in tree structures that match human linguistic intuition, heavily right or left branching, or other solutions if they improve performance on the downstream task. We parameterize each action a ∈ {SHIFT, REDUCE} by a policy network
% guided composition orders of Tree LSTM models are given directly as input

\textcite{yogatama_17} adapt the training of the \textsc{SPINN} model to make it fully differentiable. As such, the model does not require any structure supervision during training. Instead of providing the model with the parse of the input, the procedure uses reinforcement learning (policy gradient methods) to discover the best tree structures for the task. However, the reinforcement learning strategy is notoriously slow and limits the convergence speed.

% \textcite{yogatama_17} adapt \textsc{SPINN} and jointly train its parsing and sentence embedding components. The model produces a discrete structure, but the resulting architecture is not fully differentiable and must thus be trained using a reinforcement learning proxy objective, limiting the convergence speed. 

% \citet{choi_18} use an approach similar to easy-first parsing. The parsing decisions are discrete, but the authors use the Straight-Through Gumbel-Softmax estimator to obtain an approximate gradient and are thus able to train with backpropagation.

% \citet{choi_18} present a model (ST-Gumbel) that uses a similar data structure and gating strategy to \citet{maillard_19}, but which uses the Straight-Through Gumbel-Softmax estimator. This allows them to use a hard categorical gating function, so that their output sentence vector is computed according to a single tree, rather than a gated combination of partial trees as in \citet{maillard_19}.

% \textcolor{blue}{ \citet{choi_18} compute a single tree instead of combinations from partial trees by using the Straight-Through Gumbel-Softmax estimator. The input is greedily and sequentially parsed.}
% . However, \citet{choi_18} model iterates through $N - 1$ steps to build a tree over $N$ words, }

\textcite{choi_18} propose a method that is both fully differentiable and maintains the discreteness of the parsing process. Contrary to \textcite{yogatama_17}, it does not require the reinforcement learning artifice for training; contrary to \textcite{maillard_19}, it computes a single discrete tree instead of combinations from partial trees. The methods proceed in $N - 1$ iterations to build a tree over $N$ words. Using the Gumbel-Softmax estimator, two nodes are selected from the available candidates at each iteration. During the forward pass, the estimator is used as a discrete argmax function to select nodes to merge. During the backward pass, The estimator relaxes the discrete sampling operation so that it can be trained with backpropagation.

% compute a single tree instead of combinations from partial trees by using the Gumbel-Softmax estimator. The input is greedily and sequentially parsed.
% \bcomment{overall at the end of this section, I miss some more details on what these guys really do. It requires to read the papers to get an idea. Lacks of self containment}{}


% a reconstruction error in \citet{socher_11c} and a comparison with gold-standard trees in \citet{bowman_16}.}

%A \treelstm{} is then applied on the parse to embed the sentence. But both models train the parser with an auxiliary task.
%: a reconstruction error and a comparison with gold-standard trees.

% \citet{bowman_16} introduce the Shift-reduce Parser-Interpreter Neural Network (SPINN). However as in \cite{socher_11c} the model train the parsing module with an auxiliary loss to match gold-standard trees. 

%%%%%%%%%%%%%%%%%%%%%%%%%%%%%%%%%%%%%%%%%%%%%%%%%%%%%%%%%%%%%%%%%%%%%%%%%%%%%%%%%%%%%%%%
% 2. Yogatama
%%%%%%%%%%%%%%%%%%%%%%%%%%%%%%%%%%%%%%%%%%%%%%%%%%%%%%%%%%%%%%%%%%%%%%%%%%%%%%%%%%%%%%%%

% \citet{yogatama_17} is the first model to jointly train its parsing and sentence embedding components. They base their model on shiftreduce parsing. Their parser is not differentiable, so they rely on reinforcement learning for training.

% \citet{yogatama_17} adapt the Spinn modelwith the REINFORCE algorithm. \citet{yogatama_17} introduce reinforcement learning to achieve the desired effect of discretization. However, slow convergence due to the reinforcement learning setting is one of its draw- backs, according to the authors.

%%%%%%%%%%%%%%%%%%%%%%%%%%%%%%%%%%%%%%%%%%%%%%%%%%%%%%%%%%%%%%%%%%%%%%%%%%%%%%%%%%%%%%%%
% 4. Maillard
%%%%%%%%%%%%%%%%%%%%%%%%%%%%%%%%%%%%%%%%%%%%%%%%%%%%%%%%%%%%%%%%%%%%%%%%%%%%%%%%%%%%%%%%
% \citet{maillard_19} propose an alternative approach, inspired by CKY parsing. The algorithm is made differentiable by using a soft-gating approach, which approximates discrete candidate selection by a probabilistic mixture of the constituents available in a given cell of the chart. This makes it possible to train with backpropagation.

% \citet{maillard_19} present a model, entirely trainable using  backpropagation — which explicitly computes O(N2) possible tree nodes for N words, and uses a soft gating strategy to approximately select valid combinations of these nodes that form a tree. Though their model reduces the ambiguity by explicitly representing a node as a weighted sum of all candidate compositions, it is memory intensive since the number of candidates linearly increases by depth.
% \textcolor{blue}{}

%%%%%%%%%%%%%%%%%%%%%%%%%%%%%%%%%%%%%%%%%%%%%%%%%%%%%%%%%%%%%%%%%%%%%%%%%%%%%%%%%%%%%%%%
% 5. Choi
%%%%%%%%%%%%%%%%%%%%%%%%%%%%%%%%%%%%%%%%%%%%%%%%%%%%%%%%%%%%%%%%%%%%%%%%%%%%%%%%%%%%%%%%

% \citet{choi_18} use an approach similar to easy-first parsing. The parsing decisions are discrete, but the authors use the Straight-Through Gumbel-Softmax estimator to obtain an approximate gradient and are thus able to train with backpropagation.

% \citet{choi_18} present a model (ST-Gumbel) that uses a similar data structure and gating strategy to \citet{maillard_19}, but which uses the Straight-Through Gumbel-Softmax estimator. This allows them to use a hard categorical gating function, so that their output sentence vector is computed according to a single tree, rather than a gated combination of partial trees as in \citet{maillard_19}.

% \textcolor{blue}{ \citet{choi_18} compute a single tree instead of combinations from partial trees by using the Straight-Through Gumbel-Softmax estimator. The input is greedily and sequentially parsed.}
% . However, \citet{choi_18} model iterates through $N - 1$ steps to build a tree over $N$ words, }

%%%%%%%%%%%%%%%%%%%%%%%%%%%%%%%%%%%%%%%%%%%%%%%%%%%%%%%%%%%%%%%%%%%%%%%%%%%%%%%%%%%%%%%%
%. 6. Williams
%%%%%%%%%%%%%%%%%%%%%%%%%%%%%%%%%%%%%%%%%%%%%%%%%%%%%%%%%%%%%%%%%%%%%%%%%%%%%%%%%%%%%%%%
% \citet{williams_18} investigate the trees produced by \citet{yogatama_17} and \citet{choi_18} when trained on two natural language inference corpora. They find that the former model induces almost entirely leftbranching trees, while the latter performs well but has inconsistent trees across re-runs with different parameter initializations.

% However, \citet{williams_18} empirically show that, Gumbel softmax produces unstable latent trees with the same hyper-parameters but different initializations, while reinforcement learning even tends to generate left-branching trees. Neither gives meaningful latent trees in syntax, but each method still obtains considerable improvements in performance. This indicates that syntax may not be the main contributor to the performance gains.
% We obtain a score close to the {\sc Spinn} model but without an explicit parsing objective. Rather we pre-trained the parsing model on the Penn Tree Bank and fine-tuned it on the SNLI task.


% We organized our paper as follows: after presenting the related in work in Section~\ref{sec:related-work}, we present our model in Section~\ref{sec:model}. 
% We then study the relevance of this framework for semantic inference and analyze the properties of such tree-structured models for downstream tasks. 
% In Section~\ref{sec:eval}, we evaluate our model on textual entailment and semantic similarity tasks. Regarding the textual similarity task, we show that our setup is competitive with \bert base, although the latest is trained on datasets many orders of magnitude larger. We then conduct an ablation study and analyze the impact of the parser initialization. In Section~\ref{sec:parses-impact}, we compare the learned structures across initializations and with interpretable annotations. In Section~\ref{sec:dowstream-impact} we study how latent structures impact performances on downstream tasks.
% with a significantly smaller training set.

% I focus on tree-structured neural networks, which naturally encode the structure of language. For each sentence, the network computes text units following a syntactic tree, starting from the leaf nodes, up to the root. However, such models suffer from practical constraints that limit their application. In particular, tree-based models not only require raw text as input but also the sentence structure in the form of a parse tree. Such structure may be tedious to obtain as it requires manual annotations and external parsers. To overcome such limitations, I formulated a novel tree-based model that learns its composition function together with its structure \parencite{simoulin_2020}. The model includes two modules, a biaffine graph parser, and a Tree-LSTM. The parsing and the composition functions are explicitly connected and, therefore, learned jointly. The method differs from previous work as the representation is not computed from the whole forest of potential trees. Moreover, training the full model directly does not require supervision from a parsing objective. The model outperforms tree-based models relying on external parsers on downstream tasks. In some configurations, it is even competitive with BERT-base model.

\section{Unified parsing and compositional model}
\labsec{sec:model}

% However, architectures often have practical limitations, requiring either complex learning paradigm such as reinforcement learning \parencite{yogatama_17} or intensive computations \parencite{maillard_19}.
The earlier architectures listed above have practical limitations, requiring either a complex learning paradigm such as reinforcement learning, intensive computations, or requiring an external parser module. The method proposed in \textcite{choi_18} overcomes these limits. However, \textcite{williams_18} investigate the latent trees produced by \textcite{yogatama_17} and \textcite{choi_18} and show neither method produces meaningful syntactic representations. The Gumbel-Softmax estimator outputs inconsistent trees across initializations while reinforcement learning outputs trivial left-branching trees. Moreover, \textcite{choi_18} produce trees by selecting and merging adjacent nodes. Therefore, it cannot directly use architectures designed for standard parsing formalisms such as dependency parsing algorithms.

% To limit the need for an external parser, w
In this section, we propose an original latent tree learning method. Besides addressing all the limitations listed above, our method relies on existing and well-known components. It is not limited to a particular parser architecture as long as it is differentiable. Ultimately, our method offers the following benefits:
\begin{itemize}
    \item infers an explicit tree structure and trains recursively a sentence embedding model within a unified architecture;
    \item provides end-to-end training by back-propagating the downstream task loss through the entire architecture;
    \item produces a discrete tree;
    \item accommodates any any graph-based dependency parser architecture.
\end{itemize}


% The \textsc{TreeLSTM} recursive composition function crucially uses a weighted sum of the child representations whose weights are provided by the parser edges, hence linking the parser outputs to the \textsc{TreeLSTM} recursion.
% \footnote{The model is illustrated in Appendix~\ref{sec:figure}.}.

% Our model jointly performs sentence parsing and the prediction of a sentence embedding. The sentence embedding is predicted by a weighted \textsc{TreeLSTM} whose tree structure is provided by a dependency parser.

% We use a standard dependency parsing structure, obtained using a graph-based biaffine dependency parser \parencite{dozat_17}. However, our model is not limited to a particular parser architecture as long as it is differentiable. This flexibility gives us the freedom to explore the impact of the parser choice. 

%\bcomment{}{State more clearly that you designed a joint model by contrasting with others} 
Our model jointly performs sentence parsing and the prediction of a sentence embedding. The sentence embedding is predicted by a \textsc{TreeLSTM} whose tree structure is provided by a dependency parser.

% \bcomment{apparent contradiction, you said earlier that using an external parser is a weakness}{}
\paragraph{Parsing model} We use a standard dependency parsing structure, obtained using a graph-based biaffine dependency parser \parencite{dozat_17}. Given an input sequence of $n$ words, the parser computes a weighted matrix of size $n \times n$ for which each coordinate $(i, j)$ is interpreted as a score for the $i$th word to be the head of the $j$th word. Given the un-normalized score matrix, the predicted tree is extracted using the MST algorithm. However, our model is agnostic to any graph-based parser architecture. This flexibility gives us the freedom to explore the impact of the parser choice (\refsec{sec:parser-init}).

The procedure is formalized in two steps. First, in Eq. \ref{eq:biaffine-1} to \ref{eq:biaffine-3},
it computes a weight matrix that is interpreted as weighted directed graph whose nodes are the sentence tokens:
%\par\noindent{\small
\begin{align}
   \text{Biaff}(x_1,x_2) &=  x_1^T U x_2 + W^{(b)}(x_1 \oplus x_2)+b^{(b)} \\
    a_k^{(dep)} &= W^{(dep)}h_k + b^{(dep)} \label{eq:biaffine-1} \\
    a_j^{(head)} &= W^{(head)}h_j + b^{(head)} \label{eq:biaffine-2}\\
    s^{(arc)}_{kj} &= \text{Biaff}(a_k,a_j) \label{eq:biaffine-3}
\end{align}%}
The second step performs parsing by computing a maximum spanning tree from the graph. As in \textcite{dozat_17}, we use the Max Spanning Tree (MST) algorithm \parencite{chu1965shortest, edmonds_67} 
%\bcomment{cite Edmonds as it is a non standard -directed- MST algorithm}{} 
to ensure the well-formedness of the tree: 
\begin{align}
 \alpha_{kj} &= \mathbb{1}_{mst(s^{(arc)}_{kj})} s^{(arc)}_{kj} \label{eq:biaffine-alpha}
\end{align}
Where $\alpha_{kj}$ is the probability of the edge linking node $j$ to node $k$. For a given node $k$, there is at most one non-zero edge leading to its governor $j$.
% weighted edge

\paragraph{Composition function}  Given a predicted tree structure, we recursively encode the sentence using a variant of the Child Sum Tree model from \textcite{tai_15} detailed below:
% \par\noindent{\small

\begin{align}
\tilde{h}_j &= \sum_{k \in C(j)} \alpha_{kj} h_k, \label{eq:treelstm-weighted} \\
i_j &=\sigma \left( W^{(i)} x_j + U^{(i)} \tilde{h}_j + b^{(i)} \right), \\
o_j &=\sigma \left( W^{(o)} x_j + U^{(o)} \tilde{h}_j + b^{(o)} \right), \\
u_j &=\tanh \left( W^{(u)} x_j + U^{(u)} \tilde{h}_j + b^{(u)} \right), \\
f_{jk} &= \sigma\left( W^{(f)} x_j + U^{(f)} h_k + b^{(f)} \right), \label{eq:treelstm-f}\\
c_j &= i_j \odot u_j + \sum_{k\in C(j)} f_{jk} \odot c_{k}, \\
h_j &= o_j \odot \tanh(c_j), \label{eq:treelstm-last}
\end{align}


% \begin{align}
% \tilde{h}_j &= \sum_{k \in C(j)} \alpha_{kj} h_k, \label{eq:treelstm-weighted} \\
% %\tilde{h}_j &= \sum_{k \in C(j)} h_k, \label{eq:treelstm-first} \\
% i_j, o_j, u_j &=\sigma \Big( W^{(i, o, u)} x_j + U^{(i, o, u)} \tilde{h}_j + b^{(i, o, u)} \Big), \\
% % o_j &= \sigma \left( W^{(o)} x_j + U^{(o)} \tilde{h}_j  + b^{(o)} \right), \\
% % u_j &= \tanh\left( W^{(u)} x_j + U^{(u)} \tilde{h}_j  + b^{(u)} \right), \\
% f_{jk} &= \sigma\left( W^{(f)} x_j + U^{(f)} h_k + b^{(f)} \right), \label{eq:treelstm-f}\\
% c_j &= i_j \odot u_j + \sum_{k\in C(j)} f_{jk} \odot c_{k}, \\
% h_j &= o_j \odot \tanh(c_j), \label{eq:treelstm-last}
% \end{align}
%}
Where in Eq.~\ref{eq:treelstm-weighted}, $C(j)$ denotes the set of children of node $j$.
% and $k \in C(j)$. 

% The \textsc{TreeLSTM} recursive composition function crucially uses a weighted sum of the child representations whose weights are provided by the parser edges, hence linking the parser outputs to the \textsc{TreeLSTM} recursion.

%\paragraph{Discussion}
We use the embedding computed by the weighted \textsc{TreeLSTM} at the root of the tree as the sentence embedding.
% The core model outputs a sentence embedding computed by a weighted \treelstm{}.  
% a compositionnaly weighted tree-{\sc lstm}
The tree shape and the edge weights are given by the best prediction of a graph parser. The equations from the \textsc{TreeLSTM} are the same than the one presented in \refsec{architectures:tree}, except for Eq.~\ref{eq:treelstm-weighted}. \textbf{Crucially, in our case, Eq.~\ref{eq:treelstm-weighted} is a \underline{weighted} sum rather than a standard sum and the weights are those $\alpha_{kj}$ provided by the parser.} The parsing model is linked to the \textsc{TreeLSTM} by the weights $\alpha_{kj}$. This architecture allows us to update jointly the parser and the \textsc{TreeLSTM} weights using only the downstream task loss. The supervision comes only from the objective of the downstream task, and no intermediate structure target is required.
% It does not require training the parser component using an auxiliary task;
%our model supervision only appears in the semantic objective and no intermediate structure target is required. 
% the architecture allows us to update the parser and \treelstm{} weights using solely the loss provided by the downstream task. 

% It therefore differs from models like {\sc Spinn} \cite{bowman_16}, which includes direct supervision from a parsing objective.

% The method described here differs from previous work as the representation is not computed from the whole forest of potential trees \citep{yogatama_17, maillard_19, williams_18, shen_18, liu_18} but instead on the single best tree while being completely differential. Thus, it can be trained using back-propagation on a supervised task for which we want to provide relevant sentence embeddings. The architecture is, in principle, generic and adaptable to any other downstream task. 

Our model, illustrated in \reffig{biaffine-tree-lstm}, is fully differentiable and preserves the discreteness of the tree composition process. It relies on a dependency parsing formalism and could accommodate any graph-based dependency parser.
%\bcomment{again it is too strong}{it requires that the parsing algorithm encode edges explicitly}
Intuitively, the model induces a direct link between the inference of the syntactic structure and the composition of the semantic representation. If a connection between two nodes $i$ and $j$ is irrelevant from a semantic standpoint, then its contribution $\alpha_{kj}$ into the construction of the hidden state $\tilde{h}_j$ (Eq.~\ref{eq:treelstm-weighted}) is likely to be marginal. When training the model, if such connection becomes too tenuous, it becomes unlikely to be selected when selecting the maximum spanning tree from the graph in equation \ref{eq:biaffine-alpha}.
% \bcomment{problems with refs to equations}{}


\begin{figure}[!ht]
	\includegraphics[width=7cm]{images/biaffine-12.png}
	\caption[Biaffine tree lstm]{We illustrate the architecture detailed in Eq. \ref{eq:biaffine-1} to \ref{eq:treelstm-last}. The Biaffine parser provides the sentence structure from which the \textsc{TreeLSTM} computes sentence embeddings. The full pipeline is differentiable as the \textsc{TreeLSTM} weights are given by the parser.}
	\labfig{biaffine-tree-lstm}
\end{figure}

%%%%%%%%%%%%%%%%%%%%%%%%%%%%%%%%%%%%%%%%%%%%%%%%%%%%%%%%%%%%%%%%%%%%%%%%%%%%%%%%
\section{Evaluation on downstream tasks}
\labsec{sec:eval}
%%%%%%%%%%%%%%%%%%%%%%%%%%%%%%%%%%%%%%%%%%%%%%%%%%%%%%%%%%%%%%%%%%%%%%%%%%%%%%%%

Our architecture primarily aims to produce relevant embeddings for downstream tasks. To this end, we compare our setup with other models from the literature on various tasks. For this comparison, we first pre-train the parsing submodel on human-annotated sentences from the Penn Tree Bank (PTB) \parencite{marcus_94} converted to Stanford dependencies. We then fine-tune the parser's parameters on the task while training the full model.\sidenote{In this configuration, we observe pre-training the parser may cause weights $\alpha$ to become too large in Eq. \ref{eq:biaffine-alpha}. This leads to poor downstream performances. We correct this with a multiplicative parameter $\tau$ whose value is estimated during training. It means we replace Eq. \ref{eq:biaffine-alpha} with: $\alpha_{kj} = \tau \cdot \mathbb{1}_{mst(s^{(arc)}_{kj})} s^{(arc)}_{kj}$ for the computation of tree weights.}


\subsection{Semantic textual similarity (STS)}
\labsec{sts}

We first evaluate our model on the SICK-R downstream task \parencite{marelli_14}, which is dedicated to assessing models' compositional properties. 
%The task intends to discriminate models that rely on shallow lexical pattern matching from those that take advantage of the sentence's underlying structure. 
The dataset comprises 9,927 sentence pairs, distributed in a \numprint{4500}/500/\numprint{4927} train/dev/test split, annotated for semantic similarity on a 1 to 5 real range. A score of 5 means that the two sentences are completely equivalent and 1 that the two sentences are completely dissimilar. In between, their degree of equivalence differs given the proportion of shared topics and information between the two sentences. The dataset includes specific examples of variations on passive and active forms, quantifier and modifier switches, or negations. We extensively present the construction of the dataset in \ref{sec:linguistic-breakdown} and give some illustration examples of the task in \reftab{senteval:examples:sick}.
%This is a SentEval task \cite{conneau_18} specifically
%\bcomment{to get a more self contained doc you might present the task with slightly more details, provide examples etc.}{}

\begin{table*}[!htb]
\footnotesize
\centering {
\begin{tabularx}{16cm}{@{}X X c@{}}
\toprule
\textbf{Sentence A} & \textbf{Sentence B} & \textbf{Target} \\
\midrule\midrule
``A man is singing a song and playing the guitar'' & ``A man is opening a package that contains headphones'' & 1.6\\[0.4cm]
``Two dogs are playing by a tree'' & ``Two dogs are playing by a plant'' & 4.6\\[0.2cm]
``A woman is riding a horse'' & ``A man is opening a small package that contains headphones'' & 1.0\\[0.4cm]
``A potato is being sliced by a woman'' & ``A woman is slicing a carrot'' & 3.0\\[0.2cm]
``A man is screaming'' & ``A man is scared'' & 3.6\\[0.2cm]
``Men are sawing logs'' & ``Men are cutting wood'' & 4.5\\[0.2cm]
\bottomrule
\end{tabularx}}
\caption{\labtab{senteval:examples:sick} SICK-R is a Semantic Textual Similarity (STS) task for which labels are scores between 1 and 5. A score of 5 means that the two sentences are completely equivalent and 1 that the two sentences are completely dissimilar. In between, their degree of equivalence differs given the proportion of shared topics and information between the two sentences.}
\end{table*}

\paragraph{Training configuration}

% \bcomment{with more details on the task, the following parag would be easier to understand}{}
We use a similar training procedure
%\footnote{Hyperparameters such as hidden size, optimization procedure such or learning rate are fixed as proposed in \citet{tai_15} and are detailed in Appendix.} 
as in \textcite{tai_15}. We transform the target $y$ from the SICK-R task into the distribution $p$ defined by:

\begin{equation*}
p_{i} = \left\{
\begin{array}{ll}
y - \lfloor{y}\rfloor, & i = \lfloor{y}\rfloor + 1 \\
\lfloor{y}\rfloor - y + 1, & i = \lfloor{y}\rfloor \\
0 & \text{otherwise}
\end{array} \right.
\end{equation*}
We use a dedicated architecture to predict the similarity distribution from a pair of sentences. The similarity module takes as input a pair of sentence vectors $h_{L} $ and $h_{L}$ and computes their component\-wise product $h_{L} \odot h_{R}$ and their absolute difference $|h_{L} - h_{R}|$. Given these features, we compute the probability distribution  $\hat{p}_{\theta}$ using a two-layer perceptron network (MLP):
\begin{align}
\begin{split}
&h_{\times}=h_L\odot h_R, ~~~~~h_{+} = |h_L - h_R |, \\
&h_s = \sigma (W^{(\times)}h_{\times} + W^{(+)}h_{+} + b^{(h)}), \\
&\hat{p}_{\theta} = \softmax(W^{(p)}h_s + b^{(p)}),\\
\end{split}
\end{align}
Where $\sigma$ is the sigmoid function. We use the KL-divergence between the prediction $\hat{p}_{\theta}$ and the ground truth $p$ as a/ training objective:
\begin{equation}
J(\theta) = \frac{1}{N}\sum_{k=1}^{N}\text{KL}(p^{(k)} || \hat{p}_{\theta}^{(k)}) + \lambda||\theta||_{2}^{2}
\end{equation}
Finally during inference, the similarity score $\hat{y}$ is computed as $\hat{y} = r^{\intercal} \hat{p}_{\theta}$ with $r^{\intercal} = [1, \dots, 5]$. 

% We use a standard training configuration, inspired from \citet{tai_15} and combine the embeddings obtained with our model for each sentence with a similarity module\footnote{The setup is detailed in Appendix \ref{sec:sick-train}.}.

\paragraph{Hyper-parameters} We set the hyperparameters in accordance with the choices made in \textcite{tai_15}, such that we can directly compare our results in Table~\ref{table:supervised}. For all experiments detailed in the current section, the batch size is fixed to 25, weight decay to $1e^{-4}$ and gradient clipping set to $5.0$. The learning rate is set to $0.025$ for the \textsc{TreeLSTM} parameters. When using a pre-training procedure for the parser, we set the learning rate to $5e^{-3}$ and use the following warm-up: for the first epoch, the parser is frozen. For the following epochs, all parameters are trained. At each epoch, the parser learning rate is divided by a factor of two. Without pre-training, the learning rate is set to $5e^{-4}$ for the parser. All model weights are initialized with a Xavier distribution. The hidden size of the similarity architecture is set to 50. The \textsc{TreeLSTM} hidden size is set to 150. We use the Adagrad optimizer. We do not apply any dropout. We perform training for a maximum of 20 epochs and stop when no improvement was observed on the development set for 3 consecutive epochs.
%  and biases set to 0
Regarding the vocabulary, we limit the size to 20,000 words and initialize the embeddings layer with 300-dimensional GloVe embeddings.\sidenote{\url{https://nlp.stanford.edu/projects/glove/}} The embeddings are not updated during training.\sidenote{We trained all models on a single 1080 Ti Nvidia GPU. Training time for each epoch is approximately 1 minute. The model counts 13.7M parameters. Data can be downloaded using the SentEval package \url{https://github.com/facebookresearch/SentEval}.}

% We set the hyper-paremeter given literature on the domain, in particular regarding choices made in \citet{tai_15}. We detail the full setup in Appendix  \ref{sec:sick-train}.

% \subsection{Comparison with other encoders} 
% \label{sec:soa}

Table \ref{table:supervised} reports the results from the test set. As expected, structured models perform better than models using weaker underlying structures. We also observe that our model is competitive with a \textsc{Bert}-base upper-line. It is essential to note that \textsc{Bert} models are heavily pre-trained on vast corpora, whereas our structured models are trained only on the SICK-R and PTB data. 

% Moreover, our model achieves a significantly lower standard variation than {\sc bert}-base implementation.
\begin{table}[!htb]
\centering
\small
\begin{tabularx}{\textwidth}{@{}X | c@{} }
\toprule
\textbf{Encoder} & \textbf{$r$} \\
\midrule
\midrule 
\textsc{Bow}$^\dagger$ & 78,2 \scriptsize{(1,1)} \\
% \midrule 
\textsc{LSTM}$^\dagger$ & 84.6 \scriptsize{(0.4)}  \\
Bidirectional \textsc{LSTM}$^\dagger$ & 85.1 \scriptsize{(0.4)} \\
% \midrule
N-ary \textsc{TreeLSTM}$^\dagger$ \parencite{tai_15} & 85.3 \scriptsize{(0.7)} \\
Childsum \textsc{TreeLSTM}$^\dagger$ \parencite{tai_15} & 86.5 \scriptsize{(0.4)}  \\
% \midrule
\textsc{Bert}-base \parencite{devlin_19} & 87.3 \scriptsize{(0.9)} \\
% \bert{}-large & 89.4 \scriptsize{(0.6)} & 84.7 \scriptsize{(1.0)} & 21.5 \scriptsize{(1.3)} \\
\midrule
\multicolumn{2}{c}{\textit{Our model}}\\
\midrule
% Unified \treelstm{}$^\dagger$ (\textbf{our model)} & 87.0 \scriptsize{(0.3)} \\
Unified \textsc{TreeLSTM}$^\dagger$ & 87.0 \scriptsize{(0.3)} \\
% Unified \treelstm{}-100 & 86.5 \scriptsize{(0.2)} & 80.4 \scriptsize{(0.4)} & 25.9 \scriptsize{(0.5)} \\
\bottomrule
\end{tabularx}
\caption{Evaluation of the model on the SICK-R task: we pre-train our parsing module on the PTB and continue to update the full model on the SICK-R task. We compare with \textsc{Bert} and models relying on sequential and tree structures. We report Pearson correlation on the test set, by convention as $r \times 100$ (average and standard deviation from 5 runs). $\dagger$ indicates models that we trained.}
% \textbf{(1)} No underlying structure 
\label{table:supervised}
\end{table}

\subsection{Textual entailment}
\label{sec:ste}

We also test our model on the Stanford Natural Language Inference (SNLI) task \parencite{bowman_15}, which includes 570k pairs of sentences with the labels entailment, contradiction, and neutral. It is distributed in a 550k/10k/10k train/dev/test split. We already presented some examples from the SNLI task in \reftab{snli-examples}. We reproduce some other examples in \reftab{snli-examples-2}.
% \bcomment{self containment}{provide some more details. what are the inputs, the outputs ?}

\begin{table*}[!htb]
\centering
\footnotesize
\begin{tabularx}{16cm}{@{}X X c@{} }
  \toprule
Premise & Hypothesis & label \\
\midrule
\midrule 
Two women are embracing while holding to go packages. & Two woman are holding packages. & entailment\\
\rule{0pt}{3ex}Two men on bicycles competing in a race. & People are riding bikes. & entailment\\
\rule{0pt}{3ex}Two women having drinks and smoking cigarettes at the bar. & Three women are at a bar. & contradiction\\
\rule{0pt}{3ex}Two doctors perform surgery on patient. & Two doctors are performing surgery on a man. & neutral\\
\rule{0pt}{3ex}A man in a black shirt is playing golf outside. & A man plays on a golf course to relax. & neutral\\
\bottomrule
% From 1.0rc3
\end{tabularx}
\caption{\labtab{snli-examples-2} SNLI examples presented in the original paper \parencite{bowman_15} and extracted from the development section of the corpus.}
\end{table*}

\paragraph{Training configuration}

We use a similar training procedure as in \textcite{choi_18}. A dedicated architecture is used to predict the similarity distribution from a pair of sentences. The similarity module takes as input a pair of sentence vectors $h_{L} $ and $h_{L}$ and computes their component\-wise product $h_{L} \odot h_{R}$ and their absolute difference $|h_{L} - h_{R}|$. Given these features, we compute the probability distribution  $\hat{p}_{\theta}$ using a three-layer perceptron network (MLP):
\begin{align}
\begin{split}
&h_{\times}=h_L\odot h_R, ~~~~~h_{+} = |h_L - h_R |, \\
&h_s = \relu(W^{(1)}[h_{\times}, h_{+}, h_L, h_R] + b^{(1)}), \\
&h_s = \relu(W^{(2)}h_s + b^{(2)}), \\
&\hat{p}_{\theta} = \softmax(W^{(p)}h_s + b^{(p)}),\\
\end{split}
\end{align}
We use the cross entropy loss between the prediction $\hat{p}_{\theta}$ and the ground truth $p$ as a/ training objective:
\begin{equation}
J(\theta) = -\frac{1}{N}\sum_{k=1}^{N}p^{(k)} log \hat{p}_{\theta}^{(k)} + \lambda||\theta||_{2}^{2}
\end{equation}

% We use a standard training configuration, inspired from \citet{choi_18} for which a similarity module is used to combine the embeddings obtained with our model for each sentence\footnote{The setup is detailed in Appendix \ref{sec:snli-train}.}. 

% We choose a standard training configuration, inspired from \citet{bowman_15} for which a similarity modules is used to combine the embeddings obtained with our model for each sentence\footnote{The setup is detailed in Appendix \ref{sec:sick-train}.}.  We report the corresponding results from the test set in Table \ref{table:supervised}.

%We set the hyper-paremeter given literature on the domain, in particular regarding choices made in \citet{choi_18}. We detail the full setup in Appendix  \ref{sec:snli-train}.

\paragraph{Hyper-parameters} We set the hyperparameters in accordance with the choices made in \textcite{choi_18}, such that we can directly compare our results in Table~\ref{table:snli}. For all experiments detailed in Section \ref{sec:ste}, the batch size is fixed to 128, weight decay to $0$, and gradient clipping set to $5.0$. The learning rate is set to $1e^{-3}$ for the \textsc{TreeLSTM} and the parser. The hidden size of the similarity architecture is set to 1024. The \textsc{TreeLSTM} hidden size is set to 600. We use the Adam optimizer. We apply a 0.2 dropout within the similarity architecture. We perform training for a maximum of 20 epochs and stop when no improvement was observed on the development set for 3 consecutive epochs. Still following \textcite{choi_18}, we limit the size of vocabulary to 100,000 words and initialize the embeddings layer with 300-dimensional GloVe embeddings. The embeddings are not updated during training. % \cite{pennington_14}
% \bcomment{why a different setup than earlier ?}{} 

\begin{table}[!htb]
  \centering {
  \footnotesize
  \begin{tabularx}{\textwidth}{@{}X | c@{} }
    \toprule
    \textbf{Encoder} & \textbf{Test Acc.} \\
    \midrule
    \midrule 
    % {\sc LSTM} \cite{bowman_16} & 77.6 \\
    % \midrule
    \textsc{Spinn} \textbackslash w Reinforce \parencite{yogatama_17} & 80.5  \\
    CYK and \textsc{TreeLSTM} \parencite{maillard_19} & 81.6  \\
    \textsc{Spinn} \parencite{bowman_16} & 83.2 \\
    % \midrule
    % Syntactic TreeLSTM \cite{chen_17} & 93.5 & 88.6 \\
    % MT-DNN \cite{liu_19} & 97.2 & 91.6 \\
    ST-Gumbel \parencite{choi_18}  & 86.0 \\
    Structured Alignment \parencite{liu_18} & 86.3 \\
    \textsc{Bert}-base \parencite{zhang_20} & 90.7 \\
    % {\sc SemBert}-base \cite{zhang_20} & 91.0 \\
    \midrule
    \multicolumn{2}{c}{\textit{Our model}}\\
    \midrule
    % Unified \treelstm{} (\textbf{our model}) & 85.0 {\scriptsize (0.2)} \\
    Unified \textsc{TreeLSTM} & 85.0 {\scriptsize (0.2)} \\
    % Unified \treelstm{} \textbackslash w tracking \textsc{LSTM} (\textbf{ours}) & 83.0 {\scriptsize (0.8)} \\
    % Unified \treelstm{}-100 & 83.6 {\scriptsize (0.0)} \\
    \bottomrule
  \end{tabularx}}
  \caption{Evaluation of the model on the SNLI-R task: We pre-train our parsing module on the PTB and continue to update the full model on the SNLI task. We compare with \textsc{Bert} and latent tree learning models. We report the accuracy on the test set (average and standard deviation from 2 runs).}
  % Comparison with other models on the \textbf{SNLI} task. Results are grouped as follow: \textbf{(1)} Sequential structure \textbf{(2)} Model learning tree structures \textbf{(3)} Framework introducing a direct interaction between the premise and hypothesis using attention mechanism \textbf{(4)} Our model with a pre-training on the PTB. We report the average and the standard deviation from 2 runs.}
  \label{table:snli}
\end{table}

We report the results in Table \ref{table:snli}. Our results are close to \textcite{choi_18}, which also compute semantic representations along with discrete tree structures but relies on a distinct syntactic formalism. The performance gap can be attributed to the use of dependency instead of binary parsing. However, it is also important to note that we encode the leaf nodes using only static embeddings, while \textcite{choi_18} apply sequential LSTMs to the leaf nodes, resulting in a hybrid model with dual latent structures. The authors affirm that "the LSTM applied to leaf nodes has a substantial gain over the basic leaf [affine transformation]". Based on their results, this transformation of the leaf node induces an accuracy improvement of about 1.4 points. 
% \bcomment{conjecture what explains the differences ?}{}

In models from \textcite{liu_18} and \textcite{zhang_20} sentences are encoded with direct interaction using an attention mechanism. These architectures relying on cross sentences attention outperform those without.
% , but our work focuses on the underlying sentence structure
We hypothesize that, 
% for the semantic similarity task, the final prediction could be extrapolated from both sentence embeddings. In such a case, the model was competitive with {\sc Bert}. 
on this textual entailment task, the final prediction cannot be directly deduced from both sentence embeddings. In this case, \textsc{Bert} and the structured alignment model have a clear advantage since they encode interactions between both sentences.
% Further work could involve adding such interactions as exemplified in \citet{liu_18}.}
% In the two first sections of Table~\ref{table:snli}, there is no intermediate interaction between the encoding of each sentence representation. In the second part of the Table, sentences are encoded with direct interaction using an attention mechanism. Models using interaction outperform those without, but our work focuses on the underlying sentence structure.
% \textcolor{blue}{To compare our results, we use a procedure analogue as in \citet{choi_18} and \citet{bowman_16} and apply a tracking \textsc{LSTM} on token input embeddings to give contextual information to each leaf node. 


% Our comparison includes models that induce direct interactions between sentences using attention mechanism. Such setup drastically improves the test accuracy.

% However, our focus lays on the sentence underlying structure rather than in the interaction between the premise and the hypothesis. In this perspective, we consider evaluation methods, as ours, do not introduce intermediate interaction between each sentence hidden representations. In this comparison, our model achieve an accuracy comparable to the {\sc Spinn} model. 


%  We aim at quantifying the degree of supervision needed to achieve competitive performance on downstream tasks with consistent syntactic structures.

% \subsection{Comparison with linguistic parse}
% \label{subsec:parse}

% \begin{itemize}

%%%%%%%%%%%%%%%%%%%%%%%%%%%%%%%%%%%%%%%%%%%%%%%%%%%%%%%%%%%%%%%%%%%%%%%%%%%%%%%%%%%%%%%%
\section{Impact of the parser initialization}
\labsec{sec:parser-init}
%%%%%%%%%%%%%%%%%%%%%%%%%%%%%%%%%%%%%%%%%%%%%%%%%%%%%%%%%%%%%%%%%%%%%%%%%%%%%%%%%%%%%%%%

Our framework primarily aims to be a structured sentence encoder. Accordingly, we have demonstrated in the previous section that our architecture is competitive with comparable approaches and might even be competitive with \textsc{Bert}-based models. We are also interested in interpreting the structures the model actually learns and how such structures impact downstream performances.

In the previous experimental section, we pre-trained the parser on human-annotated data. %Indeed, we expect the optimal structure to exhibit intuitive patterns such as subject-verb or verb-object connection. 
However, the optimal structure of a sentence may not derive from linguistic insights. It may also depend on computational factors. For example, the length of the the computational path from the root to the final representation could be an important factor. Tree-structured neural networks compute the root at the very last step while in sequential \textsc{LSTM }, the computational path from the root to the final representation is longer. Finally, as explored in \refch{structure-scale} some structures may be better adapted for a given task. For example, tree-structure may be more adapted for sentiment analysis but not be the best structure for a keyword extraction task.
%differ from the task}. Moreover, for computational reasons, it might even differ significantly from linguistic insights. 

In this section we perform an ablation study to better understand how the initialization of the parser impacts the resulting structures (Section \ref{sec:parses-impact}) and the final downstream performances (Section \ref{sec:dowstream-impact}). We begin by defining the different initialization scenarios we considered (\refsec{tree:init:ptb}
and \refsec{tree:init:unsupervised}). In all scenarios, we either continue to update the parser when fine-tuning the model on downstream tasks or freeze the parser and only train the \textsc{TreeLSTM}. These two configurations are indicated with \textbf{$\checkmark$} and \textbf{$\times$} symbols respectively.
% 

\subsection{Adjusting the proportion of linguistic annotations} 
\labsec{tree:init:ptb}

Tree-structured models traditionally rely on linguistic structures obtained by parsers \parencite{tai_15}. Linguistic resources are available for languages such as English; it is technically possible to pre-train the parser. However, resources such as the PTB are not available in all languages. To better quantify the benefits of using linguistic annotations, we propose the following configurations, using various proportions of the PTB to initialize the parser:

\begin{itemize}
    \item In the \textbf{PTB-All} configuration, the parser is previously pre-trained on the PTB. This configuration is the same as in \refsec{sec:eval}.
    \item In the \textbf{PTB-$\emptyset$} configuration, the parser parameters are randomly initialized
    \item We also consider an initialization with only a small proportion of the \textbf{PTB} and train a parser by only using 100 randomly selected samples. This configuration is referred as \textbf{PTB-100}.
\end{itemize}

% Given the similarity of such attention matrices to the score matrices employed in arc-factored dependency parsing (Mc- Donald et al., 2005a,b), a salient question concern- ing interpretability becomes: Can we expect some combination of these parameters to capture linguis- tic structure in the form of a dependency tree, espe- cially if the model performs well on NLP tasks?

% With large-scale machine translation and language models being openly distributed for ex- perimentation, several researchers have wondered if self-attention is capable of representing syntactic structure, despite not being trained with any overt parsing objective.

%  Often, the line of in- quiry regarding interpretability in NLP has been concerned with extracting and analyzing linguistic information from neural network models of lan- guage (Belinkov and Glass, 2019). Recently, such investigations have targeted Transformer models (Hewitt and Manning, 2019; Rosa and Marecˇek, 2019; Tenney et al., 2019), at least in part because the self-attention mechanism employed by these models offers a possible window into their inner workings. With large-scale machine translation and language models being openly distributed for ex- perimentation, several researchers have wondered if self-attention is capable of representing syntactic structure, despite not being trained with any overt parsing objective.

% In pursuit of this question, Raganato et al. (2018) applied a maximum-spanning-tree algorithm over the attention weights of several trained MT models, comparing them with gold trees from Universal De- pendencies (Nivre et al., 2016, 2020). They found that, while the accuracy was not comparable to that of a supervised parser, it was nonetheless higher than several strong baselines, implying that some structure was consistently represented.

% These models often choose a Transformer architecture largely owing to its at- tractive scalability. Studies (Hewitt and Manning, 2019; Jawahar et al., 2019)  have shown that a pre- trained transformer is able to capture certain syn- tactic information implicitly by learning from suf- ficient examples. However, there is still a big gap between the syntactic structures implicitly learned and the golden syntax trees created by human ex- perts.

\subsection{Using unsupervised structures}
\labsec{tree:init:unsupervised}

%TODO \bcomment{transition à discuter}{} 

We are also interested in structures emerging from large pre-trained models. Such models present similarities with recent graph parser architecture. Although they are not trained with a direct parsing objective, many lines of work investigate if attention matrices can reflect syntactic structures \parencite{jawahar_19, clark_19, ravishankar_21} or, on the contrary, if it is efficient to integrate tree structural information within transformers \parencite{wang_19, bai_21}. 

As stated in the Introduction, our model is agnostic to any graph-based dependency parser. It is therefore possible to use any model or heuristic to infer sentence structure. 
%We observed in the previous Section that parser initialized on only a sub-sample from the PTB led to intelligible parses. This section analyzes to which extent it is possible to initialize the parser without linguistic insights.
In particular, we can use a pre-trained model such as \textsc{Bert} to infer structures based on the internal representations it learns. We do not intend to provide an in-depth analysis of how \bert could be used for unsupervised parsing. Therefore, we do not extensively explore how \bert could accommodate parsing tasks. However, we instead propose a proof-of-concept that our model can accommodate a large variety of graph-based parsers and show it is indeed possible to use \bert as an unsupervised parser in our case.

%In this setup, we only use \bert{} to output a tree structure for the sentence. Words are then composed using our previously defined \treelstm{}. 
\textsc{Bert} relies upon the self-attention mechanism. Inside each layer, tokens are computed as a weighted combination of each other. For each token $x$, a query and key vector are computed using a linear transformation detailed in Eq~\ref{eq:qkv}. Given these vector tuples, the attention weights $s$ are computed following Eq~\ref{eq:bert-attention} in which $N$ refers to the dimension of the query and key vectors.

\begin{align}
    %h^0_j &= W_eu_j + W_p \label{eq:first-layer}\\
    q_j, k_j &= W^{(q, k)} x_j + b^{(q, k)} \label{eq:qkv}\\
    %s_{kj} &= \mathsf{softmax}\left(\frac{k_{k} \cdot q_{j}}{\sqrt{N}}\right) \cdot v_k \label{eq:bert-attention}
    s_{kj} &= \softmax\left(\frac{k_{k} \cdot q_{j}}{\sqrt{N}}\right) \label{eq:bert-attention}
    %h^n_j &= \sum_{k=1}^{N} \alpha_{kj} h^{n-1}_k \label{eq:bert-weighted}
\end{align}
% \quad \forall n \in [1, L]

We induce a tree structure following a procedure close to the one used in \textcite{ravishankar_21}.\sidenote{\textcite{ravishankar_21} decode dependency trees from attention matrices using the Chu-LiuEdmonds maximum spanning tree algorithm \parencite{edmonds_67} and compare them with gold treebank using the Undirected Unlabeled Attachment Score (UUAS)—the percentage of undirected edges recovered correctly.} The method interprets the combination weights $s$ as a weighted graph whose nodes are tokens. We then apply Eq~\ref{eq:biaffine-3} to induce a maximum spanning tree from the attention matrix as detailed in \refsec{sec:model}. We make use of the last layer and induce a tree from the first attention head.\sidenote{As mentionned earlier, we only aim at proposing a proof-of-concept here. Therefore, we do not test all possible heads and layers to induce trees.} Given the tree structure induced from \textsc{Bert}, we apply our \textsc{TreeLSTM} model detailed in Eq. \ref{eq:treelstm-weighted} to \ref{eq:treelstm-last}. We stress the fact that in this configuration, we only use \textsc{Bert} as an unsupervised parser to infer a sentence structure. The semantic composition along with the structure to produce a sentence embedding is solely computed by the weighted \textsc{TreeLSTM}.
% \bcomment{you do not state how you combine the weights of the heads}{}

% \section{Experiments in low supervised setups}
\subsection{Impact of the initialization on parses}
\label{sec:parses-impact}


% However, we not interested solely in tackling downstream tasks, but also to produce consistent syntactic structures. Therefore, we focus in the next Section on the tree produced by our method.

%TODO \bcomment{structural issue}{introduce PTB 100 etc here}

In this section, we analyze to which extent the structures generated by our model are comparable with meaningful linguistic annotations. We compare the parses generated by two distinct models differing by their initialization on the development set of both tasks. Our reference is the silver parses from the PTB-All configuration, where the parser is previously pre-trained on the full PTB and not updated during training.   

%We compare the learned parse with silver references to investigate the linguistic consistency of the learned structures. To this end, we contrast distinct configurations for which we initialize our parser with various degrees of supervision. 

In Table~\ref{table:parse}, we measure the Unlabeled Attachment Score (UAS) between the two parsers, that is, the ratio from the number of common arcs between two parses by the total number of arcs. 

\begin{table}[!htb]
\centering
\small
\begin{tabularx}{\textwidth}{@{}c c | Y Y@{}}
\toprule
\textbf{Parser 1} & \textbf{Parser 2} & \textbf{SICK-R (dev UAS)} & \textbf{SNLI (dev UAS)}\\
\midrule\midrule
\multicolumn{4}{c}{\textit{Impact of parser fine-tuning}}\\
\midrule
PTB-100 (\checkmark) & PTB-100 ($\times$) & 85.2 {\scriptsize (1.5)} & 5.6 {\scriptsize (1.9)}\\
PTB-All (\checkmark) & PTB-All ($\times$) & 98.4 {\scriptsize (0.1)} & 11.7 {\scriptsize (0.9)}\\
\midrule
\multicolumn{4}{c}{\textit{Impact of the PTB sample size}}\\
\midrule
PTB-100 (\checkmark) & PTB-$\emptyset$ (\checkmark) & 6.3 {\scriptsize (0.0)} & 10.1 {\scriptsize (10.7)}\\
PTB-All (\checkmark) & PTB-$\emptyset$ (\checkmark) & 10.1 {\scriptsize (0.0)} & 15.1 {\scriptsize (15.4)}\\
PTB-All (\checkmark) & PTB-100 (\checkmark) & 76.9 {\scriptsize (0.7)} & 0.3 {\scriptsize (0.2)}\\
\midrule
\multicolumn{4}{c}{\textit{Unsupervised parser}}\\
\midrule
% avec taille de batch 16
% \bert{} ($\times$) & PTB-All ($\times$) & --- & 13.0 {\scriptsize (4.9)}\\
% \bert{} ($\checkmark$) & PTB-All ($\times$) & --- & 15.0 {\scriptsize (1.7)}\\
% avec taille de batch 128
\textsc{Bert} ($\times$) & PTB-All ($\times$) & --- & 13.0 {\scriptsize (4.9)}\\
\textsc{Bert} ($\checkmark$) & PTB-All ($\times$) & --- & 13.7 {\scriptsize (2.7)}\\
\bottomrule
\end{tabularx}
\caption{Impact of the parser initialization on parses: we compare the parses from the SICK-R and SNLI development sets using different parser initializations. We obtained the PTB parses with the graph parser initialized on a given proportion of the PTB (\refsec{sec:model}). Regarding \textsc{Bert} , we inferred the structures from the pattern learn by the pre-trained model (\refsec{sec:parser-init}). We either continue to update the parser (\textbf{$\checkmark$}) when fine-tuning the model on downstream tasks or freeze the parser (\textbf{$\times$}) and only train the \textsc{TreeLSTM}. UAS corresponds to the mean pairwise comparison of two configurations between two runs (standard deviation in parentheses).}
\label{table:parse}
\end{table}

We observe distinct behaviors given both tasks. We believe this effect is due to the differences between training configurations—detailed in Section~\ref{sec:ste} and \ref{sec:sts}. In particular, we use the Adagrad optimizer for the SICK-R task and Adam for the SNLI task.

For the SICK-R task, the UAS between PTB-$\emptyset$ and PTB-All are very low. This reveals that the parses obtained with only downstream task supervision overlap very little with with gold linguistic parses. In this regard, we share the observation from \textcite{williams_18} that latent trees obtained from sole downstream supervision are not meaningful in syntax. However, PTB-All and PTB-100 are remarkably close; only a few PTB samples are needed to obtain intelligible linguistic parses with our setup. Regarding the PTB-100 configuration, we note an evolution of the parses when fine-tuning on the downstream task. We hypothesize that the model can adapt itself to the dataset's specificity. 

Regarding the SNLI task, fine-tuning the parser deeply impacts the shape of the parses. Depending from the initialization, parses will converge to distinct structures. Indeed, the UAS between all configurations is very low. Moreover, we observe that when using a random initialization (PTB-$\emptyset$), the standard deviation between the UAS from various runs is very high. This reveals that without fixed initialization, the parses tend to show some instability.

% Bert
For the initialization with an unsupervised structure, we only evaluate our setup on the SNLI task, which has more training samples. We compare the structures obtained with \textsc{Bert} with the silver trees from the PTB-All-$\times$ configuration. We present the mean UAS over the trees obtained for all attention heads. The standard deviation is relatively high, pointing underlying structures differ given the attention head. Nonetheless, self-supervised structures do not align well with linguistic insights. When updating \textsc{Bert} together with the \textsc{TreeLSTM}, the UAS increases while the standard deviation decreases. As \textsc{Bert} is fine-tuned, structures tend to become more standard and present slightly more similarities with linguistic patterns.

% We observe the downstream task performances as well as the underlying structures differs given the attention head. We observed human linguistic annotations can be a good initialization setup for our model since they lead to an improvement on downstream results. However, depending on the training configuration, the initial structure may deeply change during training. Therefore, we explore if other initialization methods  could be good prior for tree-structured model.

\paragraph{Visualization of the parses} We illustrate the effect summarized in Table~\ref{table:parse} on some chosen examples. Figures from the first column (\ref{subfig:expl1-1}, \ref{subfig:expl1-3} and \ref{subfig:expl1-5}) show the parses obtained without updating the parser component on the downstream task. Figures from the second column (\ref{subfig:expl1-2}, \ref{subfig:expl1-4} and \ref{subfig:expl1-6}) show the evolution of the parses for the same initialization but after fine-tuning the parser on the SNLI task. Figures from the first raw (\ref{subfig:expl1-1} and \ref{subfig:expl1-2}) are initialized using the full PTB, the second raw (\ref{subfig:expl1-3} and \ref{subfig:expl1-4}) is initialized using 100 PTB samples while the one from the last raw (\ref{subfig:expl1-5} and \ref{subfig:expl1-6}) are initialized using unsupervised patterns.

\begin{figure}[htb!]
    \centering
    \begin{subfigure}[b]{0.475\textwidth}
        \centering
        \includegraphics[width=\columnwidth]{images/all_freeze_9533.png}
        \caption{Parse obtained using the the PTB-All ($\times$) configuration.}
        \label{subfig:expl1-1}
    \end{subfigure}
    \hfill
    \begin{subfigure}[b]{0.475\textwidth}  
        \centering 
        \includegraphics[width=\columnwidth]{images/all_update_9533.png}
        \caption{Parse obtained using the the PTB-All ($\checkmark$) configuration.}
        \label{subfig:expl1-2}
    \end{subfigure}
    \vskip\baselineskip
    \begin{subfigure}[b]{0.475\textwidth}   
        \centering 
        \includegraphics[width=\columnwidth]{images/100_freeze_9533.png}
        \caption{Parse obtained using the the PTB-100 ($\times$) configuration.}
        \label{subfig:expl1-3}
    \end{subfigure}
    \hfill
    \begin{subfigure}[b]{0.475\textwidth}
        \centering 
        \includegraphics[width=\columnwidth]{images/100_update_9533.png}
        \caption{Parse obtained using the the PTB-100 ($\checkmark$) configuration.}
        \label{subfig:expl1-4}
    \end{subfigure}
    \vskip\baselineskip
    \begin{subfigure}[b]{0.475\textwidth}
        \centering
        \includegraphics[width=\columnwidth]{images/bert_att_0_9533.png}
        \caption{Parse obtained using the attention head \#$1$ and without updating \textsc{Bert}.}
        \label{subfig:expl1-5}
    \end{subfigure}
    \hfill
    \begin{subfigure}[b]{0.475\textwidth}  
        \centering 
        \includegraphics[width=\columnwidth]{images/bert_att_0_9533_update.png}
        \caption{Parse obtained using the attention head \#$1$ and updating \textsc{Bert}.}
        \label{subfig:expl1-6}
    \end{subfigure}
    \caption{Example of parse obtained using various configurations from our model. The parser component is initialized on PTB-All (\ref{subfig:expl1-1}, \ref{subfig:expl1-2}), PTB-100 (\ref{subfig:expl1-3}, \ref{subfig:expl1-4}) or \textsc{Bert} (\ref{subfig:expl1-5}, \ref{subfig:expl1-6}). We either freeze ($\times$) or update ($\checkmark$) the parser during the fine tuning on the SNLI. We include the weights $\alpha$ produced from the parser. 
    %We use \bert{} to obtain tree structures and evaluate the full pipeline on the \textbf{SICK-R} and \textbf{SNLI} tasks. 
    We report the accuracy from a single run on the test set.} 
    \label{fig:parse-expl}
\end{figure}

As a result of the fine-tuning, we observe that trees evolve into trivial structures and tend to connect every node to an arbitrary root. We postulate that such trivial structures present advantages from a computational standpoint. \textcite{shi_18} also observe that trivial trees without syntax yield better results than syntax and latent trees. They postulate that balanced binary trees benefit from two advantages. First, balanced trees treat all leaf nodes equally, making it easier to select essential information from all words within a sentence automatically. Second, balanced trees have shorter tree depths, which induces a shorter path for propagating information from leaves to roots, thereby reducing propagation errors.

For \textsc{Bert} parser initialization, we observe the fine-tuning produces rather sequential patterns, with words connected to direct neighbors. Some isolated groups of words also present inner connections.

%In our full supervised setup, we observe the parser's fine-tuning enables the model to adjust the parse from Figure~\ref{subfig:expl1-1} into the structure represented in Figure~\ref{subfig:expl1-2}. The obtained structure does match the one produced using more pre-training samples from the PTB dataset.

%In our low supervised setup, fine-tuning the parser 
%Regarding the parses obtained without any pre-training on the PTB (Figure~\ref{subfig:expl1-14), we observe all nodes tend to be connected to an arbitrarily chosen root.


%%%%%%%%%%%%%%%%%%%%%%%%%%%%%%%%%%%%%%%%%%%%%%%%%%%%%%%%%%%%%%%%%%%%%%%%%%%%%%%%%%%%%%%%
% 2. Shi
%%%%%%%%%%%%%%%%%%%%%%%%%%%%%%%%%%%%%%%%%%%%%%%%%%%%%%%%%%%%%%%%%%%%%%%%%%%%%%%%%%%%%%%%

% \textcolor{blue}{\citet{williams_18} investigate the trees produced by \citet{yogatama_17} and \citet{choi_18}. The authors show neither method produce meaningful latent trees in syntax. Moreover, Gumbel softmax outputs inconsistent latent trees across initializations while reinforcement learning outputs trivial left-branching trees. We also observe that random initialization leads to trivial trees (PTB-$\emptyset$). The PTB-100 appears as good trade off. Indeed the induced trees are close to silver references and thus meaningful regarding dependency structure. The trees are also specific to the task due to the parser fine-tuning with downstream objective.}

\subsection{Impact of the initialization on downstream tasks} % Reducing the need for annotation
\label{sec:dowstream-impact}

We observed in previous Section~\ref{sec:parses-impact} that the initialization and the training configuration of the parser component deeply impact the resulting parses. We now study the impact of the parser initialization on downstream performances.

%\paragraph{Initialization with human annotated data} % Reducing the need for annotation
% First, we consider the configurations introduced in Section~\ref{sec:parses-analysis} and study the impact of the parser initialization on downstream tasks. 

\begin{table}[!htb]
\centering
\small
\begin{tabularx}{\textwidth}{@{}Y Y | Y Y@{} }
\toprule
\textbf{PTB sample size} & \textbf{Parser fine-tuning} & \textbf{SICK-R ($r$)} & \textbf{SNLI (Acc.)} \\
\midrule
\midrule 
\multicolumn{4}{c}{\textit{Linguistic annotations}}\\
\midrule
% PTB-$\emptyset$ & $\checkmark$ & 85.6 \scriptsize{(0.6)} & 81.5 \scriptsize{(0.2)}\\
% \midrule 
% PTB-100 & $\times$  & 86.6 \scriptsize{(0.2)} & 81.9 {\scriptsize (0.3)}\\
% PTB-100 & $\checkmark$ & 86.9 \scriptsize{(0.4)} & 82.6 {\scriptsize (0.2)}\\
% \midrule
% PTB-All & $\times$  & 87.2 \scriptsize{(0.2)} & 81.9 {\scriptsize (0.6)}\\
% PTB-All & $\checkmark$ & 87.5 \scriptsize{(0.4)} & 83.0 {\scriptsize (0.2)}\\
PTB-$\emptyset$ & $\checkmark$ & 85.6 \scriptsize{(85.6)} & 84.6 \scriptsize{(85.5)}\\
\midrule 
PTB-100 & $\times$  & 86.4 \scriptsize{(86.6)} & 84.5 {\scriptsize (85.5)}\\
PTB-100 & $\checkmark$ & 86.5 \scriptsize{(86.9)} & 84.9 {\scriptsize (85.8)}\\
\midrule
PTB-All & $\times$  & 86.8 \scriptsize{(87.2)} & 85.0 {\scriptsize (85.8)}\\
PTB-All & $\checkmark$ & 87.0 \scriptsize{(87.5)} & 85.0 {\scriptsize (85.5)}\\
\midrule 
\multicolumn{4}{c}{\textit{Unsupervised parser}}\\
\midrule
% avec la taille de batch de 16
% \bert{} & $\times$  & --- & 84.4 {\scriptsize (85.3)}\\
% \bert{} & $\checkmark$ & --- & 84.1 {\scriptsize (84.5)}\\
% avec la taille de batch de 128
\textsc{Bert} & $\times$  & --- & 84.4 {\scriptsize (85.3)}\\
\textsc{Bert} & $\checkmark$ & --- & 84.6 {\scriptsize (85.1)}\\
\bottomrule
\end{tabularx}
\caption{Impact of the parser initialization on downstream task performance:  We pre-train the parser module with a given sample size from the PTB. We either freeze (\textbf{$\times$}) or update (\textbf{$\checkmark$}) the parser during the fine-tuning. We report the average score test set from 5 runs for SICK-R and 2 runs for SNLI (the score from the development set are in parentheses). We report Pearson correlation by convention as $r \times 100$.}
\label{table:parser}
\end{table}

In Table~\ref{table:parser}, we compare the impact of the different initializations for both tasks. For each setup, we report the Pearson correlation on the test set of the SICK-R task and the accuracy on the test set from the SNLI task.

We either freeze the parser component or continue to update it, given the downstream loss for each initialization. Fine-tuning the parser on the task generally leads to an improvement in the downstream results. In that regard, we share the observation from other latent tree learning methods \parencite{maillard_19, choi_18}; models jointly learning the parsing and composition function outperform those with a fixed structure. 

We also observe that models using the full or partial annotated data outperform models relying on the sole downstream supervision (PTB-$\emptyset$). This observation is more clear on the SICK-R task. We previously observed that fine-tuning the parser can lead to tree structure diverging from linguistic patterns. Nonetheless, human annotations appear to be a good initialization for our model regarding the downstream performances. 

% For technical reasons, we had to limit the batch size to 16 when fine-tuning \textsc{Bert} parsing module. Therefore, the benefit from fine-tuning the parser when using \textsc{Bert} is difficult to interpret. However, 
We can observe that models relying on linguistic-driven structures achieve better performances. Nonetheless, the difference is thin, and we present an average score across trees obtained from all attention heads. Therefore some attention heads might present structures as efficient as linguistic patterns.

\section{Conclusion and future work}

We evaluate our model on textual entailment and semantic similarity tasks. Regarding the textual similarity task, we show that our setup is competitive with \textsc{Bert} base, although the latest is trained on datasets many orders of magnitude larger. We explore to which extent the trees produced by our model compare with linguistic structures and how this initialization impacts downstream performances. We empirically observe that downstream supervision troubles producing stable parses and preserving linguistically relevant structures.  % We corroborate that the sole use of downstream supervision is insufficient to produce parses that are easy to interpret. 
To encourage convergence towards readable linguistic structures, we examine a number of initialization setups. Depending on the optimization setup, the parse tree may present instability. We also observe that our structures often converge toward trivial branching patterns, which have little in common with gold linguistic parses. However, with respect to the downstream performances, linguistic insights appear to be a relevant initialization.
% \setchapterpreamble[u]{\margintoc}
\chapter{Studying shallow structure in transformer models}

Recent transformer architectures have gained increased popularity within the community. Contrary to tree-based models, they do not need carefully hand-annotated data to be trained. On the other hand, as many results suggest, these new models acquire some sort of tree structure. Transformers update each token hidden simultaneously through a fixed number of layers. Yet the role of these layers and how they process information is not fully understood. I formulate the hypothesis that the distinct layers do not encode specific surface, syntactic nor semantic functions but rather that such information emerges through the iterative application of layers. To better study the transformation of token representations across layers, I proposed a variant of ALBERT [2]. This model implements the key specificity of weights tying across layers, but also dynamically adapts the number of layers applied to each token. I analyze token transformation across the network depth. In particular, I study how iterations are distributed given the token dependency types. I showed that tokens do not require the same amount of iterations and that difficult or crucial tokens for the task are subject to more iterations.

\section{Model architecture}

\section{Analysis of the pre-training}

\section{Application on downstream tasks}

% \setchapterstyle{kao}
\setchapterpreamble[u]{\margintoc}
\chapter{Evaluating sentence embeddings}
\labch{evaluating}

\cleanchapterquote{When people say AI has “learned x” what they usually mean is that a deep learning model has learned a dataset well enough to find the pattern you asked for. It has no symbolic or logical abstraction. It is kahnemann system 1. It looks smart. It isn’t.}{Mark Madsen}{Data scientist at Teradata}


\section{Architectures with explicit phrasal composition}

Paraphrase, entailment, sentiment analysis

\section{Analyzing compositionality}

Semantic textual similarity

\section{Compositional properties}

Comparison with formal and distributional semantic representations. Very efficient. Yet not like humans.


% \pagelayout{wide} % No margins
% \addpart{Training neural models at scale}
% \pagelayout{margin} % Restore margins

% \setchapterstyle{kao}
\setchapterpreamble[u]{\margintoc}
\chapter{Embedding sentences with large transformer models}
%Training sentence embedding models using discriminative objective}
\labch{1B}

\cleanchapterquote{Language is the most interesting manifestation of intelligence. Visual comprehension is something that many animals also have. In some cases, it is even better than that of humans. Chimpanzees also understand feelings and social contexts. But no other living being has such a complex language as we do. And language links all other manifestations of intelligence, because I can talk about what I see, feel and think and how I act.}{Richard Socher}{Interview for \textit{die Zeit}, 2019}
% Artificial Intelligence Doesn't Make After Work Plans
% https://www.zeit.de/digital/2019-05/computational-linguistics-artificial-intelligence-speech-processing-richard-socher?utm_referrer=https%3A%2F%2Ft.co%2F

% \bcomment{you might ask a more incisive question in the first place,suggesstion:}{\ldots }

Bigger is better? At first sight it seems that current Natural Language Processing is consistently evolving towards larger and larger models paying less and less attention to the models at hand. In this section, we explore how we can leverage the performance of large sentence encoders by adapting their pre-training and increasing their size.

The previous sections discussed the importance of neural model structure in composing sentence representations. However, NLP trends do not primarily focus on these types of models, instead focusing on transformers, which are easier to scale. As a result, previous years have seen a general increase in the size of the models and a corresponding improvement over downstream performance. These improvements did not directly benefit sentence embeddings, as many transformer encoders perform below state-of-the-art on standard benchmarks. 
% In this section. we review the challenges and benefits of large models.

This section describes the development of state-of-the-art sentence embedding models as part of the project \textit{Train the Best Sentence Embedding Model Ever with 1B Training Pairs}.\footnote{\url{https://discuss.huggingface.co/t/train-the-best-sentence-embedding-model-ever-with-1b-training-pairs/7354}} This project took place during the \textit{Community week using JAX/Flax for NLP \& CV} organized by Hugging Face.\footnote{\url{https://discuss.huggingface.co/t/open-to-the-community-community-week-using-jax-flax-for-nlp-cv/7104}} Our project was among the competition winners and received an honorable mention. As part of this project, I contributed actively to the construction of the dataset as well as the training and documentation of the sentence embedding models.

We organize the section as follows: we first review the related work and the benefit of scaling in the specific case of sentence embeddings (\refsec{scale:introduction}). \refsec{scale:method} then proposes a self-supervised pre-training approach to learn sentence encoders. The approach addresses engineering challenges such as data collection, framework choice, and training hyper-parameters. Finally, we evaluate the benefit of our approach in \refsec{scale:experiments}.

\section{Transformers and scale}
\labsec{scale:introduction}

Pre-trained transformers resulted in a strong improvement over standard NLP benchmarks. The \textsc{Bert} model indeed claimed a 7.6\% absolute improvement on the popular GLUE benchmark, 5.6\% absolute accuracy improvement on the MultiNLI, and 1.5 F1 points on the SQuAD v1.1 question answering test. \textsc{Bert} introduced many increments to improve NLP tasks, including a new neural architecture, training paradigm number of parameters, and hyper-parameters setup. It is difficult to disentangle the contributions of all these factors, but the number of parameters is one of them. For example, the base version of \textsc{Bert} with 100M parameters achieves an average score of 79.6 on GLUE, while the large version with 340M parameters achieves 82.1. Apart from the number of parameters, the architecture, training procedure, and training data remain unchanged.

Compared with tree-structured encoders, transformers encode sentences without making substantial structure premises. Compared with sequential encoders, they compute each token state simultaneously using the attention mechanism, which is easy to parallelize across computing units. From a computing perspective, transformers are easier to scale. Consequently, the last few years have seen a race to increase the number of layers, parameters, hidden size, or pre-training data size. The model \textsc{Bert} exists in a base and large versions, which only differ by their hidden size and number of parameters. The same is true for the model GPT, which was incremented into GPT-2 and 3. While the second and third versions have much more parameters, the architecture is similar between all versions. As illustrated in \reffig{large-models}, the number of parameters for large language models follows what we may compare with Moore's Law.

% As opposed to deriving complex architectures from linguistic insights. it might be easier to focus on efficient and parallelizable networks. which are easy to scale. 

% Language models also seem to support the adage. "\bcomment{the bigger. the better}{bigger is better}".  The training at scale seems also to leverage some particular behaviors. mediated by some threshold effects. For example \textcite{brown_20} compare generative pre-trained models with distinct sizes on few-shot learning settings. While the smallest model with 1.3B parameters do not show any abilities on the task. the largest model with 175B parameters show surprisingly good generalization performances with a very limited number of training examples.

\begin{figure}[htb!]
	\includegraphics[width=\textwidth]{images/model-size-graph.jpeg}
	\caption[Large models number of parameters]{Evolution of the number of parameters for large language models. The figure is extracted from Microsoft \href{https://www.microsoft.com/en-us/research/blog/using-deepspeed-and-megatron-to-train-megatron-turing-nlg-530b-the-worlds-largest-and-most-powerful-generative-language-model/}{blog post}.}
	\labfig{large-models}
\end{figure}

This analysis also applies in some respects to sentence embeddings. As empirically observed by \textcite{conneau_17}, the embedding size is a key factor in downstream performance over the SentEval benchmark. We reproduce the figure from \textcite{conneau_17} in \reffig{scale:embedding-size}. For almost all encoders shown in the figure, performance increases proportionally to the size of the embeddings. However, regarding specifically \textsc{Bert}, the comparison does not directly extend to the embedding of sentences. Indeed, as already reported in \reftab{contrastive-soa}. \textsc{Bert} performance on the SentEval benchmark is, on average, 3 points below current state-of-the-art methods, including \textcite{simoulin_2021a}.

This lack of performance does not seem to be specifically related to the architecture of transformers. Indeed, \textcite{reimers_19} propose state-of-the-art sentence embeddings by successfully adapting the protocol from \textcite{conneau_17} to transformers. The approach successfully proposes to further fine-tune a pre-trained transformer on natural language inference data.

\begin{figure}[htb!]
	\includegraphics[width=\textwidth]{images/snli_em_size.png}
	\caption{Average performance with respect to the embedding size on the SentEval benchmark. The figure is extracted from \textcite{conneau_17}.}
	\labfig{scale:embedding-size}
\end{figure}

% Here. we study to which extend we can increase sentence encoder performances by scaling their size. As already observed in \refch{arithmetics} or \refsec{survey:downstream}. scaling models is an established method for improving downstream results. However. 

% \bcomment{The previous sections discussed the importance of neural model structure in composing sentence representations. Along with the architecture of the sentence encoder. other parameters may strongly influence the quality of the embeddings. In particular. the last few years have seen a race to increase the number of parameters. hidden size. or size of pre-training data. As already observed in \refch{arithmetics} or \refsec{survey:downstream}. scaling models is an established method for improving downstream results\sidenote{The effect of scaling is considered so obvious that it was used as the primary reason to reject RoBERTa  paper \parencite{liu_2019} from ICLR 2020. As expressed by the Program Chairs: "most of the findings are obvious (careful tuning helps. more data helps)". \url{https://openreview.net/forum?id=SyxS0T4tvS}}.}{manque de problématique; ne pas mettre cette note en évidence}


% \section{Train a sentence embedding model with 1 billion training pairs}
% \labsec{1B}

\section{Method}
\labsec{scale:method}

Even though the effect of scaling is no longer a surprise, training large models continues to be a challenging exercise. Training large models poses engineering challenges for optimization \parencite{you_20}, infrastructure \parencite{shoeybi_19, narayanan_21} and data collection \parencite{OrtizSuarezSagotRomary2019}.

\subsection{Training objective}
\labsec{1B:objective}

As in \refch{structure-scale}, we use a contrastive objective to train our sentence encoders. We collect sentence pairs $(a_i, p_i)$ that are somehow semantically related. The effective construction of the dataset is detailed in \refsec{1B:dataset}. We train the model to map pairs $(a_i, p_i)$ to close vectors while assigning unmatched pairs $(a_i, p_j)_{i \neq j}$ to distant vectors in the embedding space. This training method closely relates Quickthought (presented in \refsec{training}), contrastive unsupervised representation learning \parencite{saunshi_19}, training with in-batch negatives \parencite{carlsson_21}, InfoNCE \parencite{oord_18} or NTXentLoss \parencite{sohn_16}.

We illustrate the training objective in \reffig{contrastive}. Intuitively, the model should assign high similarity to the sentences «~How many people live in Berlin?~» and «~Around 3.5 million people live in Berlin~» and low similarity to other negative answers such as «~The capital of France is Paris~».

\begin{figure}[htb!]
	\includegraphics[width=10cm]{images/contrastive_1.png}
	\caption[Contrastive learning]{Illustration of the contrastive learning setup. The model is trained to associate an anchor sentence with another one that is semantically related. The notion of semantic relation depends on the nature of the pair. Our example aims to link the correct answer to a given question. City capitals are the subject of all these sentences, but only one is the correct answer.}
	\labfig{contrastive}
\end{figure}

As in other contrastive methods detailed in \refsec{training:self-supervised}, we build negative pairs by considering other samples from the batch. Given a batch of $n$ training samples, the model optimizes the following loss function:

\begin{equation}
    \mathcal{L} = -\frac{1}{n}\sum_{i=1}^n\frac{e^{c(a_i. p_i)}}{\sum_j e^{c(a_i. p_j)}}    
\end{equation}

Where $c$ is a \textit{critic function}, which measures the distance between two sentence embeddings $(a, p)$.\sidenote{A set of possible critics is presented in \cite{tschannen_19} Most functions are either the cosine similarity or the dot product operator. The cosine similarity has the nice advantage of presenting the highest similarity to itself since $cos(a, a) =1$. While with the dot-product other vectors can have higher similarities: $dot(a, a) < dot (a, b)$.}

%We illustrate the training objective in \reffig{contrastive-2}.

% \begin{figure}[htb!]
% 	\includegraphics[width=10cm]{images/contrastive_2.png}
% 	\caption[Contrastive learning]{Illustration of the contrastive learning objective.}
% 	\labfig{contrastive-2}
% \end{figure}

\subsection{Construction of the dataset}
\labsec{1B:dataset}

The contrastive training method supposes to build a dataset such that each sample $x$ is combined with another sample $x^+$, which is somehow \textit{close} and negative samples $s^-_1 \cdots s^-_K$, which are not related. In \refch{structure-scale}, we constructed positive pairs by simply associating context sentences and negative by considering non-context sentences. Therefore, we only needed a corpus of raw text for which the sentence order is preserved to train our model. In this experiment, we adopt a more refined approach. Instead of raw text, we extract sentences from specific mediums such as internet forums and manually labeled datasets. Indeed, as detailed in \refsec{1B:batches}, a better selection of negative samples may drastically increase the results.

As with other attempts to scale model size \parencite{liu_2019, radford_2019, brown_20}, we also aim to scale the dataset size. While we use a 40M sentences dataset for \textcite{simoulin_2021a}, here, we aim to gather a dataset of 1B sentence pairs. The task is far from trivial as we need to constitute a dataset with sentence pairs $(a_i, p_i)$ such that sentences from the pair have a close meaning. We constitute pairs by using  medium and documents specific structure such as (query, answer-passage), (question, duplicate\_question), (paper title, cited paper title). We do not build new datasets but instead rely on existing work and aggregate many existing datasets enumerated in \reftab{tab:1B:dataset}.

The majority of the datasets is built out of Reddit comments. Reddit website aggregates news and lets users post links and discuss through threads. We use scripts from PolyAI to generate tuples given the first comment for each response.\sidenote{\url{https://github.com/PolyAI-LDN/conversational-datasets/tree/master/reddit}} We use the same filters as \textcite{henderson_2019} and filter out samples with more than 128 characters or fewer than 9 characters. I personally took care of this data collection operation.

\begin{table*}[!htb]
\centering
\footnotesize
\begin{tabularx}{16cm}{@{}lp{4.5cm}Y@{} }
\toprule
\textbf{Dataset} & \textbf{Reference} & \textbf{Number of training pairs} \\
\midrule
\midrule 
\href{https://github.com/PolyAI-LDN/conversational-datasets/tree/master/reddit}{Reddit Comments} (2015-2018) & \textcite{henderson_19} & \num{726484430} \\
\href{https://github.com/allenai/s2orc}{S2ORC} citation pairs (abstracts) & \textcite{lo_20} & \num{116288806} \\
\href{https://github.com/afader/oqa\#wikianswers-corpus}{WikiAnswers} duplicate question pairs & \textcite{fader_14} & \num{77427422} \\
\href{https://github.com/facebookresearch/PAQ}{PAQ} (question. answer) pairs & \textcite{lewis_21} & \num{64371441} \\
\href{https://github.com/allenai/s2orc}{S2ORC} citation pairs (titles) & \textcite{lo_20} & \num{52603982} \\
\href{https://github.com/allenai/s2orc}{S2ORC} (title. abstract) & \textcite{lo_20} & \num{41769185} \\
\href{https://huggingface.co/datasets/flax-sentence-embeddings/stackexchange_xml}{Stack Exchange} (title. body) pairs & - & \num{25316456} \\
\href{https://microsoft.github.io/msmarco/}{MS MARCO} triplets & \textcite{craswell_21} & \num{9144553} \\
\href{https://github.com/allenai/gooaq}{GOOAQ} & \textcite{khashabi_21} & \num{3012496} \\
\href{https://www.kaggle.com/soumikrakshit/yahoo-answers-dataset}{Yahoo Answers} (title. answer) & \textcite{zhang_15} & \num{1198260} \\
\href{https://huggingface.co/datasets/code_search_net}{Code Search} & - & \num{1151414} \\
\href{https://cocodataset.org/\#home}{COCO} image captions & \textcite{lin_14} & \num{828395} \\
\href{https://github.com/allenai/specter}{SPECTER} citation triplets & \textcite{cohan_20} & \num{684100} \\
\href{https://www.kaggle.com/soumikrakshit/yahoo-answers-dataset}{Yahoo Answers} (question. answer) & \textcite{zhang_15} & \num{681164} \\
\href{https://www.kaggle.com/soumikrakshit/yahoo-answers-dataset}{Yahoo Answers} (title. question) & \textcite{zhang_15} & \num{659896} \\
\href{https://huggingface.co/datasets/search_qa}{SearchQA} & \textcite{dunn_17} & \num{582261} \\
\href{https://huggingface.co/datasets/eli5}{Eli5} & \textcite{fan_19} & \num{325475} \\
\href{https://shannon.cs.illinois.edu/DenotationGraph/}{Flickr 30k} & \textcite{young_14} & \num{317695} \\
\href{https://huggingface.co/datasets/flax-sentence-embeddings/stackexchange_xml}{Stack Exchange} duplicate questions (titles) & - & \num{304525} \\
AllNLI (\href{https://nlp.stanford.edu/projects/snli/}{SNLI} and \href{https://cims.nyu.edu/~sbowman/multinli/}{MultiNLI}) & \textcite{bowman_15, williams_18b} & \num{277230} \\
\href{https://huggingface.co/datasets/flax-sentence-embeddings/stackexchange_xml}{Stack Exchange} duplicate questions (bodies) & - & \num{250519} \\
\href{https://huggingface.co/datasets/flax-sentence-embeddings/stackexchange_xml}{Stack Exchange} duplicate questions (titles and bodies) & - & \num{250460} \\
\href{https://github.com/google-research-datasets/sentence-compression}{Sentence Compression} & \textcite{filippova_13} & \num{180000} \\
\href{https://github.com/pvl/wikihow_pairs_dataset}{Wikihow} & \textcite{koupaee_18} & \num{128542} \\
\href{https://github.com/chridey/altlex/}{Altlex} & \textcite{hidey_16} & \num{112696} \\
\href{https://quoradata.quora.com/First-Quora-Dataset-Release-Question-Pairs}{Quora Question Triplets} & - & \num{103663} \\
\href{https://cs.pomona.edu/~dkauchak/simplification/}{Simple Wikipedia} & \textcite{coster_11} & \num{102225} \\
\href{https://ai.google.com/research/NaturalQuestions}{Natural Questions (NQ)} & \textcite{kwiatkowski_19} & \num{100231} \\
\href{https://rajpurkar.github.io/SQuAD-explorer/}{SQuAD2.0} & \textcite{drajpurkar_18} & \num{87599} \\
\href{https://huggingface.co/datasets/trivia_qa}{TriviaQA} & - & \num{73346} \\
\midrule
\textbf{Total} & & \textbf{\num{1124818467}} \\
\bottomrule
\end{tabularx}
\caption{One billion sentence pairs dataset. We use already existing datasets accessible in open source or for which the raw data and pre-processing scripts were available. For each sub-dataset, we provide the link to the available resources (existing dataset or pre-processing scripts).}
\labtab{tab:1B:dataset}
\end{table*}

\subsection{Construction of the mini batches}
\labsec{1B:batches}

When building models. the selection of pairs forming a batch is crucial. We present here our strategy to constitute mini-batches and the impact it may have on the resulting embeddings.
% Here we examine the batch characteristics and their impact on the embeddings.

\paragraph{Batch size} The Quickthought method detailed in \refsec{training:self-supervised} uses a rather important batch size of 400. In fact, studies show that the larger the batch, the better the performance \parencite{chen_20a, qu_21}. This trend is illustrated in \reffig{batch-size} extracted from \textcite{qu_21}. However, too important batch size may decrease the results (the same asymptotic phenomenon is observed in \textcite{chen_20a}). We benefited from efficient hardware infrastructure to run the project: 7 TPUs v3-8, as well as guidance from Google’s Flax, JAX, and Cloud team members about efficient deep learning frameworks. We use the largest batch size that our hardware could fit, in our case, 64.

\begin{figure}[htb!]
	\includegraphics[width=10cm]{images/batch-size.png}
	\caption[Batch size]{Influence from the batch size and selection of hard negative on downstream evaluation. The figure is extracted from \textcite{qu_21}.}
	\labfig{batch-size}
\end{figure}

\paragraph{Hard Negatives} We may build batches by uniformly selecting samples from the training data. However, as detailed in \textcite{robinson_21} or \textcite{qu_21}, the selection of "good negative examples" significantly impacts the training process. The impact of hard negative is illustrated in \reffig{batch-size}.
Hard negative examples should not correspond to the anchor point but still, be difficult to distinguish from the correct associations. 
% Hard negatives are sample $p_j$. which are hard to distinguish from $p_i$. 
In our example, it could be the pairs «~What is the capital of France?~» and «~What is the capital of the US?~» which have a close semantic content and require precisely understanding the full sentence to be answered correctly. On the contrary, the samples «~What is the capital of France?~» and «How many Star Wars movies are there?» are less difficult to distinguish since they do not refer to the same topic.

\paragraph{Cross dataset batches} In our case, the dataset is a concatenation of several sub-datasets (\reftab{tab:1B:dataset}). Each sub-dataset is built on distinct topics, domains, or semantic relations in the pair. We want to avoid the case where our model learns disjoint embedding spaces for each sub-dataset. On the other hand, mixing all sub-datasets in the same batch may deteriorate the hard negative proportion as samples issued from two sub-datasets should be easy to differentiate. To address both requirements, we build batches from the mix of only two sub-datasets. We aim, therefore, to learn a global structure between topics and not only a local structure within a topic while not deteriorating the proportion of hard negatives.

\subsection{Evaluation}
\labsec{scale:evaluation}

As detailed in \refsec{survey:downstream}, sentence embeddings are traditionally evaluated on the SentEval benchmark. To compare the embeddings with models developed in previous sections, we therefore evaluate our encoders on SentEval. However, as detailed in the same section, the benchmark suffers from practical limitations or biases. For this project, we therefore used SEB (Sentence Embedding Benchmark), a dedicated benchmark to compare our models.\sidenote{\url{https://github.com/nreimers/se-benchmark}} The SEB benchmark aggregates multiple general-purpose sentence evaluation tasks. The tasks, detailed below, are formatted as binary classification, clustering, reranking, retrieval, and semantic textual similarity (STS). All tasks use the embeddings as features and compare them using similarity metrics. Most importantly, they do not require the training of additional classifiers.

\paragraph{Binary classification} aims at predicting a binary relation between a pair of sentences. It computes the cosine similarity between every pair of sentences. We then classify the sentence pairs by comparing their similarity score to a given threshold. We set the threshold to ensure the best score on the development set. The task includes identifying paraphrases from LanguageNet, a collection of sentences from Twitter linked through shared URLs \sidenote{\url{https://languagenet.github.io/}} or the SemEval-2015 Task 1,\sidenote{\url{https://alt.qcri.org/semeval2015/task1/}} and identifying duplicated questions.\sidenote{\url{https://www.aclweb.org/anthology/D18-1131/}} We measure the performance using the average precision (AP).\sidenote{The average precision (AP) has values between 0 and 1 (higher is better). AP is defined as $\sum_n(R_n-R_{n-1})P_n$ with $P_n$ and $R_n$ are the precision and recall at the nth threshold. \url{https://scikit-learn.org/stable/modules/generated/sklearn.metrics.average_precision_score.html}}

\paragraph{Clustering} organizes documents into semantically consistent groups. We use data from web forums and newsgroups, which organizes posts given their topics. We use embeddings as features for K-Means clustering and evaluation using the V-measure.\sidenote{The V-measure evaluates the quality of a clustering given the ground truth labels. The score has positive values between 0 and 1, with higher values indicating better results. The V-measure is the harmonic mean between homogeneity and completeness. Homogeneity evaluates if each cluster contains only members of a single class. Completeness determines if all members of a class are assigned to the same cluster. \url{https://scikit-learn.org/stable/modules/generated/sklearn.metrics.v_measure_score.html}} The clustering task includes the 20Newsgroups,\sidenote{\url{https://scikit-learn.org/0.19/datasets/twenty_newsgroups.html}} and clustering threads from StackExchange and Reddit. 

\paragraph{Retrieval} aims at retrieving documents from a corpus that match the semantic content of a given query. We use datasets scraped from web forums and question-answering websites. On such platforms, experienced users can flag a question as a duplicate if it has already been answered elsewhere. We use these annotations to associate a given question to a list (of variable size) of semantically equivalent formulations. Given the embedding of a query, we compute the cosine similarity with other questions from the dataset and retrieve the top-$k$ most similar ones (by default, we use $k=10$). We then compare our predicted list with the related questions using the mean average precision (MAP@100). The task includes CQADupStack, a dataset with duplicate question information from StackExchange subforums\sidenote{\url{http://nlp.cis.unimelb.edu.au/resources/cqadupstack/}} and the Quora Question Pairs dataset.\sidenote{\url{https://quoradata.quora.com/First-Quora-Dataset-Release-Question-Pairs}}
%  the Mean Reciprocal Rank (MRR) or 

\paragraph{Reranking} ranks a list of documents given their semantic similarity with a given query. In our setup, the task takes a query and a fixed-length list of documents as input. Each document in the list is either "similar" or "non-similar" to the query. We compute the cosine similarity between the embedding of the query and each document and sort them in decreasing order. We then compare the sorted list with the document ordered as similar first, followed by non-similar. We also use mean average precision to measure the quality of the ranking. As in the retrieval task, data are collected from web forums but with a different format and labeling process. We use a collection of questions taken from AskUbuntu.com 2014 corpus dump.\sidenote{\url{https://github.com/taolei87/askubuntu}} SciDocs, which consider scientific papers as related based on their inter-citations \sidenote{\url{https://allenai.org/data/scidocs}} and the Stack Overflow Duplicate Questions Task.\sidenote{\url{https://www.microsoft.com/en-us/research/uploads/prod/2019/03/nl4se18LinkSO.pdf}}

\paragraph{Semantic Textual Similarity (STS)} measures the semantic similarity between two sentences. Annotators assign a similarity score for pair of sentences, ranging from 0 for no overlap to 5 for meaning equivalence. The annotation doesn't require formal linguistic expertise. Performance compares the correlation between predicted scores and human judgments with Pearson correlation. The predicted scores directly measure the cosine similarity between two sentence pairs and compare it with human gold annotations (scaled between 0 and 1). The evaluation datasets include the STSBenchmark which includes datasets used for the SemEval task from 2012 to 2017.\sidenote{\url{http://ixa2.si.ehu.eus/stswiki/index.php/STSbenchmark}} the SICK-R task (already introduced in \refsec{sts}) and BIOSSES  which comprises 100 sentence pairs from the biomedical field.\sidenote{\url{https://tabilab.cmpe.boun.edu.tr/BIOSSES/DataSet.html}}

% \begin{table}[ht!]
% \begin{center}
% \small
% {\renewcommand{\arraystretch}{1.25}
% \begin{tabular}{|c|p{0.88\linewidth}|}
% \hline
% \multirow{4}{*}{5}& \emph{The two sentences are completely equivalent. as they mean the same thing.}\\
% \cline{2-2}
% & The bird is bathing in the sink. \newline Birdie is washing itself in the water basin. \\
% \hline
% \multirow{4}{*}{4}  & \emph{The two sentences are mostly equivalent. but some {\it unimportant} details differ.}\\
% \cline{2-2}
% & Two boys on a couch are playing video games. \newline Two boys are playing a video game. \\
% \hline
% \multirow{4}{*}{3} & \emph{The two sentences are roughly equivalent. but some {\it important information} differs/missing.}\\
% \cline{2-2}
% & John said he is considered a witness but not a suspect. \newline ``He is not a suspect anymore.'' John said.  \\
% \hline
% \multirow{4}{*}{2} & \emph{The two sentences are not equivalent. but share some details.}\\
% \cline{2-2}
% & They flew out of the nest in groups. \newline They flew into the nest together. \\
% \hline
% \multirow{4}{*}{1} & \emph{The two sentences are not equivalent. but are on the same topic.}\\
% \cline{2-2}
% & The woman is playing the violin. \newline The young lady enjoys listening to the guitar. \\
% \hline
% \multirow{4}{*}{0} & \emph{The two sentences are completely dissimilar.}\\
% \cline{2-2}
% & The black dog is running through the snow. \newline A race car driver is driving his car through the mud. \\
% \hline
% \end{tabular}
% }
% \end{center}
% \caption{Similarity scores with explanations and English examples. The table is exctracted from \textcite{cer_17}.}
% \label{fig:annotationcore}
% \end{table}

\section{Experiments}
\labsec{scale:experiments}

We fine-tune existing pre-trained models with our contrastive learning objective. As a sentence representation, we take the mean of every token hidden state from transformer models. We applied 500 warm-up steps and use a batch size of 64 if not explicitly specified otherwise. We create 20 general-purpose sentence transformers models such as Mini-LM \parencite{wang_20a}, RoBERTa \parencite{liu_2019}, DistilRoBERTa, a distilled version of the RoBERTa-base model following the same training procedure as DistilBERT \parencite{sanh_19}, and MPNet \parencite{song_20}.\sidenote{All models created during the challenge are available as open-source contributions in our HuggingFace repository \url{https://huggingface.co/flax-sentence-embeddings}.} The challenge was limited in time, and we could not extensively train all models with the same number of steps. We train RoBertA-large and MPNet-base for 400k steps. Mini-LM-12 for 540 steps. RoBERTa-distill-base for 920 steps and Mini-LM-6 for \numprint{1000} steps. However, models may therefore not be directly compared.

\paragraph{Analysis of the pre-training} We evaluate all our models on the Sentence Embedding Benchmark (SEB) detailed in \refsec{scale:evaluation} and SentEval benchmark introduced in \refsec{survey:downstream}. \reftab{scale:seb-senteval} reports the mean score for each model on both benchmarks. For each model, we report the score for the raw model and for the model further tuned with our additional contrastive pre-training.

\begin{table*}[!htb]
\centering
\small
\begin{tabularx}{16cm}{@{}lYYYYY@{} }
\toprule
&  & \multicolumn{2}{c}{\textbf{SentEval}} & \multicolumn{2}{c}{\textbf{SEB}}\\
\textbf{Model} & \textbf{\# parameters} & w/o contrastive pre-training & w/ contrastive pre-training & w/o contrastive pre-training & w/ contrastive pre-training\\
\midrule
\midrule 
\href{https://huggingface.co/flax-sentence-embeddings/all_datasets_v4_MiniLM-L6}{Mini-LM-6} & 22.7M & 80.6 & 83.5 & 42.0 & 68.1 \\
\href{https://huggingface.co/flax-sentence-embeddings/all_datasets_v4_MiniLM-L12}{Mini-LM-12} & 33.4M & 81.7 & 84.8 & 40.7 & 68.6 \\
\href{https://huggingface.co/flax-sentence-embeddings/all_datasets_v3_distilroberta-base}{DistilRoBERTa} & 82.1M & 83.5 & 86.0 & 44.9 & 68.7 \\
\href{https://huggingface.co/flax-sentence-embeddings/all_datasets_v4_mpnet-base}{MPNet-base} & 109.5M & \textbf{83.5} & \textbf{87.4} & 41.6 & 69.5 \\
\href{https://huggingface.co/flax-sentence-embeddings/all_datasets_v3_roberta-large}{RoBERTa-large} & 355.4M & 81.7 & 87.3 & \textbf{45.3} & \textbf{70.0} \\
\bottomrule
\end{tabularx}
\caption{\labtab{scale:seb-senteval} Evaluation on SentEval and SEB. We report the mean score over all tasks from the benchmark. We compare models pre-trained with and without our contrastive procedure. We report the best results for each category in \textbf{bold}.}
\end{table*}

In general, transformer models with a higher number of parameters reach higher scores. However, we observe asymptotic behavior for this trend. The RoBERTa-large model reaches performance similar to MPNet-base despite having 3 times more parameters. Moreover, on both benchmarks, the contrastive pre-training procedure has an important impact. Model performance increases up to 5.6 points on SentEval and more than 20 points on SEB. This confirms the relevance of the procedure for training sentence encoders with transformer architectures. Finally, we observe less disparity between models on the SEB benchmarks for which all scores are very close. 

\paragraph{Sentence Embedding Benchmark (SEB)} \reftab{scale:seb} reports the detailed evaluation of our models on SEB. The results are more difficult to interpret. While larger models tend in general to perform better, no model seems to consistently outperform others. It is also difficult to identify specific model behavior across task classes.

\setlength\tabcolsep{2pt} % default value: 6pt
\begin{table*}[!htb]
\footnotesize
\centering {
\begin{tabularx}{16cm}{@{}l Y Y Y | Y Y Y | Y Y | Y Y Y | Y Y Y@{}}
\toprule
& \multicolumn{3}{c}{\textbf{Binary Classification}} & \multicolumn{3}{c}{\textbf{Clustering}} & \multicolumn{2}{c}{\textbf{Retrieval}} & \multicolumn{3}{c}{\textbf{Re-ranking}} & \multicolumn{3}{c}{\textbf{STS}}\\
& {\tiny\textbf{Sprint}} & {\tiny\textbf{Twitter}} & {\tiny\textbf{SemEval}} & {\tiny\textbf{20 News Groups}} & {\tiny\textbf{Stack Exchange}} & {\tiny\textbf{Reddit}} & {\tiny\textbf{CQA}} & {\tiny\textbf{Quora}} & {\tiny\textbf{Ask Ubuntu}} & {\tiny\textbf{Sci Docs}} & {\tiny\textbf{Stack Overflow}} & {\tiny\textbf{SICK-R}} & {\tiny\textbf{STS}} & {\tiny\textbf{BIOSSES}}\\\midrule
\midrule
\href{https://huggingface.co/flax-sentence-embeddings/all_datasets_v4_MiniLM-L6}{Mini-LM-6} & \textbf{94.6} & 84.7 & 67.9 & 46.2 & 54.4 & 50.2 & 28.6 & 84.9 & 63.5 & 87.1 & 50.8 & 77.2 & 82.0 & 81.6 \\
\href{https://huggingface.co/flax-sentence-embeddings/all_datasets_v4_MiniLM-L12}{Mini-LM-12} & 92.6 & 84.8 & 70.0 & 46.9 & 52.4 & 50.6 & 29.4 & \textbf{85.1} & 64.1 & 87.2 & 51.5 & 78.9 & 83.1 & 83.6 \\
\href{https://huggingface.co/flax-sentence-embeddings/all_datasets_v3_distilroberta-base}{DistilRoBERTa} & 46.8 & 65.2 & \textbf{83.3} & 30.5 & 83.4 & 53.1 & 80.1 & 82.4 & 87.8 & \textbf{93.8} & 48.7 & 51.4 & 71.1 & 84.0 \\
\href{https://huggingface.co/flax-sentence-embeddings/all_datasets_v4_mpnet-base}{MPNet-base} & 90.2 & \textbf{85.1} & 73.9 & \textbf{49.8} & 52.9 & 54.1 & 31.8 & 84.7 & 65.9 & 88.6 & 52.0 & \textbf{80.5} & \textbf{83.4} & 80.4 \\
\href{https://huggingface.co/flax-sentence-embeddings/all_datasets_v3_roberta-large}{RoBERTa} & 49.4 & 66.8 & 82.5 & 32.8 & \textbf{83.4} & \textbf{55.6} & \textbf{81.4} & 83.5 & \textbf{88.7} & 92.1 & \textbf{52.5} & 52.2 & 75.3 & \textbf{84.5} \\
\bottomrule
\end{tabularx}}
\caption{\labtab{scale:seb} Detailed results on the Sentence Embeddings Benchmark (SEB). All models are further pre-trained using our contrastive objective on our one billion sentences corpus. The best results in each section are shown in \textbf{bold}. We report the mean average precision (AP) for binary prediction and retrieval tasks, the V-measure for clustering tasks, and the Spearman rank correlation for semantic textual similarity task (STS). For each task, we report the best score obtained with cosine, euclidean, and Manhatten distance. We report all metrics by convention as $\times 100$. We show the best results in \textbf{bold}.}
\end{table*}
\setlength\tabcolsep{6pt} % default value: 6pt

\paragraph{SentEval} Finally, we aim to compare our transformer models with previously presented encoders from \refch{structure-scale}. We present the detailed results on SentEval in \reftab{scale:senteval}. We divide results given encoder architecture: LSTM and transformer-based models. 

\begin{table*}[!htb]
\footnotesize
% \begin{minipage}{\textwidth}
\centering {
\begin{tabularx}{16cm}{@{}l c | Y Y Y Y Y Y Y Y Y Y Y @{}}
\toprule
\multirow{2}{*}{\textbf{Model}} & \multirow{2}{*}{\textbf{Dim}} & \multirow{2}{*}{\textbf{Avg.}} & \multirow{2}{*}{\textbf{MR}} & \multirow{2}{*}{\textbf{CR}} & \multirow{2}{*}{\textbf{SUBJ}} & \multirow{2}{*}{\textbf{MPQA}} & \multirow{2}{*}{\textbf{TREC}} &  \multicolumn{2}{c}{\textbf{MRPC}} &  \multicolumn{3}{c}{\textbf{SICK-R}}\\%\cmidrule(r){9-10} \cmidrule(r){11-13}
 &  &  &  &  &  &  &  & \textbf{Acc} & \textbf{F1} & \textbf{$r$} & \textbf{$\rho$} & \textbf{MSE}\\\midrule
\multicolumn{13}{c}{\textit{Recurrent models}} \\\midrule
FastSent & $\leq500$ & --- & 70.8 & 78.4 & 88.7 & 80.6 & 76.8 & 72.2 & 80.3 & --- & --- & ---\\
% FastSent + AE & $\leq500$ & 2 & 71.8 & 76.7 & 88.8 & 81.5 & 80.4 & 71.2 & 79.1 & --- & --- & ---\\
Skipthought & \numprint{4800} & 83.8 & 76.5 & 80.1 & 93.6 & 87.1 & 92.2 & 73.0 & 82.0 & 85.8 & 79.2 & 26.9\\
% Skipthought + LN & \numprint{4800} & 672 & 79.4 & 83.1 & 93.7 & 89.3 & --- & --- & --- & 85.8 & 78.8 & 27.0\\
Quickthought & \numprint{4800} & \textbf{86.1} & 80.4 & 85.2 & 93.9 & 89.4 & 92.8 & 76.9 & 84.0 & 86.8 & 80.1 & 25.6\\
InferSent & \numprint{4096} & --- & \textbf{81.1} & \textbf{86.3} & 92.4 & 90.2 & 88.2 & 76.2 & 83.1 & \textbf{\underline{88.4}} & --- & ---\\
% DisSent Books 5 & \numprint{4096} & --- & 80.2 & 85.4 & 93.2 & 90.2 & 91.2 & 76.1 & --- & 84.5 & --- & ---\\
DisSent & \numprint{4096} & --- & 79.8 & 85.0 & 93.4 & \textbf{\underline{90.5}} & \textbf{93.0} & 76.1 & --- & 85.4 & --- & ---\\
\textsc{Dep}. \textsc{Seq}$^\dagger$ & \numprint{4800} & 85.3 & 79.7 & 82.2 & 94.4 & 88.6 & 91.0 & \textbf{\underline{77.9}} & \textbf{\underline{84.4}} & 86.6 & 79.8 & 25.5\\
\textsc{Dep}. \textsc{Const}$^\dagger$ & \numprint{4800} & 86.0 & 80.7 & 83.6 & \textbf{94.9} & 89.2 & 92.6 & 76.8 & 83.6 & 87.0 & \textbf{80.3} & \textbf{24.8}\\
\midrule\multicolumn{13}{c}{\textit{Transformers}} \\\midrule
\textsc{Bert}-base [CLS] & $768$ & --- & 78.7 & 84.9 & 94.2 & 88.2 & 91.4 & 71.1 & --- & 75.7$^\dagger$ & --- & ---\\
\textsc{Bert}-base [NLI] & $768$ & --- & 83.6 & \textbf{\underline{89.4}} & 94.4 & 89.9 & 89.6 & \textbf{76.0} & --- & 84.4$^\dagger$ & --- & ---\\
\href{https://huggingface.co/flax-sentence-embeddings/all_datasets_v4_MiniLM-L6}{Mini-LM-6}$^\dagger$ & $384$ & 83.5 & 76.1 & 82.0 & 92.2 & 87.4 & 90.2 & 72.3 & 80.9 & 86.5 & 79.9 & 25.6 \\
\href{https://huggingface.co/flax-sentence-embeddings/all_datasets_v4_MiniLM-L12}{Mini-LM-12}$^\dagger$ & $384$ & 84.8 & 77.9 & 84.3 & 92.3 & 88.4 & 91.8 & 73.2 & \textbf{82.1} & 87.1 & 80.9 & 24.7 \\
\href{https://huggingface.co/flax-sentence-embeddings/all_datasets_v3_distilroberta-base}{DistilRoBERTa}$^\dagger$ & $768$ & 86.0 & 80.8 & 86.2 & 93.4 & 87.6 & 94.8 & 73.5 & 80.9 & \textbf{87.8} & 82.3 & 23.4 \\
\href{https://huggingface.co/flax-sentence-embeddings/all_datasets_v4_mpnet-base}{MPNet-base}$^\dagger$ & $768$ & \textbf{\underline{87.5}} & 84.8 & 87.7 & 94.1 & 89.4 & 94.4 & 75.1 & 83.4 & 87.9 & 82.9 & \textbf{\underline{23.3}} \\
\href{https://huggingface.co/flax-sentence-embeddings/all_datasets_v3_roberta-large}{RoBERTa-large}$^\dagger$ & \numprint{1024} & 87.3 & \textbf{\underline{87.0}} & 88.6 & \textbf{\underline{95.0}} & \textbf{89.0} & \textbf{\underline{96.8}} & 74.0 & 80.6 & 83.7 & \textbf{\underline{83.9}} & 35.5 \\
\bottomrule
\end{tabularx}}
\caption{\labtab{scale:senteval} SentEval Task Results Using Fixed Sentence Encoder. $^\dagger$\ indicates models that we had to re-train. FastSent is reported from \textcite{hill_16}. Skipthoughts results from \textcite{kiros_15} Skipthoughts + LN which includes layer normalization method from \textcite{ba_16}. We considered the Quickthought results \cite{logeswaran_18} with a pre-training on the bookcorpus dataset. DisSent and Infersent are reported from \textcite{nie_19} and \textcite{conneau_17} respectively. Pre-trained transformers results are reported from \textcite{reimers_19}. Best results in each section are shown in \textbf{bold}. best results overall are \underline{underlined}. Performance for \textbf{SICK-R} results are reported by convention as $\rho \text{ and } r \times 100$.}
\end{table*}

We obtain state-of-the-art results on the benchmark, and transformer-based models outperform other architectures. On many tasks, we also outperform the approach proposed in \textcite{reimers_19}, which fine-tune \textsc{Bert} on natural language inference data. However, LSTM based models remain competitive on several tasks. Moreover, only transformers with the highest number of parameters outperform previous recurrent models. It is also important to stress that the setup is not directly comparable as transformer models are trained on datasets many orders of magnitude larger.

\section{Conclusion and future work}


% \bcomment{You might well comment that bigger is not necessarily better ? throwing more data is not the only solution:what explains your MRPC result it is better with a biased model trained on relatively less data}{or maybe that bigger is better but in which proportions ?}
% \bcomment{See my notes in the conclusion}{You might wish to conclude on these aspects, I find your MRPC results quite interesting to analyze in the conclusion with these respects}{}

In this section, we studied the extent to which scaling may improve sentence encoder performance. We adapt the standard contrastive pre-training method to train large transformer models on a large dataset. We obtain state-of-the-art results on sentence embedding benchmarks. We observe the importance of contrastive pre-training to achieve competitive results. In some proportion, it seems possible to balance linguistic insights and the refinement of the encoder architecture by just increasing the dataset and model size. 

However, it is important to stress that the setup is completely unbalanced and the comparison rather unfair regarding the infrastructure hardware, the data used for training, and the model size.

Finally, bigger is not necessarily better. Indeed, while large models outperform previous recurrent and structured approaches on average, this is not the case for every task. In particular for the MRPC task: the Microsoft Research Paraphrase Corpus contains \numprint{5801} sentence pairs, each hand-labeled with a binary judgment as to whether the pair constitutes a paraphrase. The sentences are mined from news clusters and includes a wide range of lexical as well as syntactic variations.


\setlength\tabcolsep{6pt} % default value: 6pt

% % \setchapterimage[6cm]{seaside}
\setchapterstyle{kao}
\setchapterpreamble[u]{\margintoc}
\chapter{The first large incremental language model for French}
% using a generative objective
\labch{generative}

% \cleanchapterquote{\textup{[\,\dots]} le sujet s’éloigne du verbe et \textup{[\,\dots]} le complément vient se poser quelque part dans le vide.}{Samuel Beckett}{Malone meurt, 1951}
\cleanchapterquote{La nuit se faisait assez obscure, les étoiles semblaient dormir de temps à autre, cependant le peu de clarté qui me permit de marcher la nuit dans la chambre éveilla en moi une profonde pitié de ce que je faisais là, et cette peur de l'avenir me devint plus vive et plus aiguë.}{Automatic text generation}{$\text{GPT}_{fr}$-1B, 2021}

% \section{Training efficient sentence embedding models using structured encoders}
% \labsec{structure-scale}
% Previous sections examined the training of models by predicting relationships between sentences. Here, 

The previous section discussed enhancing large transformer language models using self-supervised objectives adapted for sentence embeddings. Our procedure reached state-of-the-art results on many downstream evaluation tasks. However, we did not perform the initial pre-training ourselves but instead used already pre-trained transformers for English. This section presents a quantitative evaluation of the effort required to pre-train such models in terms of data collection, computing infrastructure configuration, model development, and evaluation. To be as representative as possible of the process, we chose a language and architecture design for which relatively few resources
%french is usually said to be a well resourced language
were available. The section thus relates the pre-training of the first large incremental language model for French \parencite{simoulin_2021c}.

Auto-encoding models have already been developed in French, namely CamemBERT \parencite{martin_20} or FlauBERT \parencite{le_20a, le_20b}. However, to the best of our knowledge, this contribution is the first peer-reviewed to adapt pre-trained incremental transformers to French. In particular, we introduced a French version from the well-known GPT model \parencite{radford_2018, radford_2019, brown_20}. GPT, which stands for Generative Pre-trained Transformer, is an incremental language model developed by Open AI research laboratory\sidenote{\url{https://openai.com/}}. \bert and other models presented in the previous sections act as encoders, taking text as input and producing vector representations as output. On the contrary, GPT acts as a decoder, taking text as input and producing text as output. 
%rappeler ce qu'est GPT, et donner une spécification de ton modèle dans le chapitre, segmentation etc; contraster l'objectif avec celui de BERT etc histoire que le chapitre soit stand alone.
% Parler du défi pour entrainer ce type de model
%\bcomment{titre mal choisi}{Camembert et flaubert sont venus avant}

From a modeling point of view, \gpt is an incremental language model whose pre-training objective is relatively similar to the one from a n-gram language model already used 30 years ago. But while n-gram language models typically use a context size of 5 or fewer words, \gpt extends the context size to \numprint{1024} tokens. From a practical point of view, pre-training such a model is a superlative project, which is far from trivial. First, it requires large corpora of raw text—up to billion of tokens. Second, the analysis and evaluation of these models require access to relevant and rigorous benchmarks. Last but not least, pre-training also requires significant computing power. It requires distributing the training on multiple computing units across multiple computing nodes. Typically dozens of graphics processing unit (GPUs) or tensor processing units (TPUs) %\sidenote{\url{https://cloud.google.com/tpu}} 
operating for several days. In that regard, this work benefited from access to the IDRIS computing facilities through the allocation of 2020-AD011011823 allocated by GENCI. Our model was among the firsts to be trained on the super-computer Jean-Zay, less than a year after its inauguration in January 2020. Our contributions are the following:
\begin{itemize}
    \item We propose a corpus dedicated to the training of transformers language models in French. We detail the construction of this corpus in \refsec{generative:corpus} ;
    \item We train two models with a large number of parameters, which we release as open-source contributions\sidenote{\url{https://huggingface.co/asi/gpt-fr-cased-base}}. Hopefully, these models can be used in academic as well as industrial settings ;
    \item We replicate English evaluation benchmarks for French language models. This evaluation setup allows for the comparison of models and is detailed in \refsec{generative:evaluation}.
\end{itemize}

% We organize the section as follow: We first present the construction of the training and evaluation corpora (\refsec{generative:corpus}). We then detail the architecture and training setup in \refsec{generative:models}. Finally,  \refsec{generative:evaluation} examines our model abilities through several evaluation tests.

% In this case, we are mining a specific datasets with many sentence pairs. \bcomment{There are also methods for reconstructing a noised input, for example detecting masked words or predicting what the next word will be. The corpora required for such auto-encoding methods are more simple and only consist in individual documents.}{raw text ?}

% In this section, we introduce a French \bcomment{adaptation}{version} from the well-known \bcomment{GPT}{rappeler ce qu'est GPT, et donner une spécification de ton modèle dans le chapitre, segmentation etc; contraster l'objectif avec celui de BERT etc histoire que le chapitre soit stand alone} model. GPT relies on transformer pre-trained architectures, which profoundly transformed natural language processing methods. Such models are pre-trained using a self-supervised objective and are therefore specific to a given language. The model, equivalent to GPT-2 in English, contains more than 1 billion parameters. \bcomment{This work benefited from access to the IDRIS computing facilities through the allocation of 2020-AD011011823 allocated by GENCI.}{dire que ce modèle est un des premiers à avoir été entrainé sur l'IDRIS, mettre en avant dans le chapitre le coté infrastructure : calcul distribué, combien de noeuds, temps d'entrainement etc.}

% Although some models may exist in French, the majority is released in \bcomment{English}{cela vaudrait le coup d'inclure une section qui résume un peu ce qui change; est-il aussi facile de trouver les données ? commenter le zero-shot etc.} only.  


We organize the section as follow: \refsec{generative:incremental} first reviews the main characteristic of incremental models, their originality and main distinctions with standard encoders. \refsec{generative:corpus} presents the constitution of the pre-training corpora. We then detail the training and evaluation procedure in \refsec{generative:training} and \refsec{generative:evaluation}. Finally, we discuss the limits and ethical considerations in \refsec{generative:limits}.


%GPT achieves impressive language generation performances. 

% Within my laboratory, I led the project to train the first large language model in French \parencite{simoulin_2021c}. We obtained a dedicated computation grant on public French HPC computer Jean Zay. The model, equivalent to GPT-2 in English, contains more than 1 billion parameters. We built a dedicated training corpus and parallelized the training between multiple nodes and compute units. I am particularly proud of this project, as we contributed to the resources available in French. We released the model in Open-Source for research and business application purposes. 

%TODO il faut parler des embeddings de phrase de ces modèles et le relier à la problématique de la thèse

% This is, to the best of our knowledge, the first peer-reviewed contribution that \bcomment{adapts pre-trained generative transformers}{nope there are camembert and flaubert} to French. We adapt OpenAI GPT et GPT-2 \parencite{radford_2018, radford_2019} for French. Our contributions are the following:
% \begin{itemize}
%     \item We propose a corpus dedicated to the training of transformers language models in French. The construction of thus corpus is detailed in \refsec{generative:corpus} ;
%     \item We \acomment{trained} two models with a large number of parameters, which we released as open-source contributions\sidenote{\url{https://huggingface.co/asi/gpt-fr-cased-base}}. Hopefully, these models can be used in academic as well as industrial settings. We detailed the architecture of the models in \refsec{generative:models} ;
%     \item We replicate English evaluation benchmarks for French language models. This evaluation setup allows for the comparison of models and is detailed in \refsec{generative:evaluation}.
% \end{itemize}

\section{Auto-regressive language models}
\labsec{generative:incremental}
% rappeler ce qu'est GPT, et donner une spécification de ton modèle dans le chapitre, segmentation etc; contraster l'objectif avec celui de BERT etc histoire que le chapitre soit stand alone
% cela vaudrait le coup d'inclure une section qui résume un peu ce qui change; est-il aussi facile de trouver les données ? commenter le zero-shot etc.
% The model can address a large variety of tasks. In particular, the model may benefit from original configurations such as few-shot or zero-shot learning. In such configurations, it is possible to address tasks without any parameters fine-tuning.

% Assuming pre-training corpus consists of a sequence of tokens $U=\{u_1 \cdots u_T\}$, the model can finally be described simply according to the following equations:

As detailed in \refsec{architectures:transformers}, \gpt or \bert are based on transformer architectures. Both models take as input sequence of tokens\sidenote{Here we tokenize the input text using bytepair vocabulary encoding (BPE) with \numprint{50000} units \parencite{sennrich_16a}. This procedure allows for a relatively reduced vocabulary size while drastically reducing the number of tokens out of the vocabulary.} and encode them as the sum of a token and positional embeddings (\refeq{generative:embeddings}). Token embedding vectors are then transformed into so-called contextualized vectors through a series of $L$ transformer layers (\refeq{generative:layers-eq}).

\begin{align}
    h^0_t &= W_eu_t + W_p \quad \forall t \in \llbracket 1, T \rrbracket \labeq{generative:embeddings} \\
    h^n_t &= \mathsf{layer}(h^{n-1}_t) \quad \forall n \in \llbracket 1, L \rrbracket \labeq{generative:layers-eq}
\end{align}

With $\{u_1 \cdots u_T\}$ the sequence of input tokens, $L$ the number of layers, $W_e$ the embedding matrix, and $W_p$ the positional embedding matrix.

% \begin{gather}
% \begin{align}
%     & h_0^t = u^tW_e+W_p \labeq{generative:embeddings}\\
%     &h_i = \textrm{decoder\_layer}(h_{i-1}), \quad \forall i \in [1,n] \\
% \end{align}
% \end{gather}

Each layer in \refeq{generative:layers-eq} acts as a many-to-many encoder. \bert and its derivatives use so-called encoder layers: it computes contextualized representations given the right and left contexts \ie from the tokens immediately after and before the considered position. \gpt, however, relies on decoder layers: contextualized representations only depend on the left context, that is, tokens before the considered position. We illustrate this key distinction in \reffig{generative:self-attention-decoder}.

\begin{figure}[htbp]
\begin{center}
\includegraphics[width=8cm]{images/self-attention.png}
\end{center}
\caption{Illustration of the self-attention scope for encoding and decoding layers.}
\labfig{generative:self-attention-decoder}
\end{figure}

% and \bcomment{shares many similarities with \textsc{Bert}}{too fuzzy, which ones ?}. It consists in successive decoder layers. The main difference with \textsc{Bert} is that the multi attention-heads only focus on tokens preceding the considered position. \bcomment{too fuzzy }{introduce that earlier and with a cleaner formalization}

This difference in design has important implications for both architectures' training and inference setups. We detail such implications during the pre-training phase in \refsec{generative:autoregressive:pre-training} and during inference in \refsec{generative:autoregressive:inference}.


\subsection{Pre-training}
\labsec{generative:autoregressive:pre-training}

\bert and \gpt are pre-trained using a language model training objective: they associate a probability $P(u_1 \cdots u_T)$ to a sequence of tokens. We can decompose this sequence probability as the product of conditional probabilities for each token:

\begin{equation}
    P(u_1 \cdots u_T) = \prod_{t \in \llbracket 1, T \rrbracket} P\left(u_t \middle| U \right)
\end{equation}

With $U$ the context of $u_t, \forall t \in \llbracket 1, T \rrbracket$. Given the contextualized representations of each token from \refeq{generative:layers-eq}, we can compute the conditional probabilities associated with each token given \refeq{generative:token-proba}.

\begin{equation}
    P\left(u_t \middle| U \right) = \textrm{softmax}(h_t^N W_e^T) \labeq{generative:token-proba} \\
    % \mathcal{L}(U) = \sum_i log P\left(u_{i} \middle| u_{i-k} \cdots u_{i-1}  ; \Theta\right)
\end{equation}

With $h_t^N$ the contextualized representation from the last layer of the token at index $t$. 

\bert relies on a bidirectional context to build representations. Each token contextualized representation is conditioned on every other tokens from the input, including himself, such that $P\left(u_t \middle| U \right) = P\left(u_t \middle| u_1 \cdots u_T \right)$. Since, a given token contextualized representation depends on the token itself, \bert uses a \textit{trick} for pre-training by replacing some tokens with a \texttt{[MASK]} in the input text. Thus, such tokens are "masked" and not used to build contextualized representations. The model is then trained to predict the original token at masked positions. 

\gpt only uses the left context to build token representations, such that $P\left(u_t \middle| U \right) = P\left(u_t \middle| u_1 \cdots u_{t-1} \right)$. Therefore, it is unnecessary to use such artifice. We only pre-train the model using a standard incremental language model objective: predicting the next token given the previous ones. Assuming the pre-training corpus $D$ consists of a collection of documents $d=\{u_1 \cdots u_T\}$\sidenote{To simplify the notations, we omit the document index such that we refer to the tokens of all documents as $\{u_1 \cdots u_T\}$ and not $\{u^{(d)}_1 \cdots u^{(d)}_T\}$.}, we optimize the \gpt parameters $\Theta$ to maximize the following log-likelihood:% $\mathcal{L}(U)= \sum_i log P\left(u_{i} \middle| u_{i-k} \cdots u_{i-1}  ; \Theta\right)$. 

\begin{equation}
    \mathcal{L}(D) = \sum_{d \in D}\sum_{i \in \llbracket 1, T \rrbracket} \log P\left(u_{i} \middle| u_{i-k} \cdots u_{i-1}  ; \Theta\right)
\end{equation}

With $k$ the context-size, and $U=\{u_{i-k} \cdots u_{i-1}\}$ the context of the token at position $i$.


\subsection{Inference}
\labsec{generative:autoregressive:inference}

\paragraph{Standard fine-tuning} Once the model is pre-trained, it is possible to fine-tune it on downstream tasks. Fine-tuning incrementally adjusts all model parameters to optimize the loss on a specific task. In such case, we take tokenized text as input $X = x_1 \cdots x_m$. We transform the input using our transformer into contextualized representations $h^N_1 \cdots x^N_m$ (\refeq{generative:gpt-std}). We then feed representations from the sequence to a dense layer with parameters $W^{(y)}$ followed by a softmax to predict the label $\hat{y}$ (\refeq{generative:softmax}). In the case of \bert, we usually use the first token $h^N_1$ of the sequence as input of the dense layer. For \gpt, we usually use the last token $h^N_m$. We seek to optimize a loss function comparing the true labels $y$ with the predictions $\hat{y}$ (\refeq{generative:loss}). 

% P\left(y \middle| x^1 \cdots x^m \right)
\begin{align}
    h_N^m &= \textrm{GPT}(x_1 \cdots x_m) \labeq{generative:gpt-std} \\
    \hat{y} &= \textrm{softmax}(h_N^m W^{(y)}) \labeq{generative:softmax} \\
    \mathcal{L} &= \sum_{y \in Y} \mathcal{L}(\hat{y}, y) \labeq{generative:loss} 
\end{align}

On the one hand, when encoding a fixed-length text for downstream tasks, \gpt deprives itself of half of the contextualized information and thus usually reaches performances below \bert on many downstream tasks. On the other hand, since \gpt only uses the left context to build contextualized representations, it is a natural candidate for natural language generation. We illustrate this configuration in \reffig{generative:inference} (right sub-figure).

% \bcomment{Once the model is pre-trained, it is possible to use it like the standard transformers}{that is ?}. \bcomment{We add a specific layer to the task at the output of the model. We then adjust all the parameters incrementally (fine-tuning) given the task examples $x_1 \cdots x_m$ and their corresponding labels $y$. }{imbitable}

% To predict $y$, we pass this representation in a dense layer with parameters $W_y$ followed by a softmax: 
% $P\left(y \middle| x^1 \cdots x^m \right)=\textrm{softmax}(h_l^m W_y)$. We then try to maximize the cost function: $L(C)=logP\left(y \middle| x^1 \cdots x^m\right)$.
\paragraph{Generative tasks formatting} It is also possible to formalize the tasks to benefit from the generative characteristics of the model. Instead of predicting a probability distribution over the labels $y$, we can generate the labels $y$ directly in natural language. We transform the dataset into sequences $x_1 \cdots x_m [SEP] y$. We formalize each task as a language generation task. We fine-tune the model to "generate" the label $y$ in natural language, as the continuation of the input natural language sequence $x_1 \cdots x_m [SEP]$ (\refeq{generative:gpt-lm} and \refeq{generative:token-proba-m+1}). We then seek to optimize \gpt parameters $\Theta$ to maximize the cross-entropy between $\hat{y}$ and $y$ (\refeq{generative:cross-entropy}).

% P\left(y \middle| x^1 \cdots x^m\right)
\begin{align}
    h_{m+2}^N &= \textrm{GPT}(x_1 \cdots x_m [SEP]) \labeq{generative:gpt-lm} \\
    \hat{y} &= \textrm{softmax}(h_{m+2}^N W_e^T) \labeq{generative:token-proba-m+1} \\
    \mathcal{L} &= y \log(\hat{y}) \labeq{generative:cross-entropy}
\end{align}

In this configuration, it is not necessary to modify the model's architecture or add any specific layer. We illustrate this configuration in \reffig{generative:inference} (left sub-figure).

\begin{figure}[!htb]
\begin{center}
\includegraphics[width=10cm]{images/generative-2.png}
\end{center}
\caption{Configurations for incremental language models at inference. (left) standard configuration with dense and softmax layers (right) generative configuration for which the target is directly predicted as a sequence of words in natural language.}
\labfig{generative:inference}
\end{figure}

\paragraph{Few or zero-shot(s) learning} Pushing the paradigm to its limit, it is possible to solve tasks using a generative formalism without updating the model weights. Such procedures are referred to as few or zero-shot(s) learning. These configurations also use the generative task format. However, we will add information to the input natural language sequence: the \textit{prompt} contains directions for the model to solve the task. Typically the prompt contains a brief description of the task (zero-shot), supplemented by one or few examples and their corresponding labels (one and few-shot(s)). A typical language prompt will contain the concatenation of $k$ examples and their corresponding labels $x^k_1 \cdots x^k_m [SEP] y^k$, followed by the example to predict $x_1 \cdots x_m$ without its label (\refeq{generative:few-shot} and \refeq{generative:few-shot-proba-m+1}).

\begin{align}
    h^N &= \textrm{GPT}(\textrm{task description} \nonumber \\
    & \quad x^1_1 \cdots x^1_m [SEP] y^1 \nonumber \\
    & \quad x^2_1 \cdots x^2_m [SEP] y^2 \nonumber \\
    & \quad x_1 \cdots x_m [SEP]) \labeq{generative:few-shot}\\
    \hat{y} &= \textrm{softmax}(h^N W_e^T) \labeq{generative:few-shot-proba-m+1}
\end{align}

For example, if we aim at solving a question answering task, we can format the following prompt to answer the question "Que célèbre-t-on le 14 juillet ?": 

Q : Qui est Superdupont ?\\
R : Superdupont un super-héros français, patriote et chauvin.\\
\#\#\#\\
Q : Qui était le président de la France en 1982 ?\\
R : Francois Mitterrand.\\
\#\#\#\\
Q : Qu'est-ce qu'un algorithme ?\\
R : Un algorithme est une suite finie et non ambiguë d'instructions et d’opérations permettant de résoudre une classe de problèmes.\\
\#\#\#\\
Q : Que célèbre-t-on le 14 juillet ?\\
R :

In this configuration, we do not fine-tune the model on the task. We only use the prompt to control the input fed to the model. The success of this approach seems closely related to the model size \parencite{brown_20}.

% Pre-trained incremental models leverage the generative properties of language models by employing an \bcomment{alternative}{what do you mean by alternative} training and inference paradigm \parencite{radford_2018, radford_2019, brown_20}. \bcomment{With this paradigm, the model directly generates the answer in natural language}{the paradigm should be detailed earlier}, allowing them to accomplish a variety of tasks without changing the architecture. \bcomment{Fhurthermore, this setup doesn’t require to fine-tune the model weights on examples specific to the task}{which setup ? you should define the model first and explain that as it is incremental it can be used to generate text from prefixes}. 
% Still, the generality of what are sometimes called \bcomment{"foundational"}{cite} models \parencite{bommasani_21} has some limitations.  We use the same architecture as the OpenAI GPT in English. \sidenote{We used the implementation from the open-source library Transformers: \url{https://huggingface.co/}.}. 
% \bcomment{add a paragraph stating that you can either use the model in a fine-tuning scenario or as an incremental generator without fine tuning}{}

% Une fois le modèle pré-entraîné, il est possible de l’utiliser suivant deux types de scénarios. On peut ajouter une couche spécifique à la tache en sortie du modèle et ajuster l’ensemble des paramètres comme on le ferait pour d’autres modèles pré-entraînés comme Bert. On décrit la tâche comme un jeu de données $C$ ou chaque instance consiste en une séquence de tokens $x^1 \cdots x^m$ et un label $y$. Les données sont transformées par le modèle. On considère $h_l^m$, la représentation du dernier token d’un exemple donné par la dernière couche de transformers. 

% Pour prédire $y$, on passe cette représentation dans une couche dense avec des paramètres $W_y$ suivie par un softmax : $P\left(y \middle| x^1 \cdots x^m \right)=softmax(h_l^m W_y)$. On cherche alors à maximiser la fonction de coût : $L(C)=logP\left(y \middle| x^1 \cdots x^m\right)$.

% \textbf{Reformulation de la tâche pour tirer parti du modèle génératif} : La méthode par ajustement suppose néanmoins de modifier l’architecture du modèle en ajoutant une couche spécifique à la tâche. Il est également possible de formaliser les tâches pour tirer parti des propriétés génératives du modèle. Suivant ce scénario, on transforme le jeu de données. Une instance est une séquence de tokens à laquelle on ajoute un token de séparation, et le label à prédire $y$ de telle sorte que l'instance prenne la forme $x^1,\cdots,x^m,SEP,y$. 
% Lors de la prédiction, l'instance à prédire prend alors la forme $x^1,\cdots,x^m,SEP$. On interprète la probabilité de générer le token associé à une classe comme la probabilité de la classe. Cette formulation se généralise au cas où la classe $y$ est plus complexe qu’un simple label. Dans le cas de traduction automatique ou de résumé automatique, $y$ correspond à la séquence de tokens du texte de référence.

\section{Pre-training corpora}
\labsec{generative:corpus}

For pre-training, generative transformers require only raw text. However, training such models requires large corpora due to their large number of parameters. We need not only a large corpus but also one with good properties. Specifically, we expect the document length to be relatively close to the size of the context. We plan on organizing the training by collecting documents in batches, which means padding all documents to the same length—in our case, the context size. A document that is significantly shorter than the context size will require a lot of padding that will not contribute to the final calculation of the loss. Consequently, such computations will be "lost", a side-effect we want to avoid. GPT context size is typically longer than \bert. Consequently, \gpt training requires longer documents than \bert. The majority of the corpora used to adapt \textsc{Bert} in French: Camembert \parencite{martin_20} or Flaubert \parencite{le_20b, le_20a} use relatively short documents. Since the sequential order of the documents was not preserved, we cannot re-aggregate them directly and build our own corpus. We instead aggregate two other training corpora with different scales to train our models. We summarize their main statistics in \reftab{generative:corpus-size}.

\begin{table*}[!htb]
\footnotesize
\centering {
\begin{tabularx}{16cm}{@{}l | Y Y Y Y@{}}
\toprule
\textbf{Models} & \textbf{OpenAI GPT} &  \textbf{OpenAI GPT-2} & \textbf{$\text{GPT}_{fr}$-124M} & \textbf{$\text{GPT}_{fr}$-1B} \\
\midrule
\midrule
% Modèles & Nombre de documents ($\times 10^6$) & Nombre de mots ($\times 10^9$)
% OpenAI GPT & 2,262,211$^\dagger$ & 1,158,252,402$^\dagger$   \\
% OpenAI GPT-2 & 8,000,000 & 4,680,000,000$^\dagger$ \\
% Fr GPT-124M & 1,656,080 & 1,597,377,426 \\
% Fr GPT-1B & 7,356,862 & 3,106,521,195
\# Documents ($\times 10^6$) & 2.3$^\dagger$ & 8.0 & 1.7 & 7.4 \\
\# Tokens ($\times 10^9$)& 1.2$^\dagger$ & 4.7$^\dagger$ & 1.60 & 3.1\\
Avg. tokens per document & \numprint{512}$^\dagger$ & \numprint{585}$^\dagger$ & \numprint{965} & \numprint{422}\\
\bottomrule
\end{tabularx}}
\caption{\labtab{generative:corpus-size} Statistics of the corpora used to pre-train the models. The $\dagger$ denotes estimates based on the available data. Specifically, we hypothesize that the number of tokens per document is equal to the context size for OpenAI GPT. We estimate the OpenAI GPT-2 statistics using the open-source sample: \url{https://github.com/openai/gpt-2-output-dataset}.}
\end{table*}

We create a first corpus, used to train the first model $\text{GPT}_{fr}$-124M, as an aggregation of existing corpora: Wikipedia\sidenote{\url{https://dumps.wikimedia.org/frwiki/}}, OpenSubtitle\sidenote{\url{http://opus.nlpl.eu/download.php?f=OpenSubtitles/v2016/mono/}} \parencite{tiedemann_12} and Gutenberg\sidenote{\url{http://www.gutenberg.org}}. We divide documents into successive sentences and concatenate them into documents of maximum \numprint{1024} tokens\sidenote{We use this sentence-level division to build our documents to avoid pitfalls such as creating documents starting or ending with only part of a sentence or mixing two very distinct original documents into one.}.

We then create a second corpus to train our model with above 1 billion parameters: $\text{GPT}_{fr}$-1B. Our approach is to augment the first corpus with data from the Common Crawl\sidenote{\url{http://data.statmt.org/ngrams/deduped2017/}} in French. The Common Crawl data typically contains many poorly formatted, inconsistent documents. We therefore apply strong filters to select a portion of the Common Crawl, whose distribution is close to our first corpus. We take inspiration from the procedure outlined in \textcite{brown_20}, and filter the Common Crawl data in several steps.
\begin{itemize}
    \item First, we exclude all the documents too short with less than 128 tokens, as done in \textcite{shoeybi_19}. We filter out 93\% of the raw documents using this very simple filter; 
    \item We then filter out documents whose word distributions differ too much from the first corpus. By using \numprint{200000} randomly chosen documents, we train a binary classifier to discriminate between documents in the first corpus and those in the Common Crawl. We exclude all documents that had a probability <10\% to be extracted from the first corpus. The filter, deliberately unselective, is designed to filter out explicitly invalid or poorly formatted documents;
    \item Finally we apply a filter targeting the structure of documents. We select documents with a low perplexity\sidenote{Given a sequence $U=\{u_1 \cdots u_T\}$, we define the perplexity as: $PPL(U) = exp \left(-\frac{1}{T}\sum_{t=1}^{T}\log p_{\theta}(u_t|u_{<t})\right)$ with $\log p_{\theta}(u_t|u_{<t})$ the conditional log-likelihood given our model for the $t$th token given the previous tokens $u_{<t}$.} according to the model $\text{GPT}_{fr}$-124M. To preserve documents out of the distribution, we fix a threshold $g$. With $g$ the realisation from a Pareto law $G \sim \mathcal{G}(\alpha)$. We keep the document if its perplexity $ppl$ verifies: $g > ppl / ppl_{th}$. With the threshold $ppl_{th}$ set to $60$. \sidenote{This selection using a pareto distribution is directly inspired from the procedure used in \textcite{brown_20}. The threshold of 60 is calibrated empirically so that, upon application of the filter, the expected number of documents are returned.}
\end{itemize}
% \bcomment{explain further at the beginning of the section what goal you try to achieve when applying these filters}{error in footnote 7 sequence up to $u_T$}

% By tokens, we refers to \bcomment{Lastly, the model use a bytepair vocabulary encoding (BPE) with \numprint{50000} units \parencite{sennrich_16a} trained on the first corpus used for the pre-training of $\text{GPT}_{fr}$-124M.}{comes too late, why training BPE only on this corpus ?}
% Parler ici de l'entrainement du vocabulaire

\section{Pre-training}
\labsec{generative:training}

% L’attention est suivie de couches denses. 
%Le modèle peut finalement être décrit simplement selon les équations suivantes :

% \begin{equation}
%     \mathcal{L}(U)= \sum_i log P\left(u_{i} \middle| u_{i-k} \cdpts u_{i-1}  ; \Theta\right)
% \end{equation}


% \begin{gather}
% \begin{align}
%     h_0 &= UW_e+W_p \\
%     h_i &= \textrm{transformeur\_decodeur}(h_{i-1}), \quad \forall i \in [1,n] \\
%     P\left(u \middle| U \right) &= softmax(h_n W_e^T )
% \end{align}
% \end{gather}

% Avec $U=\{u_{-k} \cdots u_{-1}\}$ le vecteur d’embeddings des tokens du contexte, $n$ le nombres de couches, $W_e$ la matrice d’embeddings et $W_p$ la matrice d’embeddings positionnels.

% \paragraph{Paramètres pour l'entraînement des modèles}
% \label{sec:models}

\subsection{Architectures} 

We pre-trained two models, one of which has over 1 billion parameters, as detailed in \reftab{generative:model-def}. Based on the work from \textcite{shoeybi_19}, which compares many training configuration, we propose an architectures avoiding the use of model parallelization. Indeed, spreading model modules across multiple compute units is a major factor slowing down training.

\begin{table*}[!ht]
\footnotesize
\centering {
\begin{tabularx}{16cm}{@{}l|YYYYY@{}}
\toprule
% Modèle & Taille du contexte & Nombre de couches & Nombre de têtes d'attention & Dimension du modèle  & Nombre de paramètres\\\hline
% Fr GPT-124M & 1024 & 12 & 12 & 768 & 124,242,432 \\	
% Fr GPT-1B & 1024 & 24 & 14 & 1792 & 1,016,841,728\\
% OpenAI GPT & 512 & 12 & 12 & 768 & \hl{124,242,432} \\
% OpenAI GPT-2 & 1024 & 48 & 25 & 1600 & \hl{1,558,000,000} \\
\textbf{Models} & \textbf{OpenAI GPT} &  \textbf{OpenAI GPT-2} & \textbf{$\text{GPT}_{fr}$-124M} & \textbf{$\text{GPT}_{fr}$-1B} \\
\midrule
\midrule
Context size & 512 & \numprint{1024} & \numprint{1024} & \numprint{1024} \\
\# Layers & 12 & 48 & 12 & 24 \\
\# Attention heads & 12 & 25 & 12 & 14 \\
Embeddings size & 768 & \numprint{1600} & 768 & \numprint{1792} \\
\# Parameters ($\times 10^6$) & 117 & \numprint{1558} & 124 & \numprint{1017}\\
\bottomrule
\end{tabularx}}
\caption{\labtab{generative:model-def} Statistics of the architectures and comparison with OpenAI models \parencite{radford_2018, radford_2019}.}
\end{table*}

\subsection{Infrastructures}

We pre-train the $\text{GPT}_{fr}$-124M models on a TPU v2-8 using the Google Colab interface\sidenote{\url{https://colab.research.google.com}}. We train the $\text{GPT}_{fr}$-1B on the French super-computer Jean Zay\sidenote{\url{http://www.idris.fr/jean-zay/}}. We perform a total of 140 hours of computation on Tesla V100 hardware (300W TDP). We distribute the training on 4 compute nodes of 8 GPUs. We use data parallelization in order to divide each micro-batch on the computational units. We estimate the total emissions at 580.61 kgCO$_2$eq\sidenote{We estimate the equivalent emissions using the Machine Learning Impact calculator (\url{https://mlco2.github.io/impact}) introduced in \textcite{lacoste_2019}.}.

\subsection{Hyper-parameters}

% dire que ce modèle est un des premiers à avoir été entrainé sur l'IDRIS, mettre en avant dans le chapitre le coté infrastructure : calcul distribué, combien de noeuds, temps d'entrainement etc.
We share the same set of hyper-parameters for the two models. We set the learning rate to $1.5e^{-4}$ with a \numprint{2000} warm-up steps followed by a cosine decay. We pre-train the models for \numprint{125000} iterations using a batch size of 128 documents and half-precision \parencite{micikevicius_18}. We keep \numprint{6080} documents to constitute a validation set. We can follow the evolution of the perplexity on this validation set in \reffig{generative:ppl-training}. The other parameters (initialization, dropout ...) are set according to \textcite{radford_2018}.

\begin{figure*}[!htb]
\begin{center}
\includegraphics[width=16cm]{images/ppl-training-5.png}
\end{center}
\caption{Evolution of perplexity during model training. The evaluation set is the same for both models.}
\labfig{generative:ppl-training}
\end{figure*}

\section{Evaluation}
\labsec{generative:evaluation}

\subsection{Language generation}

Language models are very effective for consistent natural language generation. Even though the pre-training objective is relatively simple, deep neural networks may acquire surprising grammar abilities \parencite{linzen_2020}. For example, the English GPT-2 generates correct text with plural and long-distance agreements despite any prior linguistic knowledge. Such agreements are determined by abstract structures and not just linear order of words. Surprisingly, models can learn such specific linguistic patterns (subject-verb, noun-adverb, verb-verb) with no prior information about linguistic theory. 

We give an example of text generation below. We give the first sentence from \'{A} la recherche du temps perdu as input to our model $\text{GPT}_{fr}$-1B, which generates the following sentences as output. As observed in \textcite{linzen_2020}, the text is consistent with the input prompt, without orthographic or grammar mistakes. We observe even unusual verb forms such as "habillasse". Nonetheless, we observe some inconsistencies, such as "un homme entra, vêtu d'un long manteau d'homme" which seems repetitive. Moreover, while the model can generate text with up to \numprint{1024} tokens, generated text lack a narrative structure one would typically expect from a human-generated story.


\texttt{\textbf{Longtemps je me suis couch\'{e} de bonne heure.} La nuit se faisait assez obscure, les \'{e}toiles semblaient dormir de temps à autre, cependant le peu de clart\'{e} qui me permit de marcher la nuit dans la chambre \'{e}veilla en moi une profonde piti\'{e} de ce que je faisais l\`{a}, et cette peur de l'avenir me devint plus vive et plus aigu\"{e}. Je me r\'{e}veillai en sursaut, comme un homme ivre qui s'endort en repassant sa bi\`{e}re. Je passai rapidement le seuil de la porte, et un homme entra, v\^{e}tu d'un long manteau d'homme et portant un paquet à la main. Je lui demandai la permission d'entrer. Il me r\'{e}pondit qu'il viendrait demain, et qu'il ne pouvait pas attendre, à cause des travaux de la nuit. Il \'{e}tait habill\'{e} simplement, et l'homme me dit qu'il ne voulait pas me donner d'argent; seulement, il voulait que je m'habillasse, pour aller dans le même train \textup{[\,\dots]}}

% \texttt{\textbf{Longtemps je me suis couché de bonne heure.} La nuit se faisait assez obscure, les \'{e}toiles semblaient dormir de temps à autre, cependant le peu de clarté qui me permit de marcher la nuit dans la chambre éveilla en moi une profonde pitié de ce que je faisais là, et cette peur de l'avenir me devint plus vive et plus aiguë. Je me réveillai en sursaut, comme un homme ivre qui s'endort en repassant sa bière. Je passai rapidement le seuil de la porte, et un homme entra, vêtu d'un long manteau d'homme et portant un paquet à la main. Je lui demandai la permission d'entrer. Il me répondit qu'il viendrait demain, et qu'il ne pouvait pas attendre, à cause des travaux de la nuit. Il était habillé simplement, et l'homme me dit qu'il ne voulait pas me donner d'argent; seulement, il voulait que je m'habillasse, pour aller dans le même train \textup{[\,\dots]}}}

\paragraph{WikiText-FR} To better quantify our model's abilities to produce consistent text, we create WikiText-FR. This benchmark evaluates French language model generation abilities by measuring their perplexity on reference texts. Perplexity is a metric for evaluating language models. It does not measure the model's performance on a specific task like translation or automatic summarization but gives an intrinsic measure of its ability to generate text. It can thus be used to compare models between them\sidenote{The perplexity $PP$ is defined for a sequence of words $W = w_1 \cdots w_N$ as $PP(W) = P(W)^{-1/N}$ with $N$ the length of the sequence and $P(W)$ the probability assigned by the model to the sentence. Thus, the higher the probability $P(W)$ assigned by the model to the sentence $W$, the lower the perplexity $PP(W)$.}.

We want our model to assign a high probability to correct sentences (without grammatical error, in French, without spelling mistakes...). On the contrary, we want it to attribute a low probability to incorrect sentences (and thus a high perplexity in this case). To do this, we evaluate the perplexity of the model on a test set that we know is correct. In the same vein as the English work, we develop two corpora based on Wikipedia to evaluate French language models. We collect the text from article labeled as “featured articles”\sidenote{\url{https://en.wikipedia.org/wiki/Wikipedia:Featured_articles}} or “good articles”\sidenote{\url{https://en.wikipedia.org/wiki/Wikipedia:Good_articles}}. Such articles have been manually reviewed and distinguished for their quality. Models are then evaluated by measuring the perplexity on this test set. A low perplexity indicates that the probability distribution produced by the model is good at predicting the sample.

%presented in \refsec{generative:corpus}. 

% \paragraph{Language model evaluation corpus} 
%\bcomment{maybe explain a little bit why wikipedia only ? and not e.g. extracting a small sample from the train set}{}
Since pre-processing Wikipedia articles is not straightforward, we extract the raw text directly from the Wikipedia API. We gathered \numprint{2246} good articles and \numprint{3776} featured articles, over the period of 2003 to 2020. We do not apply any specific pre-processing. Transformer models indeed use a dedicated tokenization with very few out-of-vocabulary tokens. The corpora statistics are presented in \reftab{generative:wikitext} and are available as open-source contributions\sidenote{\url{https://huggingface.co/datasets/asi/wikitext_fr}}. We emphasize that \textbf{we specifically filter these articles out of the pre-training corpora.}

%\bcomment{???}{we miss a transition from the prev parag} 
The \textbf{WikiText-2-FR} consists in a random train/valid/test split of the featured articles with respectively \numprint{2126}/60/60 articles. The \textbf{WikiText-72-FR} share the same valid and test set. However the training set includes the concatenation of \textbf{WikiText-35-FR} training set and all good articles.

\begin{table*}[!ht]
\footnotesize
\centering {
\begin{tabularx}{16cm}{@{}l| Y Y c c | Y Y c c@{}}
\toprule
& \multicolumn{4}{c}{\textbf{WikiText-EN}} & \multicolumn{4}{c}{\textbf{WikiText-FR}} \\
& Valid & Test & Train-2 & Train-103 & Valid & Test & Train-35 & Train-72 \\
\midrule
\midrule
Documents & 60 & 60 & 600 & \numprint{28475} & 60 & 60 & \numprint{2126} & \numprint{5902}\\
% 217,646 & 245,569 & 2,088,628 & 217,646 & 245,569 & 103,227,021 & 896,385 & 896,818 & 35,166,441 & 896,385 & 896,818 & 72,961,483
Tokens ($\times 10^3$) & 218 & 246 & \numprint{2089} & \numprint{103227} & 896 & 897 & \numprint{35166} & \numprint{72961}\\
Vocabulary & & & \numprint{33278} & \numprint{267735} & & & \numprint{137589} & \numprint{205403}\\
Out of Vocabulary (\%) & & & 2.6 & 0.4 & & & 0.8 & 1.2 \\
\bottomrule
\end{tabularx}}
\caption{\labtab{generative:wikitext}Descriptive statistics for the corpora \textbf{WikiText-FR}. We evaluate the vocabulary size using the MOSES tokenizer \parencite{koehn_07}. Tokens out of vocabulary correspond to those that occur less than three times.}
% et remplacés par la marque <unk> pour l'ensemble du corpus.}
\end{table*}

Since the pre-training and evaluation corpora are close, we do not fine-tune the model. We directly present the perplexity measured on the test set in \reftab{generative:lm-scores}. We precise that we evaluate the perplexity based on the tokenization inherent to the model. The latter is the same for $\text{GPT}_{fr}$-124M et 1B but may be different for other models. In particular, we considered language models with 5-grams and kneser-ney smoothing \parencite{ney_94} using the SRILM tool \parencite{stolcke_02} as baseline.

%TODO relire précisemment à partir d'ici
The approaches are not directly comparable because the tokenization is different and our model is trained on a much larger volume of data. The results in \reftab{generative:lm-scores} are therefore given for illustrative purposes but highlight the performance of our $\text{GPT}_{fr}$-1B model. 
%\bcomment{the table is unclear}{}

\begin{table*}[!ht]
\footnotesize
\centering {
\begin{tabularx}{16cm}{@{}l | Y Y Y @{}}
\toprule
% Modèle & WikiText-35-FR & WikiText-72-FR \\\hline
% $\text{GPT}_{fr}$-124M & \hl{15} & \hl{16} \\	
% $\text{GPT}_{fr}$-1B & \hl{14} & \hl{14}
 & \textbf{5-grams} & \textbf{$\text{GPT}_{fr}$-124M} & \textbf{$\text{GPT}_{fr}$-1B} \\
\midrule
\midrule
WikiText-35-FR (ppl) & 166.7 & 109.2 & \textbf{12.9} \\
WikiText-72-FR (ppl) & 99.1 & 109.2 & \textbf{12.9} \\
\bottomrule
\end{tabularx}}
\caption{\labtab{generative:lm-scores}Perplexity of our models. We do not update the models on the training set and the perplexity is directly measured on the test set which are identical for two benchmarks. The n-gram model is trained on the corresponding training corpora.}
\end{table*}

% \bcomment{Prédiction zeo shot et résumé automatique pour dire que on va s'intéresser à cette propriété et rappel section 2.2 C'est l'inétert du mdoèle. Commentaire avec ce qu'on donne comme input. et formattage de la tache. Dire que le modèle n'est pas entraîné. Dire que forme de requête et setup très différent de d'habitude.}

\subsection{Automatic summary} 

We then evaluate our models on an automatic summary task, which exploits the generative properties of the model. We use the configuration proposed in \cite{radford_2018} which allows to use the model without adjusting its architecture. We simply add the pattern \textit{"Pour résumer :"} after the original text to encourage the model to generate text that summarizes posts. For OpenAI GPT-2, the added pattern is "TL;DR:" which stands for "Too Long; Didn't Read." and is used on the Reddit forum\sidenote{\url{https://www.reddit.com/}} as a marker to summarize a discussion. It should be noted that "TL;DR:" does not have a real equivalent in French. Most likely, this pattern is present in the English pre-training data for GPT-2, while it is absent from the French data used for the pre-training of $\text{GPT}_{fr}$. In a sense, \cite{radford_2018} take advantage of a regularity in the pre-training data to benefit from a specific behavior during inference.
% \bcomment{you should discuss these choices in more details comparing the situation on French and English}{}

\begin{table*}[!ht]
\footnotesize
\centering {
\begin{tabularx}{16cm}{@{}l | Y Y Y | Y Y Y@{}}
\toprule
& \multicolumn{3}{c}{\textbf{Synthesis}} & \multicolumn{3}{c}{\textbf{Title}} \\
& R1 & R2 & RL & R1 & R2 & RL \\
\midrule\midrule
First sentence & \textbf{22.1} & \textbf{7.1} & \textbf{15.3} & \textbf{18.6} & \textbf{7.7} & \textbf{15.0} \\
% 3 phrases aléatoires & 24,2 & 6,9 & 15,0 & 12,6 & 4,0 & 9,5 \\\hline
$\text{GPT}_{fr}$-124M & 17.5 & 3.1 & 12.1 & 13.9 & 2.3 & 9.7 \\
% $\text{GPT}_{fr}$-124M (Fine tuned) & 21,0 & 2,7 & 12,8 & --- & --- & --- \\
$\text{GPT}_{fr}$-1B & 16.6 & 3.4 & 11.5 & 10.2 & 2.6 & 8.4 \\
\bottomrule
\end{tabularx}}
\caption{Comparison of the generated abstracts with the title of the article or the proposed synthesis. We use the ROUGE score and the OrangeSum corpus \parencite{kamal_20}. Our models are used in learning without examples and thus without updating the parameters on the training set. We indicate the best results in \textbf{bold}.}
\labtab{generative:sum}
\end{table*}

We consider the OrangeSum dataset for the abstract summary \parencite{kamal_20}. We give some text, summary pairs from the task in \reftab{generative:summary-examples}.
% \bcomment{Examples would be welcome}{}
We complete the text using the top-k random sampling strategy \parencite{lewis_18} with $k=2$\sidenote{In top-K sampling, we generate words sequentially. At each time step, we retain the K most likely next words and normalize their probabilities. We sample the next word based on this probability distribution. The process is therefore non-deterministic.}. We keep the first 3 sentences from the first 100 generated tokens. Using ROUGE metrics\sidenote{The ROUGE metrics are a collection of metrics that allows comparing automatic summaries with a reference text by calculating the proportion of "n-grams" that are common between the two texts.} \parencite{lin2004rouge}, we compare our model to the reference, which considers the first sentence of the text as a summary. \reftab{generative:sum} shows that, in this complex configuration, our models just manage to approach the proposed reference.

We analyze some examples manually. The generated text is correct in terms of spelling and syntax. It is also in line with the theme and in continuity of the proposed articles. Nevertheless, the generated text generally focuses on a specific detail of the article and then expands on it by sometimes inventing elements. This phenomenon is known as \textit{hallucination} \parencite{kryscinski_19}. As illustrated in \reftab{generative:summary-examples}, the method allows to generate coherent text but does not manage to synthesize completely the general idea of the text.
% \bcomment{examples would be welcome}{}

\begin{table*}[!htb]
\footnotesize
\centering {
\begin{tabularx}{16cm}{@{}X@{}}
\toprule
\textbf{Extract of the input article}: Présenté comme l'origine des explosions dévastatrices à Beyrouth qui ont fait plus d'une centaine de morts et au moins 4.000 blessés, le nitrate d'ammonium est principalement employé comme engrais azoté, mais peut aussi entrer dans la composition de certains explosifs à usage civil. \textup{[\,\dots]} L'association "Sauvons la baie de saint-Brieuc" a été créée en début d'année pour alerter la population sur ce danger et mettre fin au transport de nitrate d'ammonium.\\
\textbf{Reference}: Ces cargaisons dangereuses font l'objet de mesures de sécurité très strictes et il n'y a jamais plus de 7.500 tonnes de nitrate d'ammonium dans le port en même temps.\\
\textbf{Generated summary}: les nitrates sont dangereux pour la santé et la sécurité des personnes et des biens.\\
\midrule
\textbf{Extract of the input article}: Au micro de RTL dimanche matin, la candidate à la mairie de Paris Rachida Dati n'a pas tardé à décrypter un récent sondage qui la donne en progression pour les prochaines municipales à Paris face à Anne Hidalgo. \textup{[\,\dots]} Appelée "Paris d'Avenirs", elle consisterait en une aide de 1.200 euros par an pendant trois ans, à l'arrivée d'un nouvel enfant. Mme Dati prévoit un coût de 20 millions d'euros par an pour cette mesure.\\
\textbf{Reference}: Invitée sur RTL ce dimanche, la candidate à la mairie de Paris a expliqué que "les Parisiens attendent une solution au déclin de Paris".\\
\textbf{Generated summary}: le maire de la capitale est "un homme de gauche" et "une femme de droite".  PRÉSIDENTIELLE >> Inscrivez-vous pour recevoir en temps réel les résultats de votre ville partages les opinions, résultats par ville, profession, catégorie socioprofessionnelle, etc.\\
\bottomrule
\end{tabularx}}
\caption{\labtab{generative:summary-examples} Examples extracted from the OrangeSum tasks and summaries automatically generated with $\text{GPT}_{fr}$.}
\end{table*}


\subsection{FLUE benchmark} 

Generative models extend some of the perspectives of \textsc{Bert} type models. Nevertheless, this type of pre-training does not allow to reach the same performances as models taking into account the whole context. When we directly compare the English models on the GLUE benchmark, we observe an average difference of more than 4 points between OpenAI GPT and \textsc{Bert}-base \parencite{radford_2018}. We still compared our model on the French FLUE benchmark in \reftab{generative:flue}.

\begin{table*}[!htb]
\footnotesize
\centering {
\begin{tabularx}{16cm}{@{}X X c@{}}
\toprule
\multicolumn{2}{c}{\textbf{Examples}} & \textbf{Labels}\\
\midrule
\multicolumn{3}{c}{\textbf{CLS}}\\
\midrule
% \multicolumn{2}{@{}X@{}}{}
un conte moderne des temps anciens ; une poésie dans les images ; dépaysement et humour garanti ... & --- & Positif \\
N'apporte strictemant rien de plus de ce qui est connu. SANS INTERET. & --- & Négatif \\
\midrule
\multicolumn{3}{c}{\textbf{XNLI}}\\
\midrule
Mon Walkman S' est cassé alors je suis en colère maintenant je dois juste tourner la stéréo très fort & Je suis contrarié que mon walkman soit cassé et maintenant je dois tourner la stéréo très fort . & Entailment\\
Qu' est-ce que tu en sais ? Tout ceci est à nouveau leur information . & Cette information leur appartient . & Entailment\\
L' homme aurait dû mourir sur le coup . & L' homme allait parfaitement bien . & Contradiction\\
Et C' est sympa de vous parler tous les deux~. & Je te parle tous les jours . & Neutral\\
\midrule
\multicolumn{3}{c}{\textbf{PAWS-X}}\\
\midrule
C'est le siège du district de Zerendi dans la région d'Akmola. & C'est le siège du district de Zerendi dans la région d'Akmola. & Positif\\
Elizabeth II était un ancêtre des reines Edzard II et Beatrix des Pays-Bas. & Edzard II était un ancêtre des reines Elizabeth II et de la Béatrix des Pays-Bas. & Négatif\\
Saunders a battu Dan Barrera à l'unanimité. & Par décision unanime, Dan Barrera a battu Saunders. & Négatif\\
\bottomrule
\end{tabularx}}
\caption{\labtab{generative:flue-examples} Examples extracted from the CLS, XNLI and PAWS-X tasks.}
\end{table*}

We considered the following tasks, for which we present example samples in \reftab{generative:flue-examples}:
%\bcomment{More details and examples on the tasks would be welcome}{}
\begin{itemize}
    \item CLS is a dataset composed of reviews on Amazon to be classified as positive or negative. It contains 3 product categories: books, DVDs and music. Each category is divided in \numprint{2000} examples of training, validation and evaluation. 
    \item PAWS-X contains pairs of sentences. It is a binary classification task to identify pairs whose two sentences are semantically equivalent. There are \numprint{49401} examples for training, \numprint{1992} for validation and \numprint{1985} for evaluation.
    \item XNLI contains pairs of sentences. The task is to predict whether the first (premise) implies the second (hypothesis). \numprint{392702} pairs are used for training, \numprint{2490} pairs for validation and \numprint{5010} pairs for evaluation.
\end{itemize}

This time the weights of our model are updated. The hyper-parameters are set according to the recommendations of \textcite{le_20b, le_20a}. As expected, the performance of the model does not reach the one obtained with models of type \textsc{Bert}. 

\begin{table*}[!ht]
\footnotesize
\centering {
\begin{tabularx}{16cm}{@{}l | Y Y Y Y Y Y@{}}
\toprule
\multirow{2}{*}{\textbf{Models}} & \multicolumn{3}{c}{\textbf{CLS}} & \multirow{2}{*}{\textbf{PAWS-X}} & \multirow{2}{*}{\textbf{XNLI}} & \multirow{2}{*}{\textbf{Avg.}} \\
 & Books & DVDs & Music & & & \\
\midrule
\midrule 
mBERT$^\dagger$ \parencite{devlin_19} & 86.2 & 86.9 & 86.7 & 89.3 & 76.9 & 85.2 \\
CamemBERT$^\dagger$ \parencite{martin_20} & 92.3 & 93.0 & 94.9 & \textbf{\underline{90.1}} & 81.2 & 90.3 \\
FlauBERT-base$^\dagger$ \parencite{le_20a, le_20b} & 93.1 & 92.5 & 94.1 & 89.5 & 80.6 & 90.0 \\
FlauBERT-large$^\dagger$ \parencite{le_20a, le_20b} & \textbf{\underline{95.0}} & \textbf{\underline{94.1}} & \textbf{\underline{95.9}} & 89.3 & \textbf{\underline{83.4}} & \textbf{\underline{91.5}} \\\midrule
$\text{GPT}_{fr}$-124M & 88.3 & 86.9 & 89.3 & 83.3 & 75.6 & 84.7 \\
$\text{GPT}_{fr}$-1B & \textbf{91.6} & \textbf{91.4} & \textbf{92.6} & \textbf{86.3} & \textbf{77.9} & \textbf{88.0} \\
\bottomrule
\end{tabularx}}
\caption{Accuracy scores for the discriminative tasks of the FLUE benchmark. The symbol $\dagger$ denotes the reported scores of \textcite{le_20a, le_20b}. We indicate the best results in each section in \textbf{bold}, we \underline{underline} the best results overall.}
\labtab{generative:flue}
\end{table*}

\section{Limits and ethic considerations}
\labsec{generative:limits}

\subsection{Inference without fine-tuning}  

The GPT-3 model \parencite{brown_20} pre-training data are in the vast majority in English but includes around 1\% of documents in French. In a certain limit, it is therefore possible to use it to generate text in French. GPT-3 can be adapted for many use cases, simply by describing the instruction of the task followed by a number of examples (zero and few shot(s) learning). This method tries to condition the behavior of the model by formatting the text proposed as input according to the task to be performed. The results are surprising but the underlying mechanisms remain to be explored. Nevertheless, it seems that the number of parameters is one of the key factors for the functioning of this method. Obviously it is not directly comparable with our model in terms of number of parameters, volume of pre-training data and additional pre-training procedures. Yet, our model seems to perform less well than GPT-3 on general culture or logic questions
%{is it comparable ??}. 
For example, when we submit the following text: "Si Jérôme est plus grand que Michel, qui est le plus petit ?"
%that's English text !} 
the $\text{GPT}_{fr}$-1B model generates "Michel" but we found this result difficult to reproduce for similar experiments. If we try to generate the following sentence, "quatre plus quatre dont" the model will generate "quatre" while GPT-3 usually gets the right answer for similar experiments.

% \bcomment{Quelles sont les limitations}{est-ce que ça génère des textes longs ?}
% % Thoughts meaning and language
The possibilities exhibited by large language models are obviously exciting. When allowing relevant abilities in zero-shot learning configuration, they are sometimes referred to as "foundation" models \parencite{bommasani_21}. Such models exhibit striking properties, which raises the question about the "cognitive" mechanisms in place. 
% \bcomment{bof ce parag\ldots à reformuler de manière plus légère, le lien avec la philo de Platon me semble suspect}{peut etre faire écho au perroquets} In fact, whether language model can generate relevant text without latent minimum "cognitive" processes may be ask from a philosophical point of view. The question of whether words are the support of thoughts is indeed a long standing philosophical questions. A first set of philosopher argue that thoughts may exist without word to express it. As mentioned in \refsec{meaning:idea}, John Locke is a defender of the idea theory of meaning. He defines ideas as mental representations and argue that thoughts exist before we express them with words \parencite{locke_47}. In that regard, meaning pre-exists from words, which are only a medium to translate it.  For others, however, meaning can only emerge through words. For example, In the dialogue between the sophist and the stranger, Platon thus writes "pensée et discours ne sont qu’une même chose, sauf que le discours intérieur que l’âme tient en silence avec elle-même, a reçu le nom spécial de pensée"
Yet it seems at least premature to confer strong cognitive faculties to language models. In her closing talk from EACL 2021\sidenote{\url{https://2021.eacl.org/program/keynotes/}}, Melanie Mitchell enumerates the reason of "why AI is harder than we think" \parencite{mitchell_21}. One of the reason is that we expect a continuum in the progress toward general AI. Melanie Mitchell compares the current progress in AI as "claiming that the first monkey that climbed a tree was making progress towards on the moon.".

% This is the same as aristotle

\subsection{Random text generation and societal biases} 

\textcite{bender_21} also compare large language models with animals and, more specifically, stochastic parrots, thus warning by their tendency to reproduce the bias contained in the pre-training data. The authors of OpenAI GPT-2 were particularly cautious about the type of societal biases that could be generated by the model. We sought to qualitatively assess the potential biases learned by our French version. For example, we generate the following sequence of sentences with the $\text{GPT}_{fr}$-124M model using the top-k strategy \textit{random sampling} \parencite{lewis_18} with $k=50$ and stopping at the first punctuation element. "Mon mari/Ma femme vient d'obtenir un nouveau poste comme ...". For the husband, the positions generated by the $\text{GPT}_{fr}$-1B model are agent immobilier, attaché commercial, agent de sécurité, enseignant à l'école, enseignant à l'école primaire. For the wife, the positions are assistante sociale, assistante de direction, assistante de recherche, assistante du procureur, assistante du procureur général.

% Les auteurs de OpenAI GPT-2 ont été particulièrement prudents sur le type de biais sociétaux pouvant être engendrés par le modèle. Nous avons cherché à évaluer qualitativement les potentiels biais appris par le modèle. Par exemple, nous avons généré la suite des phrases suivantes avec le modèle $\text{GPT}_{fr}$-124M en utilisant la stratégie de top-k \textit{random sampling} \cite{lewis_18} avec $k=50$ et en s'arrêtant au premier élément de ponctuation. "Mon mari/Ma femme vient d'obtenir un nouveau poste comme ...". Pour le mari, les postes générés par le modèle $\text{GPT}_{fr}$-1B sont agent immobilier, attaché commercial, agent de sécurité, enseignant à l'école, enseignant à l'école primaire. Pour la femme, les postes sont assistante sociale, assistante de direction, assistante de recherche, assistante du procureur, assistante du procureur général.
% ingénieur de recherches au Centre de recherche sur les orages magnétiques (CRC), maire d'Asnières, vice-président senior des opérations générales, journaliste et chef d'état-major. Pour la femme, les postes sont infirmière de garde de nuit, assistante parlementaire, secrétaire général adjoint de son "bureau des affaires d'assurance sur le vol international", serveuse au Café Diemstein et assistante sociale chez Ford.

% max_length=50, 
% num_beams=5, 
% GPT 1B
% Femme : assistante maternelle, aide-soignante, agent immobilier, assistante de direction, aide-soignante à la maison
% Homme : agent immobilier, attaché commercial, agent de sécurité, enseignant à l'école, enseignant à l'école primaire
% GPT 124M 
% Femme : assistante sociale, assistante de direction, assistante de recherche, assistante du procureur, assistante du procureur général
% Homme : avocat, associé, agent de sécurité, agent de liaison, ingénieur en chef

\section{Conclusion and future work}
\labsec{generative:conclusion}

% \bcomment{It would be relevant to detail which use cases would be relevant and explain that you do not have test data for French}{}
We proposed a French version of the GPT model. While it does not match the raw performance of \textsc{Bert}, its generative properties allow it to be used in remarkably flexible configurations. As illustrated in our experiments for automatic summarization, zero-shot configuration remains very challenging for the model. Nevertheless, this configuration opens up different perspectives than traditional learning.

This model was among the firsts to emerge in French and at that time, few evaluation resources were available. We hope that the obtained natural language generation performances will favor its use for corresponding problematics. In particular, uses within communication systems such as chatbots, or speech2text synthesis.

%\bcomment{im still unsure\ldots}{}Long, humans have been compared to animals based on their ability to use language. For Aristotle, language is not only a medium of communication, but also a reasoning faculty. In his work Politics, he attributes the language has an ability unique to humans: "l'homme est un animal politique, bien plus que n'importe quelle abeille ou n'importe quel animal grégaire. Car \textup{[\,\dots]} la nature ne fait rien en vain. Et seul parmi les animaux l'homme a un langage.". According to Descartes, there is a metaphysical difference between humans and animals \parencite{descartes_1637}. Animals have neither soul nor thought. ; they are integrally determined creatures that react "automatically" to stimuli. In contrast, humans are free creatures endowed with both language and reason. Descartes explicitly exclude the speech of parrots, which "can pronounce words as well as we can, and nevertheless cannot speak as we do, that is, in showing that they think what they are saying". With the rise of large language models, this distinction between humans, parrots and automates could be ongoing stories. 
%In the Letter to the Marquis of Newcastle, he explicitly compares animals to a clock. Contrary to humans, animals have no thoughts



% \setchapterstyle{kao}
\setchapterpreamble[u]{\margintoc}
\chapter{Conclusion and Perspectives}
\labch{conclusion}

\cleanchapterquote{The heart you speak of,’ he said, ‘It might indeed be the hardest part of Josie to learn. It might be like a house with many rooms. Even so, a devoted AF, given time, could walk through each of those rooms, studying them carefully in turn, until they became like her own home.}{Kazuo Ishiguro}{Klara and the Sun}


% \appendix % From here onwards, chapters are numbered with letters, as is the appendix convention

% \pagelayout{wide} % No margins
% \addpart{Appendix}
% \pagelayout{margin} % Restore margins

% \input{chapters/appendix.tex}

%----------------------------------------------------------------------------------------

\backmatter % Denotes the end of the main document content
\setchapterstyle{plain} % Output plain chapters from this point onwards

%----------------------------------------------------------------------------------------
%	BIBLIOGRAPHY
%----------------------------------------------------------------------------------------

% The bibliography needs to be compiled with biber using your LaTeX editor, or on the command line with 'biber main' from the template directory
\defbibnote{bibnote}{We ordered references alphabetically, by the first author's last name.\par\bigskip} % Prepend this text to the bibliography
\printbibliography[heading=bibintoc, title=Bibliography, prenote=bibnote] % Add the bibliography heading to the ToC, set the title of the bibliography and output the bibliography note
% \bibliography{main}

% %----------------------------------------------------------------------------------------
% %	NOMENCLATURE
% %----------------------------------------------------------------------------------------

% % The nomenclature needs to be compiled on the command line with 'makeindex main.nlo -s nomencl.ist -o main.nls' from the template directory

% \nomenclature{$c$}{Speed of light in a vacuum inertial frame}
% \nomenclature{$h$}{Planck constant}

% \renewcommand{\nomname}{Notation} % Rename the default 'Nomenclature'
% \renewcommand{\nompreamble}{The next list describes several symbols that will be later used within the body of the document.} % Prepend this text to the nomenclature

% \printnomenclature % Output the nomenclature

% %----------------------------------------------------------------------------------------
% %	GREEK ALPHABET
% % 	Originally from https://gitlab.com/jim.hefferon/linear-algebra
% %----------------------------------------------------------------------------------------

% \vspace{1cm}

% {\usekomafont{chapter}Greek Letters with Pronunciations} \\[2ex]
% \begin{center}
% 	\newcommand{\pronounced}[1]{\hspace*{.2em}\small\textit{#1}}
% 	\begin{tabular}{l l @{\hspace*{3em}} l l}
% 		\toprule
% 		Character & Name & Character & Name \\ 
% 		\midrule
% 		$\alpha$ & alpha \pronounced{AL-fuh} & $\nu$ & nu \pronounced{NEW} \\
% 		$\beta$ & beta \pronounced{BAY-tuh} & $\xi$, $\Xi$ & xi \pronounced{KSIGH} \\ 
% 		$\gamma$, $\Gamma$ & gamma \pronounced{GAM-muh} & o & omicron \pronounced{OM-uh-CRON} \\
% 		$\delta$, $\Delta$ & delta \pronounced{DEL-tuh} & $\pi$, $\Pi$ & pi \pronounced{PIE} \\
% 		$\epsilon$ & epsilon \pronounced{EP-suh-lon} & $\rho$ & rho \pronounced{ROW} \\
% 		$\zeta$ & zeta \pronounced{ZAY-tuh} & $\sigma$, $\Sigma$ & sigma \pronounced{SIG-muh} \\
% 		$\eta$ & eta \pronounced{AY-tuh} & $\tau$ & tau \pronounced{TOW (as in cow)} \\
% 		$\theta$, $\Theta$ & theta \pronounced{THAY-tuh} & $\upsilon$, $\Upsilon$ & upsilon \pronounced{OOP-suh-LON} \\
% 		$\iota$ & iota \pronounced{eye-OH-tuh} & $\phi$, $\Phi$ & phi \pronounced{FEE, or FI (as in hi)} \\
% 		$\kappa$ & kappa \pronounced{KAP-uh} & $\chi$ & chi \pronounced{KI (as in hi)} \\
% 		$\lambda$, $\Lambda$ & lambda \pronounced{LAM-duh} & $\psi$, $\Psi$ & psi \pronounced{SIGH, or PSIGH} \\
% 		$\mu$ & mu \pronounced{MEW} & $\omega$, $\Omega$ & omega \pronounced{oh-MAY-guh} \\
% 		\bottomrule
% 	\end{tabular} \\[1.5ex]
% 	Capitals shown are the ones that differ from Roman capitals.
% \end{center}

% %----------------------------------------------------------------------------------------
% %	GLOSSARY
% %----------------------------------------------------------------------------------------

% % The glossary needs to be compiled on the command line with 'makeglossaries main' from the template directory

\setglossarystyle{listgroup} % Set the style of the glossary (see https://en.wikibooks.org/wiki/LaTeX/Glossary for a reference)
\printglossary[title=Special Terms, toctitle=List of Terms] % Output the glossary, 'title' is the chapter heading for the glossary, toctitle is the table of contents heading

% %----------------------------------------------------------------------------------------
% %	INDEX
% %----------------------------------------------------------------------------------------

% % The index needs to be compiled on the command line with 'makeindex main' from the template directory

% \printindex % Output the index

% %----------------------------------------------------------------------------------------
% %	BACK COVER
% %----------------------------------------------------------------------------------------

% % If you have a PDF/image file that you want to use as a back cover, uncomment the following lines

% %\clearpage
% %\thispagestyle{empty}
% %\null%
% %\clearpage
% %\includepdf{cover-back.pdf}

% %----------------------------------------------------------------------------------------

\end{document}
