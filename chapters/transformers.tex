\setchapterstyle{kao}
\setchapterpreamble[u]{\margintoc}
\chapter{Studying shallow structure in transformer models}
\labch{transformers}

Tree-based models need carefully hand-annotated data to be trained. Alternative methods such as recurrent neural network \parencite{hochreiter_97, cho_14} or \textsc{Bert} \parencite{devlin_19} have gained increased popularity within the community. Indeed, these methods, unlike tree-based models, require only raw text as input. On the other hand, as many results suggest \parencite{linzen_16, jawahar_19, clark_19}, these new models acquire some sort of tree structure.
% \sidenote{We deep dive on the structure learned by transfomers model in \refch{transformers}.}.

% Recent transformer architectures have gained increased popularity within the community. Contrary to tree-based models, they do not need carefully hand-annotated data to be trained. On the other hand, as many results suggest, these new models acquire some sort of tree structure. 

Transformers update each token hidden simultaneously through a fixed number of layers. Yet the role of these layers and how they process information is not fully understood. I formulate the hypothesis that the distinct layers do not encode specific surface, syntactic nor semantic functions but rather that such information emerges through the iterative application of layers. To better study the transformation of token representations across layers, I proposed a variant of ALBERT \parencite{simoulin_2021b}. This model implements the key specificity of weights tying across layers, but also dynamically adapts the number of layers applied to each token. I analyze token transformation across the network depth. In particular, I study how iterations are distributed given the token dependency types. I showed that tokens do not require the same amount of iterations and that difficult or crucial tokens for the task are subject to more iterations.

\section{Analyzing transformers shallow structure}

\section{Model architecture}

\section{Analysis of the pre-training}

\section{Application on downstream tasks}
