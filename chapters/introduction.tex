\setchapterpreamble[u]{\margintoc}
\chapter{Introduction}
\labch{intro}

\cleanchapterquote{Language is a process of free creation; its laws and principles are fixed, but the manner in which the principles of generation are used is free and infinitely varied. Even the interpretation and use of words involves a process of free creation.}{Noam Chomsky}{For reasons of state, 1973}
% p 402. Chapter 9, Language and Freedom
% \bcomment{}{this quote is suspect}
% QUOI

%\bcomment{Suggestion (sth along these lines)}{Language has structure. Sentences have discrete structures, such as trees. These are fundamental hypothesis of linguistic theory since its early days. Recently, in computational linguistics emerged a new family of methods that model language with vectors. First emerged word embeddings \cite{mikolov_13b}, then deep learning truly changed the field and it becomes natural to model sentences or even longer texts with vector representations. In this context, this dissertation is about\ldots}

% An axiom, postulate, or assumption is a statement that is taken to be true, to serve as a premise or starting point for further reasoning and arguments. 
% The structure of language is a fundamental axiom, serving as a premise in linguistic theory.
% The hypothesis that language has a structure have been the cornerstones of linguistic theory since its early days.
% These are fundamental hypothesis of linguistic theory since its early days.
% There is this strong hypothesis in computational linguistics that language has a recursive structure . This hypothesis 

Linguistic theory is founded on the hypothesis that language has a structure. In computational linguistics, a strong premise is that this structure is recursive \parencite{chomsky_56} and, in the specific case of sentences, this structure forms a tree. These premises are the cornerstone of linguistic theory. Recently, a new family of methods truly changed the field of computational linguistics by modeling language with vectors. First, word embeddings emerged \parencite{mikolov_13a, mikolov_13b} and, as deep learning gained momentum, it soon became natural to model sentences or even longer texts with vector representations \parencite{hochreiter_97, cho_14}. In this context, this dissertation is about creating sentence embeddings through the composition of lexical units. Text representation is at the core of natural language processing (NLP), which develops automatic methods for inferring related attributes from those representations. Attributes can take many forms: a given class in a classification problem, the answer to a question, a list of documents with similar semantic content, or a summary of the input text. In recent years, the representation methods for text have developed significantly. Contrary to formal linguistic frameworks, which derive syntactic and semantic properties from expert rules, these methods derive representations by exploiting the implicit patterns within vast corpora. These methods are grounded on two foundational hypotheses: the distributional hypothesis to build word representations given their context and the compositionality principle to combine those words into sentence representations.


% \bcomment{}{distributional hyp = words and their context only}

This dissertation focuses on sentences and the methods to encode them. Specifically, \textit{(i)} how layouts identified by distributional methods from vast corpora relate to linguistic structures and, respectively, \textit{(ii)} how we can efficiently infuse linguistic biases in neural architectures to drive the composition function learned by self-supervised sentence embedding methods.

This section introduces the study by first discussing the applications of sentence embedding methods, the background, and context, followed by the research problem, the research aims, objectives and questions, the significance, and finally, the limitations.

% POURQUOI

\section{Background to the study}
\labsec{introduction:backgroud}

Embedding a sentence consists of assigning it to a static, fixed-length, real-valued vector, which captures its meaning. It is important to emphasize the distinction between sentence embedding and any sentence vector representation, for example, intermediate representations from neural networks. The majority of modern NLP methods works end-to-end: the intermediate representations and the inference of attributes from those representations are part of a unified process. In such a case, it is impossible to separate the representations from the final model outputs. The representations, therefore, depend on the attribute we seek to predict and will most likely only capture the information relevant for this prediction. In the context of sentence embedding, the representation of the input text and the inference of related attributes are two explicitly disconnected steps. Therefore, sentence embedding should capture an exhaustive perspective of the text input meaning as we may use them to predict a large variety of attributes. Sentence embedding methods should be highly generic, and their conception should be independent of their later use.

% \bcomment{what properties should sentence embeddings have}{sentences with similar meanings should have similar embeddings ? }{this is not stated explictly in the thesis}
% As a result of this transformation, we can compare sentence semantic characteristics directly in the sentence embedding space using standard mathematical operators. For example, it is straightforward to define a notion of mathematical distance over the vector space and to characterize sentences with close meanings as sentences for which embedding vectors are close given this distance.
% It is also possible to use sentence embedding as features for more complex models, inferring relations such as entailment between sentence pairs. Sentence embeddings are, therefore, key for a number of applications such as search engines, information retrieval, text mining, and documents clustering.
We can divide the properties we expect from sentence embeddings into two categories:
\begin{itemize}
    \item First, the notion of semantic distance in the original sentence space should be reflected in the representation space. In the sentence embedding space, it is straightforward to define a notion of mathematical distance over the vector space. Therefore, we can use standard mathematical operators to compare semantic sentence characteristics directly in the sentence embedding space. Sentences with close meanings should be mapped to close embedding vectors.
    \item Second, we expect sentence embeddings to fully capture the meaning and general characteristics—such as the sentence length or the main verb tense—of the original sentence. It should be possible to extract some specific information from the embedding vector using statistical methods. 
\end{itemize}
Therefore, sentence embeddings are essential for many unsupervised applications such as search engines, information retrieval, text mining, and documents clustering. It is also possible to use sentence embeddings as features for more supervised models, inferring relationships such as entailment between sentence pairs. 

% COMMENT

\section{Research problem}

Embedding sentences is an active subject of research, including the development of self-supervised training objectives, training datasets, evaluation benchmarks, or the release of models as open-source contributions.

Building standalone sentence embeddings is specifically hard, as an infinite number of valid sentences exist. Compositional semantics state that the meaning of a phrase is determined by combining the meanings of its sub-phrases. Models, therefore, need to compose text units, given a syntactic structure, into global semantic embeddings. However, many contributions rely on standard encoder architectures and do not question the composition mechanisms transforming text units into a global sentence representation.

Thus, sentence embedding methods present pitfalls that are common to many domains in NLP: lack of robustness toward out-of-domain generalization, shallow pattern matching rather than compositional knowledge, the requirement for large training datasets, or over-parametrization. 

\section{Research aims, objectives, and questions}

This study aims at improving sentence encoder compositional abilities. We seek to leverage both the integration of linguistic biases into neural network architectures as well as the scaling of these models and their training setup.
We define the following research objectives:

\begin{enumerate}
    \item Develop efficient methods to integrate linguistic biases into neural networks;
    \item Evaluate the effectiveness of these strategies and approaches;
    \item Compare and contrast these strategies and approaches in terms of their strengths and weaknesses.
\end{enumerate}

% \bcomment{too many bullets around}{make paragraphs} The dissertation will address the following specific research questions:
% \begin{enumerate}
%     \item Can we efficiently introduce linguistic biases within neural network architectures? 
%     \item Does self-supervised training induce structure into neural architecture?
%     \item Can we induce specific compositional abilities through neural architectures?
%     \item Can we balance the lack of linguistic insights with larger models and larger training datasets?
% \end{enumerate}

The dissertation studies sentence embeddings and their relation with sentence structures. Below, we detail the research questions we investigate. The three first questions focus on the methods to efficiently induce linguistically driven insights within neural network composition functions. The last question asks about the benefits of such insights regarding the generalization power of neural networks for automatic language processing.

\paragraph{Can we efficiently introduce linguistic biases within neural network architectures?} The recursive structure of language is a strong hypothesis in computational linguistics \parencite{chomsky_56}. Thus, computing sentence semantic representations traditionally calls for a recursive compositional function whose structure is tree-shaped. In contrast, recent deep learning architectures—such as recurrent neural networks \parencite{hochreiter_97, cho_14} or transformers \parencite{vaswani_17}—encode text without explicit hierarchical composition. The first research question focuses on bridging the gap between these two paradigms: we explore the feasibility of explicitly integrating linguistic priors within neural architectures to compose semantic representations with a hierarchical structure.

\paragraph{Does self-supervised training induce structure into neural architecture?}

Our second research question has the same objective—integrating linguistic priors within neural architectures—but different means. This time we are not focusing on the architectures of neural networks but rather the training methods. We explore the possibility of inducing latent structure within the function that neural networks operates to compose lexical units into sentence representations.

\paragraph{Can we induce specific compositional abilities through neural architectures?}

We explore the possibility of exploiting the model and training dataset size to induce linguistic structure into neural networks. While the size of the language model in natural language processing is steadily increasing \parencite{devlin_19, brown_20}, we investigate how such approaches can compensate for the lack of linguistically based insights.

\paragraph{Can we balance the lack of linguistic insights with larger models and larger training datasets?}

Finally, we investigate the role of structure in building more robust neural network architectures. By influencing the semantic composition of neural networks, we aim to improve their compositional and generalization abilities.

\section{Significance}

% \bcomment{You might also remind the reader that you are funded by a company and that these real world achievements are also motivated by the desire of acheiving industrial contributions}{}

We expect this study to contribute to the body of knowledge on sentence embeddings and neural model architectures to encode text. The publications and the ongoing experiments will contribute to the academic effort in building more robust statistical models by incorporating language biases and approaches for scaling model training. 

This thesis is funded by Quantmetry\sidenote{\url{https://www.quantmetry.com/}}, a French pioneering consulting firm working on end-to-end AI projects—from strategy to industrialization. As such, this work is also motivated by the desire of achieving industrial contributions and ready-to-use tools available for real-world applications. Besides empirical research, a large portion of this work is thus released as open-source contributions. Resources include pre-trained language models for English and French\sidenote{\url{https://huggingface.co/asi/gpt-fr-cased-base}}, training and evaluation datasets, as well as associated scripts to reproduce the results. Finally, we released a code for recursive models under a library called PyTree\sidenote{\url{https://github.com/AntoineSimoulin/pytree}}. The library was distinguished and listed among the winners of the PyTorch Hackathon 2021. We hope this empirical work and the resources will provide real-world value for organizations in a field in which knowledge and methods are undergoing rapid and continuous evolution.

\section{Limitations}

%  \bcomment{and may, in some cases,}{it is unknown to us if it} extend to paragraphs
Our study is limited to sentences, but we hypothesize that it may, in some cases, extend to paragraphs. However, the major part of our work will not apply to longer chunks of text. Although we seek to propose methods applicable to various languages, the study focuses mainly on English and French, and some experiments may be difficult to reproduce in low-resource languages. Indeed, we make use of specific training and evaluation resources.

Our study proposes efficient methods to introduce linguistic biases into neural models and better characterize model compositional behavior. Such approaches appear promising to avoid unwanted behavior for real-world applications. However, our study also underlines the long road ahead to fully realize the promises of current language models.

\section{Contributions and Outline}

\subsection{Jointly learning model structure and compositional operations}

First, we focus on tree-structured neural networks, which naturally encode the structure of language. For each sentence, the network computes text units following a syntactic tree, starting from the leaf nodes up to the root. However, such models suffer from practical constraints that limit their application. In particular, tree-based models not only require raw text as input but also the sentence structure in the form of a parse tree. Such structure may be tedious because it requires manual annotations and external parsers. To overcome such limitations, we formulated a novel tree-based model that learns its composition function together with its structure. The model includes two modules, a biaffine graph parser, and a Tree-LSTM. The parsing and the composition functions are explicitly connected and, therefore, learned jointly. The method differs from previous work as the representation is not computed from the whole forest of potential trees. Moreover, training the full model directly does not require supervision from an explicit parsing objective. The model outperforms tree-based models relying on external parsers on downstream tasks. In some configurations, it is even competitive with BERT-base.

\subsection{Studying shallow structure in transformer models}

Recent transformer architectures have gained increased popularity within the community. Contrary to tree-based models, they do not need carefully hand-annotated data to be trained. On the other hand, as many results suggest, these new models acquire some sort of hierarchical structure. Transformers update each token hidden simultaneously through a fixed number of layers. Yet, the role of these layers and how they process information is not fully understood. We formulate the hypothesis that the distinct layers do not encode specific surface, syntactic nor semantic functions but rather that such information emerges through the iterative application of layers. To better study the transformation of token representations across layers, we propose a variant of ALBERT \parencite{simoulin_2021b}. This model implements the key specificity of weights tying across layers but also dynamically adapts the number of layers applied to each token. We analyze token transformation across the network depth. In particular, we study how iterations are distributed given the token dependency types. We show that tokens do not require the same amount of iterations and that difficult or crucial tokens for the task are subject to more iterations.

\subsection{Characterizing compositional properties of neural architectures}

While transformers show outstanding performances on many NLP benchmarks, they also have some linguistic limitations. In particular regarding their ability to generalize outside their training range and to learn elementary composition rules. The benchmark COGS \parencite{kim_20} for example, highlights that deep learning models struggle to generalize to longer sequences or sentences with deeper level of recursion than seen during training. Following our work on integrating the structure into neural architecture, we aim at better characterizing how the model structure may affect its degree of compositionality. This work is currently in an experimentation phase. We are building an evaluation setup with arithmetic expressions containing specific properties. We train various models on specific subsets and observe how models generalize outside their domain. Specifically, we compare models with varying structural constraints, such as sequential, recursive, or unstructured models.

\subsection{Training sentence embedding models using a discriminative objective}

Inspired by linguistic insights, we assume structure is crucial to building consistent representations. We indeed expect sentence meaning to be a function of both syntax and semantic aspects. In that regard, we propose a self-supervised method that builds sentence embeddings from the combination of diverse explicit syntactic structures of a sentence. The novelty consists in jointly learning structured models in a contrastive multi-view framework that induces an explicit interaction between models during the training phase. We pre-train various models using a contrastive objective with a 40 million sentences corpus. We then evaluate our model on sentence embedding benchmarks and obtain state-of-the-art results. In particular on tasks that are expected, by hypothesis, to be more sensitive to sentence structure. We relate the development, training, and release of large, state-of-the-art, sentence embedding models. We use a similar contrastive objective and train models on a 1 billion sentences corpus. We develop specific evaluation benchmarks for sentence embeddings and obtain state-of-the-art results.

\subsection{Training the first large incremental language model for French}

As observed in \textcite{linzen_2020}, deep neural networks have exceptional grammatical competencies. For example, GPT-2 generates correct text with plural and long-distance agreement despite any prior linguistic knowledge. Such agreements are determined by abstract structures and not just the linear order of words. Surprisingly, models can learn such specific linguistic patterns (subject-verb, noun-adverb, verb-verb) with no prior information about linguistic theory. Within our laboratory, we led the project to train the first large language model in French \parencite{simoulin_2021c}. We obtained a dedicated computation grant for the public French HPC computer Jean Zay. The model, equivalent to GPT-2 in English, contains more than 1 billion parameters. We build a dedicated training corpus and parallelize the training between multiple nodes and compute units. We released the model in Open-Source for research and business application purposes. 

\subsection{Outline of the dissertation}

We organize the dissertation into three parts. 

Part I provides the necessary background in meaning representation, sentence embeddings, and neural model encoders. \refch{meaning} introduces meaning representations. \refch{training} reviews the architecture of standard encoders to compose words into sentence embeddings, the training objective, and evaluation methods.

Part II aims at improving the compositional properties of language models and their ability to generalize outside their training domain. We aim to integrate the recursive property of language within neural models and design architectures based on linguistic theory. \refch{structure-scale} proposes a self-supervised method that builds sentence embeddings from the combination of diverse explicit syntactic structures of a sentence. However, the tree-structured encoders require heavily structured data to compute the semantic representations. In \refch{tree}, we propose to overcome this limitation by proposing an architecture inducing trees from raw text and computing semantic representations along with the inferred structure. \refch{transformers} makes the parallel with transformers and sequential or tree-structured models. We interpret transformers as structured neural networks and layers as operations on fully connected graphs. We finally compare all models in \refch{arithmetics} by proposing an in-depth evaluation of their compositional properties.

Part III focuses on training and sharing models at scale. Indeed, the preparation of massive corpora, the training, and the use of large architectures are key for the performance of such models. \refch{1B} presents an attempt to train state-of-the-art sentence embedding models on a very large corpus. \refch{generative} proposes to train the first large generative pre-trained model in French.

\subsection{Publications}

This dissertation contains some contributions that we previously published and presented at conferences.

\begin{itemize}
    \item \refch{structure-scale} is an extended version of an article published in EACL Student Research Workshop 2021 \parencite{simoulin_2021a}. We  open-sourced the code developed for recursive models under a library called PyTree\sidenote{\url{https://github.com/AntoineSimoulin/pytree}}. The library was distinguished and listed among the winners of the PyTorch Hackathon 2021;
    \item \refch{tree} presents work currently under submission;
    \item \refch{transformers} is an extended version of an article published in ACL Research Student Workshop 2021 \parencite{simoulin_2021b};
    \item \refch{arithmetics} presents original unpublished work as well as work currently under submission;
    \item \refch{1B} relates the development of state-of-the-art sentence embedding models as part of the project \textit{Train the Best Sentence Embedding Model Ever with 1B Training; Pairs}. This project took place during the \textit{Community week using JAX/Flax for NLP \& CV} organized by Hugging Face. Our project was among the competition winners and received an honorable mention;
    \item \refch{generative} is an extended version of an article published in TALN 2021 \parencite{simoulin_2021c}.
\end{itemize}

%\footnote{\url{https://discuss.huggingface.co/t/open-to-the-community-community-week-using-jax-flax-for-nlp-cv/7104}}
%\footnote{\url{https://discuss.huggingface.co/t/train-the-best-sentence-embedding-model-ever-with-1b-training-pairs/7354}}

The code used for all experiments carried out for this dissertation, the pre-trained models, the evaluation and training datasets have been made publicly available as free softwares through the following repositories:

\begin{itemize}
    \item \url{https://github.com/AntoineSimoulin/pytree}
    \item \url{https://github.com/AntoineSimoulin/gpt-fr}
    \item \url{https://huggingface.co/datasets/asi/wikitext_fr}
    \item \url{https://huggingface.co/asi/gpt-fr-cased-base}
    \item \url{https://huggingface.co/asi/gpt-fr-cased-small}
\end{itemize}
