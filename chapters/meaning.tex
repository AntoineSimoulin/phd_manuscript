% \setchapterstyle{kao}
\setchapterpreamble[u]{\margintoc}
\chapter{Defining sentence meaning}
\labch{meaning}

\cleanchapterquote{Give orange me give eat orange me eat orange give me eat orange give me you.}{Nim Chimpsky}{Male chimpanzee}

\section{The meaning of meaning}

According to \textcite{fromkin_2017}, we can define the meaning of a sentence given our ability to judge it as true or false. Evaluating this boolean value usually requires some context: given other sentences, we deduce whether the sentence is true or false. We may use make the deduction step by step using inductive relations. We define entailment as follows: if sentence A is true, then sentence B is also true. We define paraphrases as two sentences A and B that entail each other.

However, language is ambiguous. For example, given the sentence “The boy saw the man with a telescope”, we may evaluate the sentence are true or false given the same context. Indeed, we may decompose the sentence using distinct structures.

\section{Formal semantic representations}
\labsec{meaning:formal}

\section{Distributional semantic representations}
\labsec{meaning:distributional}

Distributional semantic, vector space